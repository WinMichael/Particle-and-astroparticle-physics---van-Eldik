\chapter{Covariant description of relativistic particles}
\section{Non-relativistic quantum mechanics}
Recall the energy-momentum relation
\begin{align}
    E = \frac{1}{2m} \qty|\vec p|^2\,.
\end{align}
By identifying $E \ra \ti \hbar \pa_t$, $\vec p \ra -\ti \hbar \vec \grad$, we get an operator equation, i.e. the Schrödinger equation
\begin{align}\label{eq:schroedinger}
    \boxed{\qty(\ti \pa_t + \vec \grad ^2 \frac{1}{2m}) \phi \qty(\vec x, t) = 0}
\end{align}
\begin{compactitem}
    \item[with] $\phi \qty(\vec x, t) \in \mb C$ as the one-particle wave function.
\end{compactitem}
We can get the statistical interpretation via $\qty | \phi | \dd[3]{x}$ $\hat =$ probability to find particle in volume $\dd[3]{x}$ at time $t$ and we can obtain the localisation probability density
\begin{align}
    \ra \quad \rho = \rho \qty (\vec x, t) \defi \qty|\phi \qty(\vec x, t)|^2\,.
\end{align}
\begin{multicols}{2}
    \begin{center}
        \begin{tikzpicture}
            \draw (0.,-0.05) to [curve through = {(1.,0.) .. (1.2,0.8) .. (0.4,1.6) .. (0.,2.) .. (-0.5,0.75)}] (0.,-0.05);
            \foreach \i in {-0.2,0.2,0.6}
                \draw[-latex] (\i,1.2-\i) to [curve through = {(\i+0.2,1.8-\i)}] (\i+0.1,2.4);
            \foreach \i in {-0.2,-0.5,-0.8}
                \draw[-latex] (\i,-1.4-\i) to [curve through = {(\i-0.2,-0.6-\i)}] (\i,-0.2-\i);
            \node at (1.,2.6) {$\vec j$};
            \node at (0,0.5) {$\rho$};
            \node at (1.4,0) {$\dd[3]{x}$};
        \end{tikzpicture}
    \end{center}
    From electrodynamics we get the continuity equation:
    \begin{align}\label{eq:continuity}
        \boxed{\pa_t \rho + \vec \grad \vec j = 0}
    \end{align}
    \begin{compactitem}
        \item[with] $\vec j \qty(\vec x, t)$ as the \tb{probability density current}.
    \end{compactitem}
\end{multicols}
How does $\vec j$ depend on the wave function $\phi$?
\begin{align*}
    \qty(-\ti \phi^*) \cdot (\cref{eq:schroedinger}) & = \phi^*\pa_t \phi - \frac{\ti}{2m} \phi^* \vec \grad ^2 \phi = 0 \\
    \qty(- \ti \phi) \cdot (\cref{eq:schroedinger})^* & = - \phi \pa_t \phi^* - \frac{\ti}{2m} \phi \vec \grad ^2 \phi^* = 0
\end{align*}
By subtracting these two equations we get
\begin{align}
    \Ra \quad \underbrace{\qty(\phi^*\pa_t \phi + \phi \pa_t \phi^*)}_{\pa_t \qty(\phi^*\phi) = \pa_t \qty|\phi|^2} - \frac{\ti}{2m} \underbrace{\qty(\phi^* \vec \grad ^2 \phi - \phi \vec \grad^2 \phi^*)}_{\vec \grad \qty(\phi^*\vec \grad \phi - \phi \vec \grad \phi^*)} = 0
\end{align}
and by using the continuity \cref{eq:continuity} we get an explicit expression for the probability density current
\begin{align}
    \boxed{ \vec j = - \frac{\ti}{2m}\qty(\phi^*\vec \grad \phi - \phi \vec \grad \phi^*) }\,.
\end{align}
The probability density current, based on the Schrödinger equation, is not covariant. This means that the Schrödinger equation treats time and space different.

\section{Relativistic particles: Klein-Gordon equation}
Now we use the \tb{relativistic} energy-momentum relation
\begin{align}
    E^2 = \vec p^2 +m^2 \qquad E \ra \ti \hbar \pa_t, \quad \vec p \ra \ti \hbar \vec \grad\nonumber \\
    \Ra \quad \boxed{\qty(\pa_t^2 - \Lap +m^2) \phi \qty(\vec x, t) = 0}\,.
\end{align}
This is the so-called \tb{Klein-Gordon equation}. By using four vector notation
\begin{align}
    p^\mu \ra \mqty(E \\ \vec p) \ra \mqty(\ti \pa_t \\ - \ti \vec \grad) \ra \ti \pa^\mu \quad \Ra \quad \boxed{\qty(\pa_\mu\pa^\mu + m^2) \phi = 0}\,,
\end{align}
which denotes the Klein-Gordon equation in covariant form. Since the scalar product and the mass are Lorentz invariant, this KG equation is too.\\
$\ra$ Klein-Gordon equation describes spin-0 particles.

The continuity equation thus is
\begin{align}
    \pa_t \rho + \vec \grad \vec j = 0 \quad \ra \quad \boxed{\pa_\mu j^\mu = 0}\,.
\end{align}
How do $\rho$ and $\vec j$ depend on $\phi$?
\begin{align}\label{eq:four_density_KGe}
    \rho = \ti \qty(\phi^*\pa_t \phi - \phi \pa_t \phi^*), \quad \vec j = - \ti \qty(\phi^* \vec\grad \phi - \phi  \vec \grad \phi^*)
\end{align}
and the for vector current is
\begin{align}
    \boxed{j^\mu = \ti \qty(\phi^* \pa^\mu \phi - \phi\pa^\mu \phi^*)}\,.
\end{align}
The fundamental free-particle solution for the KGe is given by
\begin{align}\begin{split}
    \phi \qty(\vec x,t) \equiv \phi (\fvec x) & = N \cdot \exp(\ti \qty(\vec p\vec x - Et))\\
    & = N \exp(- \ti p_\mu x^\mu) = N \exp(-\ti \fvec p\fvec x)
\end{split}\end{align}
\begin{compactitem}
    \item[with] $N$ as the normalisation.
\end{compactitem}
By inserting this into the probability density current (\cref{eq:four_density_KGe}), we get
\begin{align}
    \boxed{j^\mu = 2 p^\mu \qty|N|^2}
\end{align}
for free particles. Note that espacially $\rho = 2 E \qty|N|^2$.
\begin{itemize}[$\ra$]
    \item Localisation probability $\rho \dd[3]{x}$, but under Lorentz boost $\dd[3]{x} \ra \frac 1\gamma \dd[3]{x}$, since $\rho \propto E \propto \gamma$. This effect is compensated for the probability density current $\vec j \propto \vec p$.
    \item Particle current $\vec j$ in the direction of the particle momentum $\vec p$.
\end{itemize}
What are the eigenvalues of the free-particle solution?
\begin{align}
    &\qty(\pa_\mu \pa^\mu + m^2) \exp(-\ti \fvec p\fvec x) = 0 \nonumber \\
    &\Ra\quad \qty(\qty(-\ti)^2 p_\mu p^\mu + m^2) \exp(-\ti \fvec p \fvec x) = 0 \nonumber \\
    &\Ra\quad - p_\mu p^\mu + m^2 = 0, \quad p_\mu p^\mu = m^2 = E^2 - \vec p^2 \nonumber \\
    &\Ra\quad \boxed{E = \pm \sqrt{\vec p^2 + m^2}}
\end{align}
So we get two solutions for the particle energy. However, the negative solution $E < 0 \Ra \rho < 0$ is unphysical! There is one way out, the Feynman-Stückelberg approach. Here we interpret $\rho$ as a charge density.
\begin{itemize}
    \item[$\ra$] Can be negative, since particles can have negative charge.
    \item[$\ra$] Interpret $j^\mu$ as charge current.
    \item[$\Ra$] $j^\mu \ra j'^\mu = Q j^\mu$ with $Q$ as particle charge.
\end{itemize}

\begin{example}
    electron, $ Q= -\mr e$
    
    By assuming an electron with spin 0, we get
    \begin{align}
        j^\mu \qty(\mr e^-) = - \ti \mr e \qty(\phi^* \pa^\mu \phi - \phi \pa^\mu \phi^*)\,.
    \end{align}
    Free particle solution ($E>0$):
    \begin{align}
        j^\mu \qty(\mr e^-) = -2 \mr e p^\mu \qty|N|^2 \ra - 2 \mr e \qty|N|^2 \mqty(E \\ \vec p)
    \end{align}
    For $E<0$: Consider anti muon with $E>0$
    \begin{align}
        \fvec j \qty(\mu^+) = 2 \mr e \fvec p \qty|N|^2 = 2 \mr e \qty|N|^2 \mqty(E \\ \vec p) = - 2 \mr e \qty|N|^2 \mqty(-E \\ -\vec p)  = - \fvec j \qty(\mu^-)\,.
    \end{align}
    So we get a muon with $E<0$, that moves backwards.
\end{example}
\begin{itemize}
    \item[$\ra$] solution with $E<0$ can be used to describe antiparticles with $E' = -E$
\end{itemize}
\begin{align*}
    \begin{tikzpicture}
        \begin{feynman}
        \vertex (a1) {$\mr e^-$};
        \vertex [below right = 2.5cm of a1] (b1);
        \vertex [above right = 2cm of b1] (a2) {$\mr e^-$};
        \vertex [below = 2cm of b1] (c1);
        \vertex [below left = 2cm of c1] (d1) {$\upmu^+$};
        \vertex [below right = 2cm of c1] (d2) {$\upmu^+$};
        \vertex [right = 1cm of d1] (j1);
        \vertex [left = 1cm of d2] (j2);
        \vertex [right = 1cm of a1] (i1);
        \vertex [left = 1cm of a2] (i2);
        \vertex [below = 1cm of b1] (b2);
        \vertex [right = 2.5cm of b2] (e) {$\hat =$};
        \diagram*{
        (a1) -- [fermion, momentum' = $\fvec p_A$] (b1) -- [fermion, momentum' = $\fvec p_C$] (a2);
        (b1) -- [photon, edge label' = $\upgamma$] (c1);
        (d1) -- [anti fermion, momentum = $\fvec p_B$] (c1) -- [anti fermion, momentum = $\fvec p_D$] (d2);
        (j1) -- [draw = none, half left, momentum = $\ $, edge label' = $j^\mu \qty(\upmu^+)$] (j2);
        (i1) -- [draw = none, half right, momentum' = $\ $, edge label = $j^\mu \qty(\mr e^-)$] (i2);
        };
        \vertex [right = 6cm of a1] (z1) {$\mr e^-$};
        \vertex [below right = 2.5cm of z1] (y1);
        \vertex [above right = 2cm of y1] (z2) {$\mr e^-$};
        \vertex [below = 2cm of y1] (x1);
        \vertex [below left = 2cm of x1] (w1) {$\upmu^-$};
        \vertex [below right = 2cm of x1] (w2) {$\upmu^-$};
        \vertex [right = 1cm of w1] (k1);
        \vertex [left = 1cm of w2] (k2);
        \vertex [right = 1cm of z1] (l1);
        \vertex [left = 1cm of z2] (l2);
        \diagram*{
        (z1) -- [fermion, momentum' = $\fvec p_A$] (y1) -- [fermion, momentum' = $\fvec p_C$] (z2);
        (y1) -- [photon, edge label' = $\upgamma$] (x1);
        (w1) -- [fermion, rmomentum = $-\fvec p_B$] (x1) -- [fermion, rmomentum = $-\fvec p_D$] (w2);
        (k2) -- [draw = none, half right, momentum' = $\ $, edge label = $j^\mu \qty(\upmu^-)$] (k1);
        (l1) -- [draw = none, half right, momentum' = $\ $, edge label = $j^\mu \qty(\mr e^-)$] (l2);
        };
        \end{feynman}
    \end{tikzpicture}
\end{align*}
Consequence: Can use particle states with $p^\mu \ra - p^\mu$ for description of antiparticles.

\section{Crossing symmetry}
The description of scattering processes is highly symmetric under the exchange of space and time: This originates in the fact that wave equations treat time and space the same way.
\begin{example}
    $\mr e^- \upmu^-$ scattering in QED
    
    By exchange of time and space, the Feynman diagram changes as:
    \begin{align*}
        \begin{tikzpicture}
            \begin{feynman}
            \vertex (a1) {$\mr e^-$};
            \vertex [below right = 2.5cm of a1] (b1);
            \vertex [above right = 2cm of b1] (a2) {$\mr e^-$};
            \vertex [below = 2cm of b1] (c1);
            \vertex [below left = 2cm of c1] (d1) {$\upmu^+$};
            \vertex [below right = 2cm of c1] (d2) {$\upmu^+$};
            \vertex [right = 1cm of d1] (j1);
            \vertex [left = 1cm of d2] (j2);
            \vertex [right = 1cm of a1] (i1);
            \vertex [left = 1cm of a2] (i2);
            \vertex [below = 1cm of b1] (b2);
            \vertex [right = 3cm of b2] (e) {$\hat =$};
            \diagram*{
            (a1) -- [fermion] (b1) -- [fermion] (a2);
            (b1) -- [photon, edge label' = $\upgamma$] (c1);
            (d1) -- [anti fermion] (c1) -- [anti fermion] (d2);
            (j1) -- [draw = none, half left, momentum = $\ $, edge label' = $j^\mu \qty(\upmu^+)$] (j2);
            (i1) -- [draw = none, half right, momentum' = $\ $, edge label = $j^\mu \qty(\mr e^-)$] (i2);
            };
            \vertex [right = 3cm of e] (y1);
            \vertex [below left = 2cm of y1] (z1) {$\mr e^-$};
            \vertex [above left = 2cm of y1] (z2) {$\mr e^+$};
            \vertex [right = 2cm of y1] (x1);
            \vertex [above right = 2cm of x1] (w1) {$\upmu^-$};
            \vertex [below right = 2cm of x1] (w2) {$\upmu^+$};
            \vertex [below = 1cm of w1] (k1);
            \vertex [above = 1cm of w2] (k2);
            \vertex [above = 1cm of z1] (l1);
            \vertex [below = 1cm of z2] (l2);
            \diagram*{
            (z1) -- [fermion] (y1) -- [fermion] (z2);
            (y1) -- [photon, edge label' = $\upgamma$] (x1);
            (w1) -- [anti fermion] (x1) -- [anti fermion] (w2);
            (k2) -- [draw = none, half left, momentum = $\ $, edge label' = $j^\mu \qty(\upmu^-)$] (k1);
            (l1) -- [draw = none, half right, momentum' = $\ $, edge label = $j^\mu \qty(\mr e^-)$] (l2);
            };
            \end{feynman}
        \end{tikzpicture}
    \end{align*}
    This is equivalent to a counter clockwise rotation by \SI{90}{\degree}. The resulting diagram represents $\mr e^+ \mr e^-$ annihilation followed by $\mu^+\mu^-$ creation.
\end{example}
By exchanging the incoming anti muon by an outgoing muon in the $\upmu^+\upmu^-$ annihilation, we get the diagram:
\begin{center}
    \begin{tikzpicture}
        \begin{feynman}
        \vertex (y1);
        \vertex [below = 1.8cm of y1] (s);
        \vertex [right = 0.7cm of s] (z1) {$\upmu^-$};
        \vertex [above left = 2cm of y1] (z2) {$\upmu^-$};
        \vertex [right = 2cm of y1] (x1);
        \vertex [above right = 2cm of x1] (w1) {$\mr e^-$};
        \vertex [below right = 2cm of x1] (w2) {$\mr e^+$};
        \diagram*{
        (z1) -- [anti fermion] (y1) -- [anti fermion] (z2);
        (y1) -- [photon, edge label' = $\upgamma$] (x1);
        (w1) -- [anti fermion] (x1) -- [anti fermion] (w2);
        };
        \end{feynman}
    \end{tikzpicture}
\end{center}
This is $\upmu^-$-Bremsstrahlung with subsequent pair creation.

All these processes share the same common transition amplitude, as they contain the same basic interaction.