 \documentclass[headtopline=0.08em,headsepline=0.04em, bindingoffset = 5mm]{scrbook}

\usepackage[autooneside = false]{scrlayer-scrpage}
    \automark[section]{chapter}
    \addtokomafont{pagenumber}{\bfseries}

\usepackage[british]{babel}
\usepackage{blindtext}
\usepackage{cancel}
\usepackage[figurename={Abb.}, tablename={Tab.}]{caption} %Abb. statt Abbildung bei Grafiken sowie Tab. statt Tabelle
\usepackage{chngcntr}
\usepackage{xcolor}
\usepackage{dsfont}
\usepackage[T1]{fontenc} % Schrift ordentlich rendern, damit Umlaute auch Umlaute sind und nicht Vokale mit Doppelpunkten drüber
\usepackage[utf8]{luainputenc} % Interpretation des Inputs in utf8 (inkl. Umlaute, Akzente, etc.)
\usepackage{lmodern} % Ordentliche skalierbare Schriftart
\usepackage{framed}
\usepackage{graphicx} %zum einbinden von Grafiken
\usepackage{multicol}
\usepackage{makeidx}
\usepackage{mathtools}
\usepackage{soul}

\usepackage{paralist}
\usepackage{perpage}

\usepackage{tikz}
\usepackage{tikz-3dplot}
\usepackage[compat=1.1.0]{tikz-feynman}
\usetikzlibrary{mindmap,decorations.pathreplacing,patterns,hobby,calc,3d}
\usetikzlibrary{external}
\tikzexternalize[prefix=tikz/]

\usepackage{todonotes}
\usepackage{wrapfig} %zur wrapfigure-Umgebung, damit Grafiken nicht über die ganze Seitenbreite gehen

\usepackage{nicefrac} %zum darstellen von 1/2 in "schön"
\usepackage{bm}
\usepackage{braket}
\usepackage{tensor}
\usepackage{physics}
\usepackage{siunitx}
\usepackage{upgreek}
\usepackage{marvosym}
\usepackage{amsfonts, amsmath, amssymb} %Mathezeugs
\usepackage{ulem}

\usepackage{stackengine,scalerel}
\usepackage{calc}

\usepackage{hyperref} %zum referenzieren von Abbildungen, Gleichungen etc.
\usepackage{cleveref}
\makeatletter
\def\@footnotecolor{myorange}
\define@key{Hyp}{footnotecolor}{%
 \HyColor@HyperrefColor{#1}\@footnotecolor%
}
\def\@footnotemark{%
    \leavevmode
    \ifhmode\edef\@x@sf{\the\spacefactor}\nobreak\fi
    \stepcounter{Hfootnote}%
    \global\let\Hy@saved@currentHref\@currentHref
    \hyper@makecurrent{Hfootnote}%
    \global\let\Hy@footnote@currentHref\@currentHref
    \global\let\@currentHref\Hy@saved@currentHref
    \hyper@linkstart{footnote}{\Hy@footnote@currentHref}%
    \@makefnmark
    \hyper@linkend
    \ifhmode\spacefactor\@x@sf\fi
    \relax
  }%
\makeatother
\hypersetup{
  linkcolor = blue,
  citecolor  = myorange,
  urlcolor   = myblue,
  colorlinks = true,
}
%
\definecolor{myred}  {HTML}{A3061E}
\definecolor{myblue} {RGB} {0,63,119}
\definecolor{myyellow} {cmy} {0,0.263,0.741}
\definecolor{mygreen} {HTML}{0B6E4F}
%
\colorlet{myorange} {myyellow!60!myred}
\colorlet{myviolett} {myred!50!myblue!80}

\definecolor{orange}{rgb}{1, 0.5, 0} %selbsterklärend
\definecolor{yellow}{rgb}{1, 1, 0}

\renewcommand\thechapter{\Roman{chapter}}
\renewcommand*{\thesubsection}{\thesection.\arabic{subsection}}

\MakePerPage{footnote} %Fußnoten werden am Ende jeder Seite statt am Ende des Dokuments angezeigt

\numberwithin{equation}{chapter} %Damit Formeln im Format Chapter-Zahl.Formelzahl nummeriert werden (z.B.: I.17, II.3)
\counterwithin{figure}{chapter} %Damit Abbildungen im Format Chapter-Zahl.Abbildungszahl nummeriert werden (z.B.: I.17, II.3)
\counterwithin{table}{chapter} %Damit Abbildungen im Format Chapter-Zahl.Tabellenzahl nummeriert werden (z.B.: I.17, II.3)


\setcounter{tocdepth}{2} %Im Inhaltsverzeichnis werden Chapter und Sections angezeigt.

\newlength\shlength
\setlength{\parindent}{0pt}
\setlength{\parskip}{0.5\baselineskip}

\newcommand{\epsi}{\varepsilon}
\newcommand{\epso}{\varepsilon_0}
\newcommand{\ti}{\mathrm{i}}
\newcommand{\inte}{\int\limits}
\newcommand{\suml}{\sum\limits}
\newcommand{\ents}{$\hat{=}$\ }
\newcommand{\lt}{\leadsto}
\newcommand{\ra}{\rightarrow}
\newcommand{\Ra}{\Rightarrow}
\newcommand{\ua}{\uparrow}
\newcommand{\Ua}{\Uparrow}
\newcommand{\lra}{\leftrightarrow}
\newcommand{\Lra}{\Leftrightarrow}
\newcommand{\da}{\downarrow}
\newcommand{\Da}{\Downarrow}
\newcommand{\cra}{\curvearrowright}
\newcommand{\cla}{\curvearrowleft}
\newcommand{\icra}{\raisebox{\depth}{\scalebox{1}[-1]{\curvearrowright}}}
\newcommand{\icla}{\raisebox{\depth}{\scalebox{1}[-1]{\curvearrowleft}}}
\renewcommand{\braket}{\Braket}
\renewcommand{\bra}{\Bra}
\renewcommand{\ket}{\Ket}
\renewcommand{\mel}{\matrixelement}
\renewcommand{\set}{\Set}
\renewcommand{\bar}{\overline}
\renewcommand{\mathbb}[1]{\mathds{#1}}
\newcommand{\pa}{\partial}
\newcommand{\defi}{\coloneqq} % :=
\newcommand{\Chi}{\mathcal{X}}
\newcommand{\dom}{\mathrm{dom}}
\newcommand{\mr}[1]{\mathrm{#1}}
\newcommand{\mf}[1]{\mathfrak{#1}}
\newcommand{\mc}[1]{\mathcal{#1}}
\newcommand{\mb}[1]{\mathbb{#1}}
\newcommand{\ms}[1]{\mathscr{#1}}
\newcommand{\tb}[1]{\textbf{#1}}
\newcommand{\tk}[1]{\textit{#1}}
\newcommand{\ora}[1]{\overrightarrow{#1}}
\newcommand{\define}{\coloneqq} % :=
\newcommand{\thor}{\Lightning{}}
\newcommand{\Hil}{\mathscr{H}}
\newcommand{\1}{\mb{1}}
\newcommand{\Lag}{\mc{L}}
\newcommand{\E}{\mc{E}}
\newcommand{\Ham}{\mc{H}}

\newcommand{\Lap}{\bm{\Delta}}
\newcommand{\quab}{\pmb{\square}}
\newcommand{\fvec}[1]{{\bm{#1}}}
\newcommand{\mat}[1]{\begin{pmatrix} #1 \end{pmatrix}}
\renewcommand{\dfrac}[2]{\frac{\Pa #1}{\Pa #2}}
\newcommand{\defrac}[2]{\frac{\delta #1}{\delta #2}}
\newcommand{\delfrac}[2]{\frac{\pa #1}{\pa #2}}
\newcommand{\blue}[1]{\color{blue!80!black} #1 \color{black}}
\newcommand{\imp}[1]{\subsubsection{\blue{ #1 }}}
\newcommand{\mar}[1]{\textbf{#1}}
\newcommand{\J}{\mc{J}}
\newcommand{\Ort}{\mc{O}}
\newcommand{\Rot}{\mc{R}}
\newcommand{\RM}[1]{\MakeUppercase{\romannumeral #1{.}}}
\newcommand{\nbox}[1]{\fbox{\parbox{.5\textwidth}{#1}}}
\renewcommand{\Re}{\mathfrak{Re}}
\renewcommand{\Im}{\mathfrak{Im}}
\newcommand{\uul}[1]{\uuline{\mkern-1mu #1\mkern-1mu}\mkern2mu}
\newcommand{\poisson}[2]{ \lb \delfrac{#1}{q^\alpha}\delfrac{#2}{p_\alpha} - \delfrac{#1}{p_\alpha}\delfrac{#2}{q^\alpha}\rb }




\newcommand{\vorlesung}[1]{{\par% Doppelte Klammern wichtig, damit graue Farbe eingeschränkt wird
  \color{gray}
  \bigskip%
  \hrule height 0.5pt%
  \kern 5pt%
  \hbox to \textwidth{\hfil\small\smash{Ende Vorlesung am \textbf{#1}}\vphantom{M}\hfil}%
  \kern 5pt%
  \hrule height 0.5pt%
  \kern\medskipamount%
}} % Doppelte Klammern wichtig, damit graue Farbe eingeschränkt wird
%\renewcommand{\vorlesung}[1]{} %Einkommentieren, um Vorlesungsdaten auszublenden

\newenvironment{example}{
    \begin{leftbar}
    \tb{Example:}
}
{
    \end{leftbar}
}

%\RedeclareSectionCommands[tocdynnumwidth]{chapter,section}
\RedeclareSectionCommands[tocdynnumwidth]{chapter,section}

\begin{document}
{\KOMAoptions{twoside = false}
\begin{titlepage}
	\centering
	\vfill
	{\scshape\LARGE Friedrich-Alexander-Universität \\ Erlangen-Nürnberg \par}
	\vfill
	{\scshape\Large Transcript of the lecture advanced experimental physics B\\   \par}
	\vfill
	{\huge\bfseries Particle and astroparticle physics\par}
	\vfill
	\includegraphics[width=0.9\textwidth]{imgs/title.jpg}
	\vfill
	{\Large\itshape following the lecture of Prof. Dr. C. van Eldik\par}
	\vfill
	{\large written by Michael Winter \par}
    \vfill
    {\large last compiled on \today}
    \vfill
	{\large summer semester 2019 + 2020\par}
	\vfill
\end{titlepage}}

\tableofcontents %Fügt ein Inhaltsverzeichnis ein
\clearpage %Beendet die Seite

\chapter*{Preface}
Before the essential contents of the lectures are presented, some remarks have to be made:\\
This script -- or rather this lecture transcript -- was written in the course of the summer semester 2019 accompanying the lecture Advanced experimental physics: Particle and astroparticle physics, read by Prof. Dr. van Eldik. In its peculiarity, this transcript is probably also incorrect and inaccurate.

\tb{If you have an overleaf account, you can mark the appropriate places in the code and leave a comment, what needs to be improved.} So others can benefit from a correction as well.

In addition, the script is constantly being revised with regard to spelling, structure and content.

\chapter{Recap particle physics}
\section{Units}
In general natural units are used: $\hbar = \mathrm c = 1$
\begin{compactitem}
    \item[$\ra$] energy, mass and momentum: $\lsb E\rsb$
    \item[$\ra$] length and time: $\lsb E\rsb ^{-1}$
    \item[$\ra$] cross section: $\lsb E\rsb ^{-2}$
\end{compactitem}
\begin{example}
\begin{align*}
    \hbar \mr c \approx \SI{200}{\mega\eV\femto\meter} \,\hat =\, 1 \longrightarrow \SI{1}{\femto \meter} \,\hat\approx\, \SI{5}{\per\giga\eV}\\
    \hbar = \SI{6.6 e-25}{\giga \eV \second} \,\hat = 1\, \longrightarrow \SI{1}{\per \giga \eV} \,\hat \approx\, \SI{6.6e-25}{\second}
\end{align*}
\end{example}
\section{Relativistic kinematics}
We will use four vector notation, i.e.
\begin{align}
    \fvec A = \lb A^0, \vec A\rb \longrightarrow A^\mu & & \text{contravariant}\\
    \lb A^0, -\vec A\rb \longrightarrow A_\mu & & \text{covariant.}
\end{align}
\paragraph{Lorentz trafos}
Remember that the Lorentz transformation $\uul \Lambda$:
\begin{align}
    A^\mu \mapsto A'^\mu = \tensor{\Lambda}{^\mu_\nu}A^\nu
\end{align}
The transformation along the $z$-axis  is a Lorentz-boost
\begin{align}
    \tensor{\Lambda}{^\mu_\nu} \ra \uul \Lambda = \mat{\gamma & 0 &  0 &- \nicefrac \gamma \beta \\ 0 & 1 & 0 & 0 \\ 0 & 0 & 1 & 0 \\ - \nicefrac \gamma \beta & 0 & 0 & \gamma} \qquad \text{ with } \beta = \frac vc, \ \gamma = \frac{1}{\sqrt{1-\beta^2}}\,.
\end{align}
The Lorentz transformation is a \tb{unitary} transformation: ''rotation in 4-space``
\paragraph{Momentum four vector}
\begin{align}
    x^\mu \ra \fvec x = \lb t, \vec x\rb, \qquad E = \gamma m \mr c^2 = \gamma m, \qquad \vec p = \gamma m \vec v = \gamma m \vec \beta
\end{align}
So it is easy to see that the momentim vector $\vec p$ is measured in units of energy. Thus we may choose the following four vector as momentum four vector (and indeed it is)
\begin{align}
    p^\mu \ra \fvec p = \lb E, \vec p \rb, \qquad \Ra E^2 = \vec p^2 +m^2\,.
\end{align}
\paragraph{Basic invariant} under Lorentz transformation is the scalar product
\begin{align}
    \fvec A \cdot \fvec B = A^\mu B_\mu = \tensor{g}{_\mu_\nu} A^\mu B^\nu = A_\mu B^\mu = A^0B^0 - \vec A \vec B\\
    \text{with } \tensor{g}{_\mu_\nu} = \tensor{g}{^\mu^\nu} = \mat{1 & 0 & 0 & 0 \\ 0 & -1 & 0 & 0 \\ 0 & 0 & -1 & 0 \\ 0 & 0 & 0 & -1} \text{ as metric tensor.} \nonumber
\end{align}
\begin{example}
    scalar product of four momentum $\fvec p$
    \begin{align}
        p^2 = p^\mu p_\mu = E^2 - \vec p^2 = m^2\,,
    \end{align}
    since $E^2 = \vec p^2 +m^2$. Therefore the (rest) mass of the particle is the same in all frames of reference.
\end{example}
\begin{example}
    total energy $\sqrt s$ of particle collision
    \begin{center}
        \begin{tikzpicture}
            \draw[very thick, -latex] (0,0) node [left] {$\fvec p_1$} -- (3,0);
            \draw[very thick, latex-] (3.5,0) -- (6.5,0) node [right] {$\fvec p_2$};
        \end{tikzpicture}
    \end{center}
    \begin{align}
        s \defi \lb \fvec p_1 + \fvec p_2\rb^2 = \lb \fvec p_1 + \fvec p_2 \rb^\mu \lb \fvec p_1 + \fvec p_2 \rb_\mu 
    \end{align}
    is Lorentz invariant since it is a scalar product. Hence it is the same in all reference frames.
\end{example}
\paragraph{Four vector derivatives}
\begin{align}
    \pa_\mu = \pdv{x^\mu} \ra \lb \pa_t, \vec \grad \rb
\end{align}
transforms like a covariant vector. Whereas
\begin{align}
    \pa^\mu = \pdv{x_\mu} \ra \qty(t, - \vec \grad)
\end{align}
transforms like a contravariant vector. The ''scalar product`` of these two is the d'Alembert operator
\begin{align}
    \Box \defi \pa_\mu\pa^\mu = \pa_t^2 - \Lap\,,
\end{align}
\begin{compactitem}
    \item[with] the Laplace operator $\Lap = \vec \grad^2$.
\end{compactitem}

\section{Elementary particles}
\paragraph{Fermions (spin $\nicefrac 12$)} Lepton sector:
\begin{align}
    \mqty(\upnu_\mr{e} \\\mr e^- ), \quad \mqty(\upnu_\upmu \\ \upmu^-),\quad \mqty(\upnu_\uptau \\ \uptau^-) \qquad \text{with charges } \mqty{Q = 0 \\ Q= -\mr e}
\end{align}
They come with a certain mass hierarchy $m_\mr{e} < m_\upmu < m_\uptau$ and the assumption in the standard model that $m_\upnu = 0$ regardless of the neutrino flavour. However we know $m_\upnu >0$ but very small.

Quark sector:
\begin{align}
    \mqty(u \\ d), \quad \mqty(c \\ s), \quad \mqty( t \\ b) \qquad \text{with charges } \mqty{Q = \frac 23 \mr e \\ Q = - \frac 13 \mr e}
\end{align}
Free particles are bounded states with $qq'q''$ as baryons and $q\bar q$ as mesons. Additionally, only colour-neutral particle states are observed.

Both sectors come with the respective 6 anti-particles.

\paragraph{Bosons (spin $1$)} exchange particles of interactions
\begin{compactitem}
    \item[$\upgamma$:] em. interaction, $m=0$, $Q=0$ couples to electric charge
    \item[$Z^0$:] weak interaction, $m = \SI{91}{\giga\eV}$, $Q=0$, couples to weak charge
    \item[$W^\pm$:] weak interaction, $m = \SI{80}{\giga\eV}$, $Q = \pm \mr e$, couples to weak charge
    \item[$g$:] strong interaction, $m=0$, $Q=0$\\
    8 gluons in total, carry colour charge\\
    couple to quarks and to one another (''self-coupling``)
\end{compactitem}

\paragraph{Scalar (spin $0$)}
\begin{compactitem}
    \item[$H^0$:] Higgs boson, brings mass to $Z^0$, $W^\pm$ (''spontaneous symmetry breaking``)\\
    $m = \SI{125}{\giga\eV}$
\end{compactitem}

\section{Feynman diagrams}
These give pictorial representations of particle reactions. Perturbative expansion of scattering ''in a potential`` into leading order and higher order terms, if necessary.
\begin{example}
    electromagnetic scattering (leading order)
    \begin{multicols}{2}
        \begin{center}
            \begin{tikzpicture}
                \begin{feynman}
                \vertex (a1);
                \vertex [below right = 2cm of a1] (b1);
                \vertex [above right = 2cm of b1] (a2);
                \vertex [below = 2cm of b1] (c1);
                \vertex [below left = 2cm of c1] (d1);
                \vertex [below right = 2cm of c1] (d2);
                \vertex [below left = 0.5cm of d1] (z1);
                \vertex [right = 1.6cm of z1] (z2);
                \vertex [below right = 0.5cm of d2] (z4);
                \vertex [left = 1.6cm of z4] (z3);
                \vertex [below = 2.2cm of c1] (y1) {particles};
                \diagram*{
                (a1) -- [fermion, edge label = $\mr e^-$, momentum' = $\fvec p_1$] (b1) -- [fermion, edge label = $\mr e^-$, momentum' = $\fvec p_1'$] (a2),
                (b1) -- [photon, edge label' = $\upgamma$, momentum = $\fvec q$] (c1),
                (d1) -- [fermion, edge label' = $\upmu^-$, momentum = $\fvec p_2$] (c1) -- [fermion, edge label' = $\upmu^-$, momentum = $\fvec p_2'$] (d2);
                };
                \draw[decoration={brace}, decorate] (z2.south) -- node[below] {incoming} (z1.south west);
                \draw[decoration={brace}, decorate] (z4.south east) -- node[below] {outgoing} (z3.south);
                \end{feynman}
            \end{tikzpicture}
        \end{center}
        \begin{compactitem}
            \item[with] $\fvec q$: 4-momentum transfer
        \end{compactitem}
        We will describe this in $\fvec q^2$ (which is Lorentz invariant)
        \begin{align}
            \fvec q^2 = \qty(\fvec p_1 - \fvec p_1')^2 \neq 0
        \end{align}
        The inequity to zero means that it is off mass-shell\\
        $\Ra$ virtual particle because it carries to much/less energy
    \end{multicols}
    
    Next question to ask is: What is the \tb{transition amplitude}?
    \begin{multicols}{2}
        \begin{center}
            \begin{tikzpicture}
                \begin{feynman}
                    \vertex (b1);
                    \vertex [above left  = 1cm of b1] (a1) {$\mr e^-$};
                    \vertex [above right = 1cm of b1] (a2) {$\mr e^-$};
                    \vertex [below = 1cm of b1] (c1);
                    \vertex [below left  = 1cm of c1] (d1) {$\upmu^-$};
                    \vertex [below right = 1cm of c1] (d2) {$\upmu^-$};
                    \diagram*{
                        (a1) -- [fermion] (b1) -- [fermion] (a2),
                        (b1) -- [scalar, momentum = {$\fvec q, M$}] (c1),
                        (d1) -- [fermion] (c1) -- [fermion] (d2);
                    };
                \end{feynman}
            \end{tikzpicture}
        \end{center}
    \begin{align}
        amplitude \propto g_1 \frac{1}{\fvec q^2 - M^2} g_2
    \end{align}
    \begin{compactitem}
        \item[with] $g_{1,2}$: coupling strength between fermion and exchange particle
        \item[] $\nicefrac{1}{\fvec q^2-M^2}$: boson propagator for exchange particle of mass $M$
    \end{compactitem}
    \end{multicols}
    Here the electromagnetic and the weak interaction contribute, i.e.
    \begin{align*}
        \feynmandiagram[vertical= b to d, baseline = -2cm]{
        a [particle = $\mr e^-$] -- [fermion] b -- [fermion] c [particle = $\mr e^-$],
        b -- [photon, edge label = $\upgamma$] d,
        e [particle = $\upmu^-$]-- [anti fermion] d -- [anti fermion] f [particle = $\upmu^-$],
        };
        + 
        \feynmandiagram[vertical= b to d, baseline = -2cm]{
        a [particle = $\mr e^-$] -- [fermion] b -- [fermion] c [particle = $\mr e^-$],
        b -- [scalar, edge label = $Z^0$] d,
        e [particle = $\upmu^-$]-- [anti fermion] d -- [anti fermion] f [particle = $\upmu^-$],
        };
        = e \frac{1}{\fvec q^2}e + g_1 \frac{1}{\fvec q^2 - M_Z^2} g_2
    \end{align*}
    Last question: What is the \tb{cross section}?\\
    It is proportional to the transition probability
    \begin{align}
        \sigma \propto W, \qquad W \propto \qty| amplitude_1 + amplitude_2 |^2 \propto \frac{\mr e^4}{\fvec q^4}
    \end{align}
    The factor $\nicefrac{1}{\fvec q^4}$ leads to a rapid decrease of the cross section as momentum transfer increases.
    \begin{center}
        \begin{tikzpicture}
            \draw[-latex] (-3.25,0) node [left] {$\fvec p_1$} -- (-.25,0);
            \draw[latex-] (.25,0) -- (3.25,0) node [right] {$\fvec p_2$};
            \draw[-latex] (0,0,0) -- (3,1.5,0);
            \coordinate (O) at (0,0,0);
            \tdplotsetcoord{P1}{3.5}{90}{30};
            \tdplotsetcoord{P2}{3.5}{90}{20};
            \tdplotsetcoord{P3}{3.5}{30}{30};
            \tdplotsetcoord{P4}{3.5}{0}{30};
            \draw[-latex] (P1) -- (P2);% -- (P3) -- (P4) -- (P1);
            \draw[dashed] (3.5,90,30) arc (30:20:3.5);
        \end{tikzpicture}
    \end{center}
\end{example}
\chapter{Covariant description of relativistic particles}
\section{Non-relativistic quantum mechanics}
Recall the energy-momentum relation
\begin{align}
    E = \frac{1}{2m} \qty|\vec p|^2\,.
\end{align}
By identifying $E \ra \ti \hbar \pa_t$, $\vec p \ra -\ti \hbar \vec \grad$, we get an operator equation, i.e. the Schrödinger equation
\begin{align}\label{eq:schroedinger}
    \boxed{\qty(\ti \pa_t + \vec \grad ^2 \frac{1}{2m}) \phi \qty(\vec x, t) = 0}
\end{align}
\begin{compactitem}
    \item[with] $\phi \qty(\vec x, t) \in \mb C$ as the one-particle wave function.
\end{compactitem}
We can get the statistical interpretation via $\qty | \phi | \dd[3]{x}$ $\hat =$ probability to find particle in volume $\dd[3]{x}$ at time $t$ and we can obtain the localisation probability density
\begin{align}
    \ra \quad \rho = \rho \qty (\vec x, t) \defi \qty|\phi \qty(\vec x, t)|^2\,.
\end{align}
\begin{multicols}{2}
    \begin{center}
        \begin{tikzpicture}
            \draw (0.,-0.05) to [curve through = {(1.,0.) .. (1.2,0.8) .. (0.4,1.6) .. (0.,2.) .. (-0.5,0.75)}] (0.,-0.05);
            \foreach \i in {-0.2,0.2,0.6}
                \draw[-latex] (\i,1.2-\i) to [curve through = {(\i+0.2,1.8-\i)}] (\i+0.1,2.4);
            \foreach \i in {-0.2,-0.5,-0.8}
                \draw[-latex] (\i,-1.4-\i) to [curve through = {(\i-0.2,-0.6-\i)}] (\i,-0.2-\i);
            \node at (1.,2.6) {$\vec j$};
            \node at (0,0.5) {$\rho$};
            \node at (1.4,0) {$\dd[3]{x}$};
        \end{tikzpicture}
    \end{center}
    From electrodynamics we get the continuity equation:
    \begin{align}\label{eq:continuity}
        \boxed{\pa_t \rho + \vec \grad \vec j = 0}
    \end{align}
    \begin{compactitem}
        \item[with] $\vec j \qty(\vec x, t)$ as the \tb{probability density current}.
    \end{compactitem}
\end{multicols}
How does $\vec j$ depend on the wave function $\phi$?
\begin{align*}
    \qty(-\ti \phi^*) \cdot (\cref{eq:schroedinger}) & = \phi^*\pa_t \phi - \frac{\ti}{2m} \phi^* \vec \grad ^2 \phi = 0 \\
    \qty(- \ti \phi) \cdot (\cref{eq:schroedinger})^* & = - \phi \pa_t \phi^* - \frac{\ti}{2m} \phi \vec \grad ^2 \phi^* = 0
\end{align*}
By subtracting these two equations we get
\begin{align}
    \Ra \quad \underbrace{\qty(\phi^*\pa_t \phi + \phi \pa_t \phi^*)}_{\pa_t \qty(\phi^*\phi) = \pa_t \qty|\phi|^2} - \frac{\ti}{2m} \underbrace{\qty(\phi^* \vec \grad ^2 \phi - \phi \vec \grad^2 \phi^*)}_{\vec \grad \qty(\phi^*\vec \grad \phi - \phi \vec \grad \phi^*)} = 0
\end{align}
and by using the continuity \cref{eq:continuity} we get an explicit expression for the probability density current
\begin{align}
    \boxed{ \vec j = - \frac{\ti}{2m}\qty(\phi^*\vec \grad \phi - \phi \vec \grad \phi^*) }\,.
\end{align}
The probability density current, based on the Schrödinger equation, is not covariant. This means that the Schrödinger equation treats time and space different.

\section{Relativistic particles: Klein-Gordon equation}
Now we use the \tb{relativistic} energy-momentum relation
\begin{align}
    E^2 = \vec p^2 +m^2 \qquad E \ra \ti \hbar \pa_t, \quad \vec p \ra \ti \hbar \vec \grad\nonumber \\
    \Ra \quad \boxed{\qty(\pa_t^2 - \Lap +m^2) \phi \qty(\vec x, t) = 0}\,.
\end{align}
This is the so-called \tb{Klein-Gordon equation}. By using four vector notation
\begin{align}
    p^\mu \ra \mqty(E \\ \vec p) \ra \mqty(\ti \pa_t \\ - \ti \vec \grad) \ra \ti \pa^\mu \quad \Ra \quad \boxed{\qty(\pa_\mu\pa^\mu + m^2) \phi = 0}\,,
\end{align}
which denotes the Klein-Gordon equation in covariant form. Since the scalar product and the mass are Lorentz invariant, this KG equation is too.\\
$\ra$ Klein-Gordon equation describes spin-0 particles.

The continuity equation thus is
\begin{align}
    \pa_t \rho + \vec \grad \vec j = 0 \quad \ra \quad \boxed{\pa_\mu j^\mu = 0}\,.
\end{align}
How do $\rho$ and $\vec j$ depend on $\phi$?
\begin{align}\label{eq:four_density_KGe}
    \rho = \ti \qty(\phi^*\pa_t \phi - \phi \pa_t \phi^*), \quad \vec j = - \ti \qty(\phi^* \vec\grad \phi - \phi  \vec \grad \phi^*)
\end{align}
and the for vector current is
\begin{align}
    \boxed{j^\mu = \ti \qty(\phi^* \pa^\mu \phi - \phi\pa^\mu \phi^*)}\,.
\end{align}
The fundamental free-particle solution for the KGe is given by
\begin{align}\begin{split}
    \phi \qty(\vec x,t) \equiv \phi (\fvec x) & = N \cdot \exp(\ti \qty(\vec p\vec x - Et))\\
    & = N \exp(- \ti p_\mu x^\mu) = N \exp(-\ti \fvec p\fvec x)
\end{split}\end{align}
\begin{compactitem}
    \item[with] $N$ as the normalisation.
\end{compactitem}
By inserting this into the probability density current (\cref{eq:four_density_KGe}), we get
\begin{align}
    \boxed{j^\mu = 2 p^\mu \qty|N|^2}
\end{align}
for free particles. Note that espacially $\rho = 2 E \qty|N|^2$.
\begin{itemize}[$\ra$]
    \item Localisation probability $\rho \dd[3]{x}$, but under Lorentz boost $\dd[3]{x} \ra \frac 1\gamma \dd[3]{x}$, since $\rho \propto E \propto \gamma$. This effect is compensated for the probability density current $\vec j \propto \vec p$.
    \item Particle current $\vec j$ in the direction of the particle momentum $\vec p$.
\end{itemize}
What are the eigenvalues of the free-particle solution?
\begin{align}
    &\qty(\pa_\mu \pa^\mu + m^2) \exp(-\ti \fvec p\fvec x) = 0 \nonumber \\
    &\Ra\quad \qty(\qty(-\ti)^2 p_\mu p^\mu + m^2) \exp(-\ti \fvec p \fvec x) = 0 \nonumber \\
    &\Ra\quad - p_\mu p^\mu + m^2 = 0, \quad p_\mu p^\mu = m^2 = E^2 - \vec p^2 \nonumber \\
    &\Ra\quad \boxed{E = \pm \sqrt{\vec p^2 + m^2}}
\end{align}
So we get two solutions for the particle energy. However, the negative solution $E < 0 \Ra \rho < 0$ is unphysical! There is one way out, the Feynman-Stückelberg approach. Here we interpret $\rho$ as a charge density.
\begin{itemize}
    \item[$\ra$] Can be negative, since particles can have negative charge.
    \item[$\ra$] Interpret $j^\mu$ as charge current.
    \item[$\Ra$] $j^\mu \ra j'^\mu = Q j^\mu$ with $Q$ as particle charge.
\end{itemize}

\begin{example}
    electron, $ Q= -\mr e$
    
    By assuming an electron with spin 0, we get
    \begin{align}
        j^\mu \qty(\mr e^-) = - \ti \mr e \qty(\phi^* \pa^\mu \phi - \phi \pa^\mu \phi^*)\,.
    \end{align}
    Free particle solution ($E>0$):
    \begin{align}
        j^\mu \qty(\mr e^-) = -2 \mr e p^\mu \qty|N|^2 \ra - 2 \mr e \qty|N|^2 \mqty(E \\ \vec p)
    \end{align}
    For $E<0$: Consider anti muon with $E>0$
    \begin{align}
        \fvec j \qty(\mu^+) = 2 \mr e \fvec p \qty|N|^2 = 2 \mr e \qty|N|^2 \mqty(E \\ \vec p) = - 2 \mr e \qty|N|^2 \mqty(-E \\ -\vec p)  = - \fvec j \qty(\mu^-)\,.
    \end{align}
    So we get a muon with $E<0$, that moves backwards.
\end{example}
\begin{itemize}
    \item[$\ra$] solution with $E<0$ can be used to describe antiparticles with $E' = -E$
\end{itemize}
\begin{align*}
    \begin{tikzpicture}
        \begin{feynman}
        \vertex (a1) {$\mr e^-$};
        \vertex [below right = 2.5cm of a1] (b1);
        \vertex [above right = 2cm of b1] (a2) {$\mr e^-$};
        \vertex [below = 2cm of b1] (c1);
        \vertex [below left = 2cm of c1] (d1) {$\upmu^+$};
        \vertex [below right = 2cm of c1] (d2) {$\upmu^+$};
        \vertex [right = 1cm of d1] (j1);
        \vertex [left = 1cm of d2] (j2);
        \vertex [right = 1cm of a1] (i1);
        \vertex [left = 1cm of a2] (i2);
        \vertex [below = 1cm of b1] (b2);
        \vertex [right = 2.5cm of b2] (e) {$\hat =$};
        \diagram*{
        (a1) -- [fermion, momentum' = $\fvec p_A$] (b1) -- [fermion, momentum' = $\fvec p_C$] (a2);
        (b1) -- [photon, edge label' = $\upgamma$] (c1);
        (d1) -- [anti fermion, momentum = $\fvec p_B$] (c1) -- [anti fermion, momentum = $\fvec p_D$] (d2);
        (j1) -- [draw = none, half left, momentum = $\ $, edge label' = $j^\mu \qty(\upmu^+)$] (j2);
        (i1) -- [draw = none, half right, momentum' = $\ $, edge label = $j^\mu \qty(\mr e^-)$] (i2);
        };
        \vertex [right = 6cm of a1] (z1) {$\mr e^-$};
        \vertex [below right = 2.5cm of z1] (y1);
        \vertex [above right = 2cm of y1] (z2) {$\mr e^-$};
        \vertex [below = 2cm of y1] (x1);
        \vertex [below left = 2cm of x1] (w1) {$\upmu^-$};
        \vertex [below right = 2cm of x1] (w2) {$\upmu^-$};
        \vertex [right = 1cm of w1] (k1);
        \vertex [left = 1cm of w2] (k2);
        \vertex [right = 1cm of z1] (l1);
        \vertex [left = 1cm of z2] (l2);
        \diagram*{
        (z1) -- [fermion, momentum' = $\fvec p_A$] (y1) -- [fermion, momentum' = $\fvec p_C$] (z2);
        (y1) -- [photon, edge label' = $\upgamma$] (x1);
        (w1) -- [fermion, rmomentum = $-\fvec p_B$] (x1) -- [fermion, rmomentum = $-\fvec p_D$] (w2);
        (k2) -- [draw = none, half right, momentum' = $\ $, edge label = $j^\mu \qty(\upmu^-)$] (k1);
        (l1) -- [draw = none, half right, momentum' = $\ $, edge label = $j^\mu \qty(\mr e^-)$] (l2);
        };
        \end{feynman}
    \end{tikzpicture}
\end{align*}
Consequence: Can use particle states with $p^\mu \ra - p^\mu$ for description of antiparticles.

\section{Crossing symmetry}
The description of scattering processes is highly symmetric under the exchange of space and time: This originates in the fact that wave equations treat time and space the same way.
\begin{example}
    $\mr e^- \upmu^-$ scattering in QED
    
    By exchange of time and space, the Feynman diagram changes as:
    \begin{align*}
        \begin{tikzpicture}
            \begin{feynman}
            \vertex (a1) {$\mr e^-$};
            \vertex [below right = 2.5cm of a1] (b1);
            \vertex [above right = 2cm of b1] (a2) {$\mr e^-$};
            \vertex [below = 2cm of b1] (c1);
            \vertex [below left = 2cm of c1] (d1) {$\upmu^+$};
            \vertex [below right = 2cm of c1] (d2) {$\upmu^+$};
            \vertex [right = 1cm of d1] (j1);
            \vertex [left = 1cm of d2] (j2);
            \vertex [right = 1cm of a1] (i1);
            \vertex [left = 1cm of a2] (i2);
            \vertex [below = 1cm of b1] (b2);
            \vertex [right = 3cm of b2] (e) {$\hat =$};
            \diagram*{
            (a1) -- [fermion] (b1) -- [fermion] (a2);
            (b1) -- [photon, edge label' = $\upgamma$] (c1);
            (d1) -- [anti fermion] (c1) -- [anti fermion] (d2);
            (j1) -- [draw = none, half left, momentum = $\ $, edge label' = $j^\mu \qty(\upmu^+)$] (j2);
            (i1) -- [draw = none, half right, momentum' = $\ $, edge label = $j^\mu \qty(\mr e^-)$] (i2);
            };
            \vertex [right = 3cm of e] (y1);
            \vertex [below left = 2cm of y1] (z1) {$\mr e^-$};
            \vertex [above left = 2cm of y1] (z2) {$\mr e^+$};
            \vertex [right = 2cm of y1] (x1);
            \vertex [above right = 2cm of x1] (w1) {$\upmu^-$};
            \vertex [below right = 2cm of x1] (w2) {$\upmu^+$};
            \vertex [below = 1cm of w1] (k1);
            \vertex [above = 1cm of w2] (k2);
            \vertex [above = 1cm of z1] (l1);
            \vertex [below = 1cm of z2] (l2);
            \diagram*{
            (z1) -- [fermion] (y1) -- [fermion] (z2);
            (y1) -- [photon, edge label' = $\upgamma$] (x1);
            (w1) -- [anti fermion] (x1) -- [anti fermion] (w2);
            (k2) -- [draw = none, half left, momentum = $\ $, edge label' = $j^\mu \qty(\upmu^-)$] (k1);
            (l1) -- [draw = none, half right, momentum' = $\ $, edge label = $j^\mu \qty(\mr e^-)$] (l2);
            };
            \end{feynman}
        \end{tikzpicture}
    \end{align*}
    This is equivalent to a counter clockwise rotation by \SI{90}{\degree}. The resulting diagram represents $\mr e^+ \mr e^-$ annihilation followed by $\mu^+\mu^-$ creation.
\end{example}
By exchanging the incoming anti muon by an outgoing muon in the $\upmu^+\upmu^-$ annihilation, we get the diagram:
\begin{center}
    \begin{tikzpicture}
        \begin{feynman}
        \vertex (y1);
        \vertex [below = 1.8cm of y1] (s);
        \vertex [right = 0.7cm of s] (z1) {$\upmu^-$};
        \vertex [above left = 2cm of y1] (z2) {$\upmu^-$};
        \vertex [right = 2cm of y1] (x1);
        \vertex [above right = 2cm of x1] (w1) {$\mr e^-$};
        \vertex [below right = 2cm of x1] (w2) {$\mr e^+$};
        \diagram*{
        (z1) -- [anti fermion] (y1) -- [anti fermion] (z2);
        (y1) -- [photon, edge label' = $\upgamma$] (x1);
        (w1) -- [anti fermion] (x1) -- [anti fermion] (w2);
        };
        \end{feynman}
    \end{tikzpicture}
\end{center}
This is $\upmu^-$-Bremsstrahlung with subsequent pair creation.

All these processes share the same common transition amplitude, as they contain the same basic interaction.
%\chapter{Electrodynamics of spinless particles}
\begin{enumerate}[1)]
    \item Introduction of charged particle in potential
    \item Potential provided by the other scattering matter
\end{enumerate}

\section{Covariant electrodynamics}
From electrodynamics recall Maxwell's equation:
\begin{align*}
    \vec \grad \vec E & = \rho   & \vec \grad \times \vec E & = - \pa_t \vec B \\
    \vec \grad \vec B & = 0      & \vec \grad \times \vec B & = \vec j + \pa_t \vec E
\end{align*}
Express the fields $\vec E$, $\vec B$ by potentials $\phi$, $\vec A$:
\begin{align}
    \vec B = \vec \grad \times \vec A, \qquad \vec E = - \vec \grad \phi - \pa_t \vec A
\end{align}
Gauge freedom: Choose Coulomb gauge? $\vec \grad \vec A = 0$\\
Here: We will use the Lorentz gauge $\boxed{\vec \grad \vec A + \pa_t \phi = 0}$
\begin{align}
    & \Ra \quad \Lap \vec A - \pa_t^2 \vec A = - \vec j, \qquad \Lap \phi - \pa_t^2 \phi = - \rho
\end{align}
With $A^\mu \ra \qty(-\phi, \vec A)$ and $j^\mu \ra \qty(\rho, \vec j)$, we get Maxwell's equations in covariant form
\begin{align}
    \boxed{ \pa_\mu \pa^\mu A^\nu = j^\nu }\,.
\end{align}
Not that the Lorentz gauge can also be written as $\boxed{\pa_\mu A^\mu = 0 = \pa^\mu A_\mu}$.

\section{The spinless electron in the electromagnetic field}
In classical electrodynamics the canonical momentum is
\begin{align}
    \begin{rcases} E \ra E - Q \phi \\ \vec p \ra \vec p - Q \vec A \end{rcases} \qquad p^\mu \ra p^\mu - Q A^\mu\,.
\end{align}
We will substitute by operators:
\begin{align}
    \begin{rcases} E - Q \phi \ra \ti \pa_t - Q \phi \\ \vec p - Q \vec A \ra - \ti \vec \grad - Q \vec A \end{rcases} \qquad \ti \pa^\mu - Q A^\mu
\end{align}
In total, for electrons of charge $Q= -\mr e$:
\begin{align}
    \boxed{p^\mu \ra \ti \pa^\mu + \mr e A^\mu}
\end{align}
By insertion into Klein-Gordon equation (see \cref{eq:KGe}):
\begin{align}\begin{split}
    & \qty(\ti \pa_\mu + \mr e A_\mu) \qty(\ti \pa^\mu + \mr e A^\mu) \phi\qty(\fvec x) - m^2 \phi \qty(\fvec x) = 0 \\
    & H_0 \ra H_0 + V \\
    & \Ra \quad \bigg( \pa_\mu\pa^\mu \underbrace{- \ti \mr e \qty(A_\mu \pa^\mu + \pa_\mu A^\mu ) - \mr e^2 A^2 + m^2}_{V_\mr{EM}} \bigg) \phi \qty(\fvec x) = 0
\end{split}\end{align}
Perturbation of free particle Hamiltonian $H_0$ due to the coupling of electron to $A^\mu$. We are only interested in the leading order contribution $\ra$ neglect the term $\propto \mr e^2$
\begin{align}
    \boxed{ V_\mr{EM} \qty(\fvec x) = - \ti \mr e \qty(\pa_\mu A^\mu + A_\mu \pa^\mu) }
\end{align}
with this the charge current density is
\begin{align}
    j^\mu = - \ti \mr e \qty( \phi_\mr{f}^* \pa^\mu \phi_\mr{i} - \qty( \pa_\mu \phi_\mr{f}^*) \phi_\mr{i} )
\end{align}
\begin{center}
    \begin{tikzpicture}
        \begin{feynman}
            \vertex (b1);
            \vertex [above left = 1.5cm of b1] (a1) {$\phi_\mr{i}$};
            \vertex [above right = 1.5cm of b1] (a2) {$\phi_\mr{f}$};
            \vertex [below = 1cm of b1] (c1) {$V\qty(\fvec x)$};
            \vertex [right = 1cm of a1] (j1);
            \vertex [left = 1cm of a2] (j2);
            \diagram*{
            (a1) -- [fermion] (b1) -- [fermion] (a2);
            (b1) -- [photon] (c1);
            (j1) -- [draw = none, momentum' = $\ $, edge label = $j^\mu$, half right] (j2);
            };
        \end{feynman}
    \end{tikzpicture}
\end{center}
\begin{align}\begin{split}
    T_\mr{fi} & = - \ti \int \phi_\mr{f}^* \qty(\fvec x) V_\mr{EM} \qty(\fvec x) \phi_\mr{i}\qty(\fvec x) \dd[4]{x} \\
    & = - \ti \int \qty(-\ti \mr e) \phi_\mr{f}^* \qty(\pa_\mu A^\mu + A_\mu \pa^\mu) \phi_\mr{i} \dd[4]{x}
\end{split}\end{align}
with integration by parts
\begin{align}
    \int\limits_{-\infty}^{\infty} \phi_\mr{f}^* \pa_\mu A^\mu \phi_\mr{i} \dd[4]{x} = \underbrace{\qty[\phi_\mr{f}^* \sum_\mu A^\mu \phi_\mr{i}]_{-\infty}^{\infty}}_{=0} - \int\limits_{-\infty}^{\infty} \qty(\pa_\mu \phi_\mr{f}^*) A^\mu \phi_\mr{i} \dd[4]{x}
\end{align}
hence
\begin{align}\begin{split}\label{eq:transition_amplitude_tree}
    T_\mr{fi} & = -\ti \int \qty(-\ti \mr e) \qty[ - \qty(\pa_\mu \phi_\mr{f}^*) A^\mu \phi_\mr{i} + \phi_\mr{f}^* A^\mu \pa_\mu \phi_\mr{i} ] \dd[4]{x} \\
    & = - \ti \int \underbrace{\overbrace{\qty(-\ti\mr e)}^{\substack{\text{coupling}\\\text{constant}}} \qty[ \phi_\mr{f}^* \pa_\mu \phi_\mr{i} - \qty(\pa_\mu \phi_\mr{f}^*) \phi_\mr{i} ]}_{\substack{\text{four vector current of} \\ \text{an electron (see \cref{eq:four_density_KGe}}}} \underbrace{\ A^\mu\ }_{\mathrlap{\substack{\text{interaction} \\ \text{four potential}}}} \dd[4]{x}\\
    \Aboxed{ T_\mr{fi} & = -\ti \int j_\mu^G A^\mu \dd[4]{x}}
\end{split}\end{align}
The boxed equation gives us the coupling between electron charge current and electromagnetic potential with coupling strenth $\mr e$.\\
$\ra$ The Feynman diagram can be extended with new information:
\begin{center}
    \begin{tikzpicture}
        \begin{feynman}
            \vertex (b1);
            \vertex [above left = 2cm of b1] (a1) {$\mr e^-$};
            \vertex [above right = 2cm of b1] (a2) {$\mr e^-$};
            \vertex [below = 1cm of b1] (c1) {$A^\mu$};
            \vertex [right = 1cm of a1] (j1);
            \vertex [left = 1cm of a2] (j2);
            \diagram*{
            (a1) [particle = $\mr e^-$] -- [fermion] (b1) -- [fermion] (a2);
            (b1) -- [photon] (c1);
            (j1) -- [draw = none, momentum' = $\ $, edge label = $j_\mu^G \qty(\mr e^-)$, half right] (j2);
            };
        \end{feynman}
    \end{tikzpicture}
\end{center}
Free-particle approximation:\todo[color = none]{Similar calculation can be done for $u,c,t \ra \frac 23 \mr e$ and $d,s,b \ra - \frac 13 \mr e$}
\begin{align}\begin{split}
    & \phi_\mr{i}\qty(\fvec x) = N_\mr{i} \exp(- \ti \fvec p_\mr{i} \fvec x), \qquad \phi_\mr{f} \qty(\fvec x) = N_\mr{f} \exp(- \ti \fvec p_\mr{f} \fvec x) \\
    & \Ra \quad \boxed{ j_\mu^\mr{fi} \qty(\fvec x) = - \mr e N_\mr{i} N_\mr{f} \qty(\fvec p_\mr{i} + \fvec p_\mr{f})_\mu \exp(-\ti \qty(\fvec p_\mr{i} - \fvec p_\mr{f}) \fvec x) }
\end{split}\end{align}
This denotes the charge density current of scattering an electron in free-particle  approximation.

\pagebreak
\section{Spinless electron-muon scattering}
The electromagnetic potential $A^\mu$ now is provided by a $\mu$-current. Thus the Feynman diagram is

\begin{center}
    \begin{tikzpicture}
        \begin{feynman}
            \vertex (b1);
            \vertex [above left = 2cm of b1] (a1) {$\mr e^-$};
            \vertex [above right = 2cm of b1] (a2) {$\mr e^-$};
            \vertex [below = 1cm of b1] (c1) {$A^\mu$};
            \vertex [right = 1cm of a1] (j1);
            \vertex [left = 1cm of a2] (j2);
            \diagram*{
            (a1) [particle = $\mr e^-$] -- [fermion] (b1) -- [fermion] (a2);
            (b1) -- [photon] (c1);
            (j1) -- [draw = none, momentum' = $\ $, edge label = $j_\mu^G \qty(\mr e^-)$, half right] (j2);
            };
            \vertex [right = 6cm of a1] (z1) {$\mr e^-$};
            \vertex [below right = 2.5cm of z1] (y1);
            \vertex [above right = 2cm of y1] (z2) {$\mr e^-$};
            \vertex [below = 1.5cm of y1] (x1);
            \vertex [below left = 2cm of x1] (w1) {$\upmu^-$};
            \vertex [below right = 2cm of x1] (w2) {$\upmu^-$};
            \vertex [right = 1cm of w1] (k1);
            \vertex [left = 1cm of w2] (k2);
            \vertex [right = 1cm of z1] (l1);
            \vertex [left = 1cm of z2] (l2);
            \diagram*{
            (z1) -- [fermion, momentum' = $\fvec p_A$] (y1) -- [fermion, momentum' = $\fvec p_C$] (z2);
            (y1) -- [photon, edge label' = $\upgamma$] (x1);
            (w1) -- [fermion, momentum = $\fvec p_B$] (x1) -- [fermion, momentum = $\fvec p_D$] (w2);
            (l1) -- [draw = none, half right, momentum' = $\ $, edge label = $j_\nu^{(1)}$] (l2);
            (k2) -- [draw = none, half right, rmomentum' = $\ $, edge label = $j_\nu^{(2)}$] (k1);
            };
        \end{feynman}
    \end{tikzpicture}
\end{center}

Connection between potential $A^\mu$ and $\mu$-current $j_\mu^{(2)}$ vie Maxwell relation
\begin{align}
    \pa_\mu \pa^\mu A^\nu = j_{(2)}^\nu 
\end{align}
where
\todo[color=none]{Note that this is the charge density current of a muon in the free-particle approximation.}
\begin{align}
    j_{(2)}^\nu = - \mr e N_B N_D \qty( \fvec p_B + \fvec p_D)^\nu \exp( - \ti \qty(\fvec p_B - \fvec p_D) \fvec x)\\
    \ra \quad \pa_\mu \pa^\mu A^\nu = - \mr e N_B N_D \qty(\fvec p_B + \fvec p_D)^\nu \exp(\ti \fvec q \fvec x)
\end{align}
since $q^\mu = \qty(\fvec p_A - \fvec p_C)^\mu = \qty(\fvec p_B - \fvec p_D)^\mu$.

The solution for the four potential is
\begin{align}
    \boxed{A^\mu \qty(\fvec x) = - \frac{1}{\fvec q^2} j_{(2)}^\mu \qty(\fvec x)}\,,
\end{align}
which is the electromagnetic potential due to fly-by muon.

Transition amplitude on tree level (see \cref{eq:transition_amplitude_tree}):
\begin{align}
    T_\mr{fi} = -\ti \int j_\mu^{(1)} \qty(\fvec x) A^\mu \qty(\fvec x) \dd[4]{x} = -\ti \int j_\mu^{(1)} \qty(\fvec x) \qty(-\frac{1}{\fvec q^2}) j_{(2)}^\mu \qty(\fvec x) \dd[4]{x}
\end{align}
Integration similiar to \cref{sec:Interaction_particle_potential}:
\begin{align}
    T_\mr{fi} = - \ti \underbrace{N_A N_B N_C N_D}_{\substack{\text{wave functions} \\ \text{normalisation}}} \qty(2\pi)^4 \underbrace{\delta^4 \qty(\fvec p_D + \fvec p_C - \fvec p_B - \fvec p_A)}_{\substack{\text{component-wise}\\ \text{energy-momentum conservation}}} m
\end{align}
With the \tb{invariant amplitude $m$} being
\begin{align}
    -\ti m = \underbrace{\ti \mr{e} \qty(\fvec p_A + \fvec p_C)^\mu}_{\substack{\text{electron current} \\ \text{(coupling const. $\mr e$)}}} \underbrace{\qty(- \frac{\ti}{\fvec q^2} \tensor{g}{_\mu_\nu})}_{\substack{\text{photon} \\ \text{propagator}}} \underbrace{\ti \mr e \qty(\fvec p_B + \fvec p_D)^\nu}_{\substack{\text{muon current} \\ \text{(coupling const. $\mr e$)}}} \,.
\end{align}
$m$ describes the physics of the process:
\begin{center}
    \begin{tikzpicture}
        \begin{feynman}
            \vertex (y1);
            \vertex [above left = 2cm of y1] (z1) {$\mr e^-$};
            \vertex [above right = 2cm of y1] (z2) {$\mr e^-$};
            \vertex [below = 1.5cm of y1] (x1);
            \vertex [below left = 2cm of x1] (w1) {$\upmu^-$};
            \vertex [below right = 2cm of x1] (w2) {$\upmu^-$};
            \vertex [right = 1cm of w1] (k1);
            \vertex [left = 1cm of w2] (k2);
            \vertex [right = 1cm of z1] (l1);
            \vertex [left = 1cm of z2] (l2);
            \diagram*{
            (z1) -- [fermion, momentum' = $\fvec p_A$] (y1) -- [fermion, momentum' = $\fvec p_C$] (z2);
            (y1) -- [photon, edge label' = $\upgamma$] (x1);
            (w1) -- [fermion, momentum = $\fvec p_B$] (x1) -- [fermion, momentum = $\fvec p_D$] (w2);
            (l1) -- [draw = none, half right, momentum' = $\ $, edge label = \scriptsize{$\mqty{j^\nu_{(1)} = \\ \ti \mr e \qty(\fvec p_A + \fvec p_C)^\nu}$}] (l2);
            (k2) -- [draw = none, half right, rmomentum' = $\ $, edge label = \scriptsize{$\mqty{j^\nu_{(2)} = \\ \ti \mr e \qty(\fvec p_B + \fvec p_D)^\nu}$}] (k1);
            };
        \end{feynman}
    \end{tikzpicture}
\end{center}
FOr some processes many realisations are possible, e.g.
\begin{align}
    \feynmandiagram[vertical = b to d, baseline = -1.4cm, small]{
        a [particle = $\mr e^-$] -- [fermion] b -- [fermion] c [particle = $\mr e^-$],
        b -- [photon] d,
        e [particle = $\mr e^+$] -- [anti fermion] d -- [anti fermion] f [particle = $\mr e^+$],
    };
    +
    \feynmandiagram[horizontal = b to d, baseline =(b.base),small]{
        a [particle = $\mr e^+$] -- [anti fermion] b -- [anti fermion] c [particle = $\mr e^-$],
        b -- [photon] d,
        e [particle = $\mr e^+$] -- [fermion] d -- [fermion] f [particle = $\mr e^-$],
    };
    \quad \longrightarrow \quad \qty|T_\mr{fi}^{(1)} + T_\mr{fi}^{(2)}|^2
\end{align}

\section{Towards the \texorpdfstring{$\mr e^- \upmu^- \ra \mr e^- \upmu^-$}{} cross section}
Wave function: $hi \qty(\fvec x) = N \exp(-\ti \fvec p \fvec x)$; Probability density: $\rho = 2E \qty|N|^2$

The probability to find a particle in arbitrary (large) volume $V$: $\int_V \rho \dd[e]{x} \overset{!}{=} 1$
\begin{align}
    \ra \quad N = \frac{1}{\sqrt{2EV}}\,,
\end{align}
which is the covariant normalisation. Accordingly, the transition rate is:
\begin{align}
    W_\mr{fi} = \Gamma_\mr{fi} = \frac{\qty| T_\mr{fi}|^2}{TV} = \qty(16 V^4 E_A E_B E_C E_D) \qty(2\pi)^4 \delta^4 \qty(\fvec p_D + \fvec p_B - \fvec p_A - \fvec p_C) \qty|m|^2
\end{align}
In the next step, we want to calculate the cross section. This resembles the effective area that a particle sees from a target when undergoing scattering.
\begin{align}
    \qty[\sigma] = \si{\square\meter}, \qquad \SI{1}{\barn} = \SI{e-28}{\square\meter}
\end{align}
So what is the cross section for a process $AB \ra CD$?

In the laboratory system:
\begin{center}
    \begin{tikzpicture}
        \draw[thick, -latex] (0,0) node [left] {$N_A$} -- node [above] {$\vec v_A$} (3,0);
        \draw (3.1,-1.5) rectangle (3.3,1.5);
        \draw[thick, -latex] (3.4,0.15) -- (6.4,1.55) node [right] {$\dd{N_C} = \dd{N_A}$};
        \draw[thick, dashed,-latex] (3.4,0) -- (6.4,0) node [right] {$N_A - \dd{N_A}$};
        \node [below] at (3.2,-1.5) {$n_B$};
        \node [above] at (3.2,1.5) {$\rightarrow \dd{x} \leftarrow$}; 
    \end{tikzpicture}
\end{center}
absorption:
\begin{align}
    \dd{N_A} = - N_A \frac{\dd{x}}{\lambda} = -N_A \dd{x} n_B \sigma
\end{align}
\begin{compactitem}
    \item[with] $\lambda$: scattering length
    \item[] $\sigma$: cross section
    \item[] $n_B$: target density
\end{compactitem}
\begin{align}
    \ra \quad N_A \qty(\Delta x) = N_A \qty(0) \mr e^{-n_B \sigma \Delta x} \approx N_A\qty(0) \qty(1-n_B \sigma \Delta x)
\end{align}
From the setup it is clear that the number of absorbed particles is the same as the number of scattered particles. Thus:
\begin{align}
    N_C \qty(\Delta x) = N_A\qty(0) n_B \Delta x \cdot \sigma = L \cdot \sigma 
\end{align}
Here we define the \tb{integrated luminsity} $L$ ($\qty[L] = \si{\per\square\meter}$). It covers beam and target-related properties.

\paragraph{Connection to the transition rate $W_\mr{fi}$?}
Consider the interaction of particles $A$ and $B$ in volume $V$:
\begin{center}
    \begin{tikzpicture}
        \draw[thick, -latex] (0,0) node [left] {$A$} -- node [above left] {$\vec v_A$} (2,0);
        \draw[fill=black] (2.25,0) circle (0.1) node [above right] {$B$};
        \draw (1.25,-0.5) rectangle (3.25,0.5);
        \draw[latex-latex] (1.25,-0.75) -- node [below] {$\Delta x$} (3.25,-0.75);
        \draw[dashed] (2.5,0) -- (4.5,0);
    \end{tikzpicture}
\end{center}
The interaction time (particle $A$ is in $V$):
\begin{align}
    \Delta T & = \frac{\Delta x}{\qty|\vec v_A|} \nonumber \\
    \sigma & = \frac{N_C}{L} = \frac{N_C}{\Delta T\, V}\frac{\Delta T \, V}{N_A \Delta x \, n_B} = W_\mr{fi} \frac{1}{n_A} \frac{1}{\qty|\vec v_A|} \frac{1}{n_B} \qty(\# \text{final states}) \nonumber \\
    & \ra \quad \boxed{\sigma = \frac{W_\mr{fi}}{n_A \qty|\vec v_A| n_B} \qty(\# \text{final states}) }
\end{align}
\begin{compactitem}
    \item[with] $n_A \qty|\vec v_A|$: flux density of particles A (dimension: \si{\per \square \meter \per \second})
    \item[] $n_A = n_B = \nicefrac 1V$: one particle each in the volume
\end{compactitem}

\paragraph{What is the number of final states?}
Assuming fermions, each final state particle occupies phase volume $\mr h^3 = \qty(2\pi \hbar)^3$ in 6-dimensional phase space
\begin{align}
    \# \text{states} = \frac{V \dd[3]{p}}{\qty(2\pi \hbar)^3} \cdot \underbrace{\frac{1}{\# \text{particles in } V}}_{\substack{\equiv 1 \text{ (wave} \\ \text{function normalisation)}}}\,.
\end{align}
The cross section $\dd{\sigma}$ for scattering final states into momentum elements $\dd[3]{p_C} \dd[3]{p_D}$
\begin{align}
    \dd{\sigma} & = \frac{W_\mr{fi} V^2}{\qty|\vec v_A|} \cdot \frac{V \dd[3]{p_C}}{\qty(2\pi)^3}\frac{V \dd[3]{p_D}}{\qty(2\pi)^3} \nonumber \\
    & = \underbrace{\frac{1}{\qty|\vec v_A| 2 E_A 2 E_B}}_{\text{flux factor } \mc F} \qty|m|^2 \underbrace{\qty(2\pi)^4 \delta^4 \qty(\fvec p_C + \fvec p_D - \fvec p_A - \fvec p_B) \frac{\dd[3]{p_C} \dd[3]{p_D}}{\qty(2\pi)^6} \frac{1}{E_C E_D}}_\text{Lorentz-invariant phase space factor}
\end{align}
The flux factor transfers all properties of incoming particles. In short the above expression is
\begin{align}
    \dd{\sigma} = \frac{\qty|m|^2}{\mc F} \dd{Q}\,.
\end{align}
For general colinear collisions:
\begin{align}
    \mc F = 4 \sqrt{\qty(\fvec p_A \fvec p_B)^2 - m_A^2 m_B^2}
\end{align}
This factor additionally is Lorentz-invariant. In the centre-of-mass system (CMS):
\begin{align}
    \boxed{\mc F = 4 \qty| \vec p_A| \sqrt s = 4 \qty|\vec p_i| \sqrt s}\\
    (\text{for } \vec p_A = - \vec p_B, \ \qty|\vec p_A| = \qty|\vec p_B| = \qty|\vec p_i|)\nonumber 
\end{align}

%\chapter{Atomkerne und Kernmodelle}
\begin{itemize}
\item[$\ra$] Geometrie von Kernen
\item[$\ra$] Kernzusammensetzung
\item[$\ra$] Modelle für Kernmassen etc.
\end{itemize}

\section{Kernradien und Formfaktor}
\begin{itemize}
\item Experimentelle Untersuchung der Kerngrößen durch Streuexperimente
\begin{align}
\begin{matrix}
(e^- , \alpha) & A & \ra & (e^- , \alpha)\ A & \text{(el. Ladung)}\\
\underbrace{\ n \ } & \underbrace{\ A \ } & \underbrace{\ \ra \ } & n\ A & \text{(Massendichte)} \\
 \text{\glqq Sonde\grqq} & \text{Kern} & \text{el. Streuung} & & 
\end{matrix}
\end{align}
Typische Energien: $E_\mr{kin}\sim 1 \text{ bis } 100\,\mr{MeV}$\\
Gute Näherung: $E_\mr{kin} \ll M_A$\\
$\Rightarrow$ Energieübertrag auf Kern ist vernachlässigbar
\begin{figure}[!ht]
	\centering
	\begin{tikzpicture}
	\begin{feynman}
	\vertex (a1);
	\node[right = 3cm of a1, blob] (a2) {A};
	\vertex[above right = 3cm of a2] (c1);
	\vertex[below right = 3cm of a2] (d1);
	
	\diagram*{
	(a1) -- [fermion, edge label = {$e$,$\alpha$,$\vec p$}] (a2),
	(a2) -- [fermion, edge label = {$e$,$\alpha$,$\vec p^{\;\prime}$}] (c1),
	(a2) -- [fermion] (d1)
	};
	\end{feynman}
	\end{tikzpicture}
	\caption{Energieübertrag auf Kern durch Stoß\label{fig:4.1}}
\end{figure}
\begin{align}
E^A_\mr{kin}=\frac{q^2}{2M_A} \ll \frac{p^2}{2m_{\alpha,e}}=E^{\alpha,e}_\mr{kin}
\end{align}
(Achtung: für $\alpha$-Streuung an leichten Kernen nicht gültig, da $\labs\vec{p}\rabs = \labs \vec{p}^\prime\rabs$)\\
\begin{figure}[!ht]
	\centering
	\includegraphics[width=.35\textwidth]{imgs/ep5-fig-4-2.pdf}
	\caption{Impulsübertrag $q$ eines elastisch gestreuten Teilchens \label{fig:4.2}}
	\end{figure}
Aus Abb. \ref{fig:4.2} ergibt sich für $q^2= (\vec p - \vec{p}^\prime)^2$:
\begin{align}
q^2=\lb 2\labs \vec{p}\rabs \sin\frac{\theta}{2}\rb ^2 = \lb  2 p \sin \frac{\theta}{2}\rb ^2
\end{align}
\begin{figure}[!ht]
	\centering
	\begin{tikzpicture}
    \begin{feynman}
    \vertex (a1);
    \vertex [below right= 3cm of a1] (a2);
    \vertex [above right= 3cm of a2] (a3);
    \node [below= of a2, blob] (b2);
    
    \diagram*{
    (a1) -- [fermion, edge label = $e^-$] (a2),
    (a2) -- [fermion, edge label = $e^-$] (a3),
    (a2) -- [photon, edge label = $\gamma$] (b2),
    };
    \end{feynman}
    \end{tikzpicture}
	\caption{Feynman-Diagramm der elektromagnetischen Wechselwirkung mit dem Kern $A$\label{fig:4.3}}
	\end{figure}
    
$\Rightarrow$ der WQ ist elm. (vgl.Abb.\ref{fig:4.3}), daher Coulomb-Streuung
\begin{align}
\begin{split}
\frac{\Pa \sigma}{\Pa \Omega}\sim (e^2)(Ze^2)\frac{1}{q^4}\sim Z^2\alpha^2\frac{1}{q^4}\\
q^2 \text{ in } q^4 \text{ gegeben durch } q^2=\underbrace{\lb E-E^\prime \rb ^2}_{=0}-\vec{q}^2\\
\Rightarrow q^2=-4p^2\sin^2\frac{\theta}{2}\\
\Rightarrow \boxed{\frac{d\sigma}{d\Omega}\sim \frac{Z^2\alpha^2}{16p^4 \sin^4 \frac{\theta}{2}}}\\
\sim\text{ Rutherford-WQ}
\end{split}
\end{align}
\begin{itemize}
\item[$\ra$] starke $p$- und $\theta$-Abhängigkeit
\item[$\ra$]  maximales $\labs \vec{q} \rabs$ für $\theta=180^{\circ}$
\begin{itemize}
\item[$\ra$]  stärkste Annäherung
\item[$\ra$]  kleinster WQ
\end{itemize}
\item[$\ra$]  $e^-$ relativistisch
\item[$\ra$]  $\alpha$ nicht-relativistisch
\end{itemize}

\item Was ändert sich, wenn A ausgedehnte Ladungsverteilung $\rho\lb \vec{r}\rb $ hat?
\begin{align}
\left. \frac{\Pa \sigma}{\Pa \Omega}\right|_\mr{Coul}\rightarrow \left. \frac{\Pa \sigma}{\Pa \Omega}\right|_\mr{Coul}\cdot \left|F\lb \vec{q}\rb \right|^2
\end{align}

Hierbei ist der Formfaktor $F\lb \vec{q}\rb $ eingeführt worden. Er errechnet sich als Fouriertransformierte der Ladungsverteilung.
\begin{align}
\boxed{F\lb \vec{q}\rb =\frac{1}{Ze}\int e^{i\vec{q}\vec{r}}\rho \lb \vec{r}\rb  \Pa^3 r}
\end{align}
(Offenbar gleich zu Beugungseffekt: QM Störungsrechnung (mehr in Kap. 6))
\begin{itemize}
\item Messung von $\frac{\Pa \sigma}{\Pa \Omega}$
\item[$\ra$] Bestimmung von $\labs F\lb \vec{q}\rb  \rabs ^2$
\item[$\ra$] Bestimmung von $\rho(\vec{r})$
\end{itemize}

Ergebnis für $\rho(\vec{r})=\rho\lb r\rb $ bei homogener Ladungsverteilung (Abb. \ref{fig:4.4}):
\begin{align}
\boxed{\rho\lb r\rb =\frac{\rho_0}{1+e^{\frac{r-a}{b}}}}
\end{align}
Typische Werte der Parameter $a$, $b$:
\begin{align}
\boxed{a=1.07 \mr{A^{\nicefrac{1}{3}}\,fm} \qquad b=0.54\,\mr{fm}}
\end{align}
Gleiches $\rho_0$ homogener Ladungsverteilung, scharfer Rand:
\begin{align*}
\boxed{R_A=1.21 A^{\nicefrac{1}{3}}\,\mr{fm}}
\end{align*}
\end{itemize}
\begin{figure}[!ht]
	\centering
	\includegraphics[width=.35\textwidth]{imgs/ep5-fig-4-4.pdf}
	\caption{Homogene Ladungsverteilung in einem Teilchen/Kern\label{fig:4.4}}
	\end{figure}

\section{Kernaufbau und Bindungsenergie}
Ein Atomkern besteht aus
\begin{align*}
\boxed{ \begin{matrix}
Z & \text{Protonen (p)}\\ N & \text{Neutronen (n)}\\ Z + N = A & \text{Nukleonen (p+n)}
\end{matrix} \ A: \ \text{Massenzahl} }
\end{align*}
Schreibweise: $^A_Z X_N$ mit $X$ als Elementsymbol\\
z.B. $^{12}_6 C_6$, auch $^{12}_6 C$, $^{12}C\ \lsb C12,12C\rsb $ \\
\begin{figure}[!ht]
	\centering
	\includegraphics[width=.5\textwidth]{imgs/ep5-fig-4-5.pdf}
	\caption{Nuklidkartenskizze in der N-Z-Ebene \label{fig:4.5}}
	\end{figure}

\begin{itemize}
\item[$\ra$] \textbf{Bindungsenergie}\\
Energie $\mathcal{O}(10\,MeV)$ nötig, um Nukleonen aus Kern zu lösen.\\
Gesamte Bindungsenergie eines Atoms:
\begin{align}
E_B(A,Z)=\underbrace{E(Zp+Nn+Ze^-)}_{ZM_p+NM_n+Zm_e}-\underbrace{E(^A_Z X_N)}_{M(^A_Z X_N)=M(A,Z)}
\end{align}
Oft:
\begin{align}
\boxed{\Delta M=-E_B \,\widehat{=}\, \text{Massendefekt}}
\end{align}
Messung von $E_B$ erfordert präzise Messung von Atom/Kernmassen
\begin{itemize}
\item[$\Ra$] Massenspektroskopie (Ablenkung in $E$- und $B$-Feldern, $\frac{\Delta M}{M}\sim \mathcal{O}(10^{-6})$)
\item[$\Ra$] Ergebnis: $E_B \simeq 1\%$ von $M(A,Z)$
\item[$\lt$] viel größer als bei elm. WW ($10^{-6}$ bis $10^{-8}$ bei Atomen)
\end{itemize}
\begin{figure}[!ht]
	\centering
	\includegraphics[width=.5\textwidth]{imgs/ep5-fig-4-6.pdf}
	\caption{Bindungsenergie pro Nukleon über die Nukleoenenzahl aufgetragen. Erkennbar ein Maximum, zu welchem Kerne fusioniert werden können zur Energiegewinnung und oberhalb dessen Kerne gespalten werden \label{fig:4.6}}
	\end{figure}
\begin{itemize}
\item[$\ra$] stabile Kerne mit $A\leq 60$:\\
Energiegewinn durch Fusion (Sonne!)
\item[$\ra$] stabile Kerne mit $A>60$:\\
Energiegewinn durch Spaltung (Kernkraftwerke, -waffen)
\end{itemize}
Können wir $\frac{E_B}{A}$ verstehen?
\begin{itemize}
\item[$\ra$] wenn starke WW langreichweitig (wie Coulomb)\\
$E_B\sim \frac{1}{2}A(A-1)\leftarrow$ Zahl der Nukleonenpaare\\
Aber: $E_B\sim A$
\item[$\Ra$] $\boxed{\text{starke WW kurzreichweitig (Nukleon \glqq sieht\grqq nur nächste Nachbarn)}}$
\end{itemize}
\end{itemize}

\section{Tröpfchenmodell und Weizsäcker-Massenformel}
Das sogenannte Tröpfchenmodell war das erste Modell der Kernphysik zur Beschreibung deren Eigenschaften. Da es das erste ist, ist es auch durchaus inkomplett.\footnote{Der Physiker Weizsäcker ist der Bruder des ehemaligen Bundespräsidenten}

\begin{itemize}
\item[$\rightarrow$] Kerndichte $\approx \ const.$
\item[$\rightarrow$] Kerne sphärisch
\item[$\rightarrow$] \glqq Verdampfungsenergie\grqq{} $\sim \ M$
\item[$\Rightarrow$] Wie bei (Wasser)tröpfchen!
\end{itemize}
Modell aus 1930er Jahren (formuliert von Weizsäcker, Williams, Gamov, Bohr):
\begin{align}
\boxed{ E_B = a_v A - a_s A^{\nicefrac{2}{3}} - a_c \frac{Z^2}{A^{\nicefrac{1}{3}}} - a_a \frac{(N-Z)^2}{A} - \frac{\delta}{A^{\nicefrac{1}{2}}}}
\end{align}
\begin{compactitem}
\item[mit] $a_vA \ =$ \textbf{Volumenterm}\\
$a_v$ Bindung der Nukleonen an ihre Nachbarn
\item[] $-a_sA^{\nicefrac{2}{3}}\ = $ \textbf{Oberflächenterm}\\
$a_s \approx 18\,$MeV, Nukleonen an der Oberfläche haben weniger Nachbarn, Effekt $\sim R^2 \sim A^{\nicefrac{2}{3}}$ mit $R^3 \sim A$
\item[] $-a_c \frac{Z^2}{A^{\nicefrac{1}{3}}} \ =$ \textbf{Coulombterm}\\
$a_c \approx 0.7\,$MeV, Energie einer homogen geladenen Kugel $\sim \frac{Q^2}{R}$ (Abstoßung der p)
\item[] $-a_a\frac{(N-Z)^2}{A}\ = $ \textbf{Asymmetrieterm}\\
$a_a \approx 23$\,MeV, p, n haben Spin $\nicefrac{1}{2}$ (Fermionen)\\
$\Rightarrow$ Pauli-Prinzip
\item[] $-\frac{\delta}{A^{\nicefrac{1}{2}}} \ =$ \textbf{Paarungsterm},
\begin{align}
\delta = \left\lbrace \begin{matrix} -11\,\mathrm{MeV}& gg\\ 0 & gu,\ ug\\ +11\,\mathrm{MeV} & uu \end{matrix}\right.
\end{align}
$u=$ ungerade, $g=$ gerade, $gg=$ $N$ gerade und $Z$ gerade
\end{compactitem}

\begin{figure}[!ht]
\centering
\includegraphics[width=.5\textwidth]{imgs/ep5-fig-4-7.pdf}
\caption{Umwandlung eines Protons in ein Neutron \label{fig:4.7}}
\end{figure}

Gepaarte Nukleonen sind stärker gebunden als \glqq einzelne\grqq{}
\begin{itemize}
\item[$\leadsto$] Modell kann $E_B(A,Z)$ grob beschreiben (insbesondere auch die Region der stabilen Kerne).
\item[$\rightarrow$] Aber:
\begin{compactitem}
\item QM-Effekte \glqq von Hand\grqq
\item Keine Dynamik (Nukleonen im Kern bewegen sich: $\Delta x \cdot \Delta p \gtrsim \hbar$)
\item Keine feineren Strukturen in $E_B(A,Z)$
\item keine Aussagen über Spin, magn. Moment, Parität
\end{compactitem}
\item[$\leadsto$] Interessant: Mit Gravitations- statt Coulombterm $\rightarrow$ Neutronenstern $\approx$ stabile Lösung
\end{itemize}

\section{Das Fermigasmodell}
\begin{figure}[!ht]
\centering
\includegraphics[width=.5\textwidth]{imgs/ep5-fig-4-8.pdf}
\caption{Potentialtopf nach dem Fermigasmodell \label{fig:4.8}}
\end{figure}
Nukleonen gebunden durch WW mit allen anderen Nukleonen.
\begin{itemize}
\item[$\rightarrow$] Effektives Potential
\item[$\rightarrow$] Einfache Näherung: Kastenpotential
\item[$\leadsto$] Wie groß sind $E_F$, $p_F= \sqrt{2M_NE_F}$, $E_0$, $\nicefrac{E_B}{A}$
\item[$\leadsto$] Achtung: $E_0^{(n)} < E_0^{(p)}$ wegen Coulomb-WW der $p$
\begin{compactitem}
\item[$\leadsto$] zunächst vernachlässigt
\end{compactitem}
\end{itemize}
Erinnerung an Quantenmechanik:
\begin{center}
$\boxed{\text{QM: 1 Zustand pro Phasenraumvolumen }(\Delta x\, \Delta p)^3 = \mathrm{h}^3}$
\end{center}
Gesamtes Phasenraumvolumen im Kern:
\begin{align}
\Delta x^3 \rightarrow V_K = \frac{4 \pi}{3} R_n^3\\\Delta p^3 \rightarrow V_p = \frac{4 \pi}{3}p_F^3
\end{align}
\begin{itemize}
\item[$\Rightarrow$] Zahl der Zustände
\begin{align}
\boxed{n= \frac{1}{h^3} \lb  \frac{4\pi}{3} R_n^3\rb \lb \frac{4\pi}{3} p_F^3\rb = \frac{N+Z}{4}= \frac{A}{4}}
\end{align}
\begin{compactitem}
\item[mit] $R_n = R_0 A^{\nicefrac{1}{3}}$, $R_0 = 1.21$\,fm
\end{compactitem}
\begin{align}
\Rightarrow\ \boxed{p_F = \frac{\hbar}{R_0}\lb \frac{9 \pi}{8}\rb ^{\nicefrac{1}{3}} = \frac{197\,\mathrm{MeV\,fm}}{1.21\,\mathrm{fm}}\lb \frac{9 \pi}{8}\rb ^{\nicefrac{1}{3}} = 250\,\mathrm{MeV}}
\end{align}
\item[$\leadsto$] hoher Impuls, etwa wie von Unschärferelation erwartet $\lb \Delta p \sim \frac{\hbar}{R_0}\rb $
\item[$\leadsto$] durch $eA$-Streuung bestätigt
\begin{align}
\lt \ \boxed{{E_F} = \frac{p_F^2}{2M_N} \approx 33\,\mathrm{MeV} \Rightarrow\ E_0 = E_F+ \frac{E_B}{A} \approx 40 \, \mathrm{MeV}}
\end{align}
\item[$\leadsto$] Genauer: mit Coulombpotential
\item[$\leadsto$] Rechnung mit $p_F^{(n)} \neq p_F^{(p)}$ ergibt:
\begin{align}
E_\mathrm{min}^\mathrm{tot} \sim A + \frac{5}{9} \frac{(N-Z)^2}{A}
\end{align}
zweiter Summand erklärt Asymmetrieterm
\item[$\leadsto$] Paarungsterm: Beispiel $^{40}$K, $^{40}$Ca
\end{itemize}

\begin{figure}[!ht]
\centering
\includegraphics[width=.5\textwidth]{imgs/ep5-fig-4-9.pdf}
\caption{Potentialtopf mit Berücksichtigung der verschiedenen Bindungsenergien für Neutronen und Protonen \label{fig:4.9}}
\end{figure}

\begin{figure}[!ht]
\centering
\includegraphics[width=.5\textwidth]{imgs/ep5-fig-4-10.pdf}
\caption{Potentialtopf am Beispiel für Kalium und Calcium \label{fig:4.10}}
\end{figure}

Fermigasmodell:
\begin{compactitem}
\item einfachstes QM-Modell von Kernen
\item erklärt qualitativ QM-Terme in Weizsäcker-Formel
\item aber: keine Vorhersagen zu Spin, magn. Moment und Parität
\end{compactitem}

\section{Das Schalenmodell}
Einteilchen-Wellenfunktion in effektivem Kernpotential $\lt$ Schrödinger-Gleichung
\begin{itemize}
\item \textbf{Potential}\\
$V_S (r) \sim \varrho_K (r) \ =$ Nukleonendichte
\begin{figure}[!ht]
\centering
\includegraphics[width=.5\textwidth]{imgs/ep5-fig-4-11.pdf}
\caption{Skizze des Woods-Saxon-Potentials \label{fig:4.11}}
\end{figure}
\begin{align}
\boxed{
V_S(r) = -V_0 \frac{1}{1+e^{\nicefrac{(r-a)}{b}}}
}\\
\sim \text{ Woods-Saxon-Potential}\nonumber
\end{align}
Achtung:
\begin{compactitem}
\item effektives Potential
\item keine zentrale \glqq Kraftquelle\grqq
\item Kräfte \textbf{viel} stärker als in Atom
\end{compactitem}
\item[$\lt$] SG analytisch nicht lösbar
\item[$\lt$] Näherung als harmonischer Oszillator\\
$V_S\lb r\rb  = -V_0 + \frac{1}{2}kr^2, \ r \leq R_k$
\item[$\lt$] In jedem Fall: $V_S\lb  \vec{r}\rb  = V_S(r)$ (sphärisch symmetrisch)
\begin{compactitem}
\item[$\Ra$] Winkelanteil $Y_{lm} (\vartheta, \varphi )$
\item[$\Ra$] Hauptquantenzahl $n$
\item[$\Ra$] $E= \underbrace{E(n,l)}_{2(2l+1)\text{-fach entartet}}$\\
Vorfaktor 2 durch 2 Fermionen pro WF, $2l+1$ Werte für $m$
\end{compactitem}
\begin{align}
\boxed{ E(n,l) = \lb N+\frac{3}{2}\rb  E_0 + \underbrace{\Delta E \lb  n,l\rb }_{=0\text{ für h.O.}}}\\
N=2(n-1)+l\nonumber \\
\Delta E(n,l) = \llb \begin{matrix} \text{klein für kleine } n \text{ und große } l \\ \text{groß für große } n \text{ und kleine } l \end{matrix} \right.
\end{align}
\item[$\Ra$] Baue \glqq Schalen\grqq{} (wie in Atomphysik)
\begin{table}
\centering
\begin{tabular}[!ht]{c|cccc}
$N$ & $nl$ & $2(2l+1)$ & $\sum 2(2l+1)$ & beob?\\
\hline
0 & 1s & 2 &2 & Ja\\
1 & 1p & 6 & 8 & Ja\\
2 & 1d & 10 & 18 & Nein\\
2 & 2s & 2 & 20 & Ja\\
3 & 1f & 14 & 34 & Nein\\
3 & 2p & 6 & 40 & Nein
\end{tabular}
\end{table}
\item[$\ra$] Was fehlt? LS-Kopplung\\
\textbf{Wichtig:} Erfolgt durch \textbf{starke} WW $\Ra$ großer Beitrag zu $V(r)$
\begin{compactitem}
\item[$\lt$] $V_{LS}(r) = \underbrace{\lla V_{LS}\rra}_{<0}\lla \vec{l}\vec{s}\rra$
\item[$\lt$] $\vec{j} = \vec{l} + \vec{s}$ wird bevorzugt maximal\\
$\Ra \ \lla \vec{l}\vec{s}\rra = \frac{1}{2} \lsb  \lla \vec{j}^2\rra - \lla \vec{l}^2\rra - \lla \vec{s}^2 \rra \rsb $\\
$\lla \vec{j}^2\rra = j(j+1)$, $\lla \vec{l}^2\rra = l(l+1)$, $\lla \vec{s}^2\rra = s(s+1)$
\end{compactitem}
\begin{align}
\Ra \ \boxed{ \lla \vec{ls} \rra = \llb \begin{matrix}
\frac{l}{2} \text{ für } j = l+\frac{1}{2}\\
-\frac{l+1}{2} \text{ für } j = l - \frac{1}{2}
\end{matrix} \right. }
\end{align}
\item[$\Ra$] Termschema mit besonders großen Lücken bei
\begin{align*}
\boxed{
\begin{matrix}
 & \text{He} & \text{O} & \text{Ca} & \text{Ni} & \text{Sn} & \text{Pb}\\
 Z = & 2, & 8, & 20, & 28, & 50, &82\\
 N= & 2, & 8, & 20, & 28, & 50, & 82, & 126
\end{matrix}
}
\end{align*}
sogenannte \glqq magische Zahlen\grqq{} $\lt$ Schalen
\item[$\Ra$]  Kerne mit abgeschlossenen Schalen besonders stabil (\glqq magisch\grqq)
\item[$\Ra$] Noch stabiler: \glqq doppelt-magische\grqq{} Kerne:\\
$\boxed{^4_2\text{He}_2, \ ^{16}_8\text{O}_8, \ ^{40}_{20}\text{Ca}_{20}, \ ^{48}_{20}\text{Ca}_{28}, \ ^{208}_{82}\text{Pb}_{126}}$
\item[$\Ra$] Nomenklatur $nL_j$\\ z.B. 2$P_{\nicefrac{3}{2}}$
\item[$\Ra$] Wie in Atomphysik: abgeschlossene Schalen haben Spin, magnetisches Moment $=0$
\item[$\Ra$]  Verhalten von Kernen mit 1 zusätzlichen Nukleon wird durch dieses bestimmt (\glqq Leuchtnukleon\grqq)\\
\textbf{Beispiel:}
\begin{align*}
^{17}_8\text{O}_9, \ Z= 8:\qquad \underbrace{\underbrace{(1S_{\nicefrac{1}{2}})}_2 \underbrace{(1P_{\nicefrac{3}{2}})}_4 \underbrace{(1P_{\nicefrac{1}{2}})}_2}_{\text{abgeschlossen}}\\
N= 9 :\qquad (1S_{\nicefrac{1}{2}})(1P_{\nicefrac{3}{2}})(1P_{\nicefrac{1}{2}}) + (1D_{\nicefrac{5}{2}})\\
\Ra \text{ Spin} \lb  ^{17}_8\text{O}_9\rb  = \underbrace{\text{Spin}\lb  ^{16}_8\text{O}_8\rb }_{=0} + \frac{5}{2}\\
\Ra \text{ magn. Mom.: } \mu\lb  ^{17}_8\text{O}_9\rb  = \mu\lb  ^{16}_8\text{O}_8\rb  + \mu_n  = -1.91\mu_N
\end{align*}
\end{itemize}

%\chapter{Kernzerfälle und -spaltung}
\begin{compactitem}
\item Verschiedene Zerfallsmoden
\item (Historisch) erster Zugang zur schwachen WW
\item Kernspaltung $\lt$ Kettenreaktionen
\end{compactitem}
\section{\texorpdfstring{$\alpha$}{}-Zerfall von Kernen}
Emission eines $\alpha$-Teilchens $= \ ^4_2\text{He}_2$-Kern (He-Kern doppelt magisch)
\begin{align}
\boxed{
^A_Z X_N \ra ^{A-4}_{Z-2} Y_{N-2} + \alpha
}
\end{align}
\begin{figure}[!ht]
\centering
\includegraphics[width=.4\textwidth]{imgs/ep5-fig-5-1.pdf}
\caption{$\alpha$-Zerfall in der Nuklidebene \label{fig:5.1}}
\end{figure}
\begin{itemize}
\item Energiebilanz:
\begin{align}
\boxed{
M(A,Z) = M(A-4, Z-2) + M_\alpha + Q
}
\end{align}
$Q\ =\ Q$-Wert
\item[$\lt$] Zerfall ist möglich, wenn $Q>0$
\item[$\lt$] Wegen $M_\alpha \overset{\text{i.d.R.}}{\ll} M(A-4, Z-2)$ ist $Q\approx T_\alpha$
\item[$\lt$] Bei p, n-Emission ist $Q$ um $E_B(\alpha)$ kleiner
\begin{compactitem}
\item[$\lt$] kommt nur sehr selten vor
\end{compactitem}
\item[$\lt$] 2-Körper-Endzustand
\begin{itemize}
\item[$\Ra$] für gegebenen Zerfall hat $T_\alpha$ immer den gleichen Wert
\item[$\Ra$] \glqq Linienspektrum\grqq
\end{itemize}
Deutung/Interpretation:\\
Im Kern formiert sich $\alpha$-Teilchen, das Kern durch Tunnelvorgang verlassen kann

\begin{minipage}[c]{.45\textwidth}
\captionsetup{type=figure}
\includegraphics[width=\textwidth]{imgs/ep5-fig-5-2.pdf}
\captionof{figure}{Potentialtopf mit Coulomb- und Tunnelbarriere \label{fig:5.2}}
\end{minipage}
\begin{minipage}[c]{.45\textwidth}
\captionsetup{type=figure}
\includegraphics[width=\textwidth]{imgs/ep5-fig-5-3.pdf}
\captionof{figure}{Tunnelbarriere für ein $\alpha$-Teilchen aus dem Kernpotential \label{fig:5.3}}
\end{minipage}

\item \textbf{Tunnelwahrscheinlichkeit} für Barriere mit Breite $\Pa r$ und Höhe $v(r)$
\begin{align}
\boxed{
\Pa p_i = \exp \left( - \frac{2}{\hbar} \sqrt{2 m_\alpha (v(r) -Q)}\ \Pa r \right)
}
\end{align}
Damit ergibt sich die gesamte Tunnelwahrscheinlichkeit zu 
\begin{align}
P \ = \prod_i \Pa p_i = \exp \left( - \frac{2}{\hbar} \int_{R_k}^{r_C} \sqrt{ 2 m_\alpha \left(v(r) -Q\right)} \Pa r \right) = \exp \left(-2G\right)\\
\boxed{G = \frac{1}{\hbar} \int_{R_k}^{r_C} \sqrt{2 m_\alpha \left(v(r) -Q\right)}\Pa r}\\
\sim \ \text{Gamov-Faktor}\nonumber
\end{align}
Annahme (starke Näherung):
\begin{align}
\begin{split}
V(r) = V_C(r), \ r>R_k\\
\Ra V(r) = \frac{2 \left( Z-2\right)\alpha}{r} = \frac{r_C}{r} \underbrace{V\left(r_C\right)}_Q = \frac{r_C}{r} \underbrace{Q}_{T_\alpha}
\end{split}
\end{align}
Hier ohne Rechnung:
\begin{align}
\begin{split}
G = \sqrt{\frac{2 m_\alpha}{T_\alpha}}\cdot 2 \cdot \left(Z-2\right) \cdot \alpha \cdot f\left(\frac{r_C}{R_k}\right)\\
G \approx \frac{2 \pi \alpha \left(Z-2\right)}{\beta_\alpha}
\end{split}
\end{align}
\item Zerfallskonstante:
\begin{align}
\boxed{ \lambda = \omega \cdot \frac{\beta_\alpha}{2 R_k} \cdot e^{-2G}}
\end{align}
\begin{compactitem}
\item[mit] $\omega$: Wskt., dass ein $\alpha$ im Kern gebildet wird
\item[] $\frac{\beta_\alpha}{2R_k}$: Rate der Tunnelversuche
\item[] $e^{-2G}$: Tunnelwahrscheinlichkeit
\end{compactitem}
\begin{align}
\ln \lambda = - \ln \tau = - a_1 \frac{Z-2}{\sqrt{T_\alpha}} + a_2 \dots\\
\sim \ \text{Geiger-Nutall-Regel}\nonumber
\end{align}
\item \tb{Anmerkungen:}
\begin{itemize}
\item $V_C$, $T_\alpha$ sind sehr unterschiedlich für $\alpha$-Strahlung:
\begin{itemize}
\item[$\Ra$] Großer Wertebereich für $e^{-2G}$
\item[$\Ra$] Lebensdauer $\boxed{\tau = 10^{-8}\,\mr{s} \dots 10^{17}\,\mr{a}}$
\end{itemize}
\item Die Rechnung hängt von der genaueren Form von $V(r)$ ab
\item Nicht berücksichtigt: Bahndrehimpuls (Zentrifugalbarriere)
\end{itemize}
\end{itemize}

\section{Beta-Zerfall von Kernen\label{sec:5.2}}
\begin{itemize}
\item[$\ra$] $\beta \ \hat{=}\ e^-\text{ oder }e^+$
\item[$\ra$] 3 Varianten
\item \tb{$\beta^-$-Zerfall}
\begin{align}
\boxed{ \underbrace{^A_ZX_N}_{M\left(A,Z\right)} \longrightarrow \underbrace{^A_{Z+1} X^\prime_{N-1} + e^-}_{M\left(A,Z+1\right)} + \bar{\nu}_e }
\end{align}
Energetisch möglich, wenn:
\begin{align}
\boxed{ M\left(A,Z\right) > M\left(A,Z+1\right) + m_\nu } \qquad m_\nu \approx 0
\end{align}
\begin{itemize}
\item Zerfall durch schwache Wechselwirkung
\begin{figure}[!ht]
    \centering
    \begin{tikzpicture}
        \begin{feynman}
            \vertex (a1) {$u$};
            \vertex[right=6cm of a1] (a2) {$u$};
            \vertex[right=3cm of a1] (a3);
            \vertex[below=2em of a1] (b1) {$d$};
            \vertex[below=2em of a2] (b2) {$d$};
            \vertex[right=3cm of a1] (b3);
            \vertex[below=2em of b1] (d1) {$u$};
            \vertex[right=3cm of d1] (d2);
            \vertex[below=2em of b2] (d3) {$d$};
            %% Equivalent way to obtain (d):
            % \vertex at ($(b2)!0.5!(b3) + (0, -0.5cm)$) (d);
            \vertex[below=of d3] (c1) {$e^-$};
            \vertex[below=2em of c1] (c3) {$\bar \nu_e$};
            \vertex at ($(c1)!0.5!(c3) - (2cm, 0)$) (c2);
            \diagram* {
            (b1) -- [fermion] (b2),
            (d1) -- [fermion] (d2) -- (d2) -- [fermion] (d3),
            (c3) -- [fermion] (c2) -- [fermion] (c1),
            (d2) -- [scalar, edge label=$W^-$] (c2),
            (a1) -- [fermion] (a2),
            };
            \draw [decoration={brace}, decorate] (d1.south west) -- (a1.north west)
            node [pos=0.5, left] {$n$};
            \draw [decoration={brace}, decorate] (a2.north east) -- (d3.south east)
            node [pos=0.5, right] {$p$};
        \end{feynman}
    \end{tikzpicture}
    \caption{Feynmandiagramm des $\beta^-$-Zerfalls \label{fig:5.4}}
\end{figure}
\item 3-Körper-Zerfall\\
$\Ra$ kontinuierliches Spektrum, kontinuierliche Energieverteilung auf Elektron und Neutrino (später genauer)
\item Möglichkeit, z.B. für freie Neutronen
\begin{align}
n \ra p + e^- + \bar{\nu}_e \qquad \tau = 880\,\mr{s}
\end{align}
\end{itemize}

\item \tb{$\beta^+$-Zerfall:}
\begin{align}
\boxed{ \underbrace{^A_ZX_N}_{M\left(A,Z\right)} \longrightarrow \underbrace{^A_{Z-1} X^\prime_{N+1} + e^+}_{M\left(A,Z-1\right)+2 m_e} + \nu_e }
\end{align}
$2m_e$ durch entstehendes $e^+$ und übriges $e^-$\\
Energetisch möglich:
\begin{align}
\boxed{ M\left(A,Z\right) > M\left(A,Z-1\right) + 2m_e + m_\nu }
\end{align}
\begin{itemize}
\item Feynman-Diagramm (schwacher Zerfall)
\begin{figure}[!ht]
    \centering
    \begin{tikzpicture}
        \begin{feynman}
            \vertex (a1) {$u$};
            \vertex[right=6cm of a1] (a2) {$u$};
            \vertex[right=3cm of a1] (a3);
            \vertex[below=2em of a1] (b1) {$d$};
            \vertex[below=2em of a2] (b2) {$d$};
            \vertex[right=3cm of a1] (b3);
            \vertex[below=2em of b1] (d1) {$u$};
            \vertex[right=3cm of d1] (d2);
            \vertex[below=2em of b2] (d3) {$d$};
            %% Equivalent way to obtain (d):
            % \vertex at ($(b2)!0.5!(b3) + (0, -0.5cm)$) (d);
            \vertex[below=of d3] (c1) {$\nu_e$};
            \vertex[below=2em of c1] (c3) {$e^+$};
            \vertex at ($(c1)!0.5!(c3) - (2cm, 0)$) (c2);
            \diagram* {
            (b1) -- [fermion] (b2),
            (d1) -- [fermion] (d2) -- (d2) -- [fermion] (d3),
            (c3) -- [fermion] (c2) -- [fermion] (c1),
            (d2) -- [scalar, edge label=$W^+$] (c2),
            (a1) -- [fermion] (a2),
            };
            \draw [decoration={brace}, decorate] (d1.south west) -- (a1.north west)
            node [pos=0.5, left] {$p$};
            \draw [decoration={brace}, decorate] (a2.north east) -- (d3.south east)
            node [pos=0.5, right] {$n$};
        \end{feynman}
    \end{tikzpicture}
    \caption{Feynmandiagramm des $\beta^+$-Zerfalls \label{fig:5.5}}
\end{figure}
\item Dieser Zerfall ist nicht möglich für freie Protonen, da $m_p < m_n$. Für gebundene Protonen dagegen wird die Massendifferenz durch die Bindungsenergie kompensiert.
\end{itemize}

\item \tb{Elektron-Einfang} (vgl. Abb.\ref{fig:5.6})\\
Ein $e^-$ aus der Atomhülle wird vom Kern eingefangen.\\
Typisch: \glqq K-Einfang\grqq, Einfang aus der K-Schale
\begin{align}
^A_ZX_N \ra ^A_{Z-1} X_{N+1}^{\prime\,(*)} + \nu_e
\end{align}
Energetisch möglich, wenn
\begin{align}
\boxed{ M\left(A,Z\right) \geq M\left(A,Z-1\right) + \mc{E}}
\end{align}
\begin{compactitem}
\item[mit] $\mc{E}$: Anregungsenergie der Atomhülle des Tochterkerns
\end{compactitem}
\begin{itemize}
\item Auffüllen des Loches in der Elektronenhülle erzeugt charakteristische Röntgenstrahlung

\begin{figure}[!ht]
\centering
    \begin{tikzpicture}
    \begin{feynman}
    \vertex (a1);
    \vertex[below = 5em of a1] (b1) {$u$};
    \vertex[below = 2em of b1] (c1) {$u$};
    \vertex[below = 2em of c1] (d1) {$d$};
    \vertex[right = 6cm of a1] (a2);
    \vertex[below = 5em of a2] (b2) {$d$};
    \vertex[below = 2em of b2] (c2) {$u$};
    \vertex[below = 2em of c2] (d2) {$d$};
    \vertex[right = 3cm of b1] (b3);
    \vertex[above = 4em of b3] (a3);
    
    \diagram*{
    (a1) -- [fermion, edge label = $e^-$] (a3) -- [fermion, edge label = $\nu_e$] (a2),
    (b1) -- [fermion] (b3) -- [fermion] (b2),
    (c1) -- [fermion] (c2),
    (d1) -- [fermion] (d2),
    (a3) -- [scalar, edge label = $W^-$] (b3),
    };
    \draw [decoration={brace}, decorate] (d1.south west) -- (b1.north west)
    node [pos=0.5, left] {$p$};
    \draw [decoration={brace}, decorate] (b2.north east) -- (d2.south east)
    node [pos=0.5, right] {$n$};
    \end{feynman}
    \end{tikzpicture}
\caption{Feynmandiagramm des Elektroneneinfangs \label{fig:5.6}}
\end{figure}
\item Energetisch günstiger als $\beta^+$-Zerfall wegen der fehlenden $2m_e$-Terme
\end{itemize}
\item \tb{Betazerfall und Massenparabeln}\\
Tröpfchenmodell (festes $A$):
\begin{align}
\boxed{ M\left(A,Z\right) \ \hat{=} \ \text{Parabel } \left(Z^2\right)}
\end{align}

\begin{figure}[!ht]
\centering
\includegraphics[width=.75\textwidth]{imgs/ep5-fig-5-7.pdf}
\caption{Massenparabel für verschiedene Ausgangsbedingungen\label{fig:5.7}}
\end{figure}

\item \tb{Betazerfall in der Nuklid-Ebene}

\begin{figure}[!ht]
\centering
\includegraphics[width=.5\textwidth]{imgs/ep5-fig-5-8.pdf}
\caption{Umwandlung eines Kerns unter $\beta^+$-Zerfall (links) und $\beta^-$-Zerfall (rechts) in der Nuklid-Ebene \label{fig:5.8}}
\end{figure}
\item \tb{$e^{\pm}$-Energiespektrum}\\
Beobachtet: $e^{\pm}$ haben kontinuierliches Energiespektrum
\begin{align*}
0 \leq E_\mr{kin}^e \leq \lno E_\mr{kin}^e\rabs_\mr{max} \approx Q
\end{align*}
\begin{figure}[!ht]
\centering
\includegraphics[width=.5\textwidth]{imgs/ep5-fig-5-9.pdf}
\caption{Energiespektrumsgrafik der $e$-Energie\label{fig:5.9}}
\end{figure}
$\leadsto$ 1930: Verletzung von $E$- und $p$-Erhaltung oder neues Teilchen!
\begin{compactitem}
\item[$\Ra$] Pauli postuliert Neutrino
\item[$\Ra$] Nachgewiesen 1956
\end{compactitem}
\item \tb{$\beta$-Zerfall und Neutrinomasse}\\
Wenn $m_\nu > 0$ $\Ra$ $\lno E_\mr{kin}^e\rabs_\mr{max} \approx Q-m_\nu$\\
$\Ra$ Änderung des E-Spektrums bei $\lno E^e_\mr{kin}\rabs_\mr{max}$
\begin{figure}[!ht]
\centering
\includegraphics[width=.5\textwidth]{imgs/ep5-fig-5-10.pdf}
\caption{Korrektur (schwarz), wenn Neutrinos massenbehaftet sind\label{fig:5.10}}
\end{figure}

Messung erfordert:
\begin{compactitem}
\item Höchste Messgenauigkeit
\item kleiner $Q$-Wert
\end{compactitem}
Derzeit:
\begin{align}
\boxed{m_\nu < 1.1\,\mr{eV} \text{ mit } 90\,\% C.L.}
\end{align}
aus Tritium-Zerfall
\begin{align*}
^3_1\mr H \underset{Q=18.6\,\mr{keV}}{\ra}\, ^3_2 \mr{He} + e^- + \bar{\nu}_e
\end{align*}
(zukünftig: KATRIN in Karlsruhe: $\sim 0.2\,\mr{eV}$)
\end{itemize}

\section{Doppelbeta-Zerfall}
Falls mehrere stabile Isobare ($\ra$ gg-Kerne):\\
Zerfall durch \glqq 2 simultane Betazerfälle\grqq{} .
\begin{figure}[!ht]
\centering
\includegraphics[width=.5\textwidth]{imgs/ep5-fig-5-11.pdf}
\caption{Doppelbeta-Zerfall aufgetragen mit der Masse über die Ordnungszahl\label{fig:5.11}}
\end{figure}
\begin{align}
 ^A_Z X_N \ra ^A_{Z\pm 2} Y_{N\pm 2} + 2e^\pm + 2 \bar{\nu}_e (\text{oder } + 2\nu_e)
\end{align}
\begin{itemize}
\item[$\lt$] Prozess höherer Ordnung (extrem unterdrückt, aber möglich)\\
\begin{figure}[!ht]
    \centering
    \begin{tikzpicture}
        \begin{feynman}
            \vertex (a1) {$u$};
            \vertex[right=6cm of a1] (a2) {$u$};
            \vertex[right=3cm of a1] (a3);
            \vertex[below=2em of a1] (b1) {$d$};
            \vertex[below=2em of a2] (b2) {$d$};
            \vertex[right=3cm of a1] (b3);
            \vertex[below=2em of b1] (d1) {$u$};
            \vertex[right=3cm of d1] (d2);
            \vertex[below=2em of b2] (d3) {$d$};
            \vertex[below=2em of d3] (c1) {$\nu_e$};
            \vertex[below=2em of c1] (c3) {$e^+$};
            \vertex at ($(c1)!0.5!(c3) - (2cm, 0)$) (c2);
            
            \vertex[below=10em of d1] (f1) {$u$};
            \vertex[right=6cm of f1] (f2) {$d$};
            \vertex[right=3cm of f1] (f3);
            \vertex[below=2em of f1] (g1) {$d$};
            \vertex[below=2em of f2] (g2) {$d$};
            \vertex[right=3cm of f1] (g3);
            \vertex[below=2em of g1] (h1) {$u$};
            \vertex[right=3cm of h1] (h2);
            \vertex[below=2em of g2] (h3) {$u$};
            \vertex[above=2em of f2] (i1) {$\nu_e$};
            \vertex[above=2em of i1] (i3) {$e^+$};
            \vertex at ($(i1)!0.5!(i3) - (2cm, 0)$) (i2);

            \diagram* {
            (b1) -- [fermion] (b2),
            (d1) -- [fermion] (d2) -- [fermion] (d3),
            (c3) -- [fermion] (c2) -- [fermion] (c1),
            (d2) -- [scalar, edge label'=$W^+$] (c2),
            (a1) -- [fermion] (a2),
            (g1) -- [fermion] (g2),
            (h1) -- [fermion] (h3),
            (i3) -- [fermion] (i2) -- [fermion] (i1),
            (f3) -- [scalar, edge label=$W^+$] (i2),
            (f1) -- [fermion] (f3) -- [fermion] (f2),
            };
            \draw [decoration={brace}, decorate] (d1.south west) -- (a1.north west)
            node [pos=0.5, left] {$p$};
            \draw [decoration={brace}, decorate] (a2.north east) -- (d3.south east)
            node [pos=0.5, right] {$n$};
            \draw [decoration={brace}, decorate] (h1.south west) -- (f1.north west)
            node [pos=0.5, left] {$p$};
            \draw [decoration={brace}, decorate] (f2.north east) -- (h3.south east)
            node [pos=0.5, right] {$n$};
        \end{feynman}
    \end{tikzpicture}
    \caption{Simultaner $\beta$-Zerfall zweier Protonen \label{fig:5.12}}
\end{figure}
\item[$\lt$] \tb{Beispiel:}
\begin{align}
\begin{split}
^{48}_{20} \mr{Ca} \overset{2\beta^-}{\longrightarrow} ^{48}_{22}\mr{Ti} \qquad \tau = 4\times 10^{19}\,\mr{a}\\
^{76}_{32} \mr{Ge} \overset{2\beta^-}{\longrightarrow} ^{76}_{34}\mr{Se} \qquad \tau = 1.5\times 10^{21}\,\mr{a}
\end{split}
\end{align}
\item \tb{Neutrinoloser Doppelbeta-Zerfall}\\
Hypotetisch, aber höchst interessant
\begin{figure}[!ht]
\centering
    \begin{tikzpicture}
        \begin{feynman}
            \vertex (a1) {$u$};
            \vertex[right=6cm of a1] (a2) {$u$};
            \vertex[right=3cm of a1] (a3);
            \vertex[below=2em of a1] (b1) {$d$};
            \vertex[below=2em of a2] (b2) {$d$};
            \vertex[right=3cm of a1] (b3);
            \vertex[below=2em of b1] (d1) {$u$};
            \vertex[right=3cm of d1] (d2);
            \vertex[below=2em of b2] (d3) {$d$};
            \vertex[below=3em of d3] (c1) {$e^+$};
            \vertex[left=2cm of c1] (c2);
            \vertex[below=8em of d1] (f1) {$u$};
            \vertex[right=6cm of f1] (f2) {$d$};
            \vertex[right=3cm of f1] (f3);
            \vertex[below=2em of f1] (g1) {$d$};
            \vertex[below=2em of f2] (g2) {$d$};
            \vertex[right=3cm of f1] (g3);
            \vertex[below=2em of g1] (h1) {$u$};
            \vertex[right=3cm of h1] (h2);
            \vertex[below=2em of g2] (h3) {$u$};
            \vertex[above=3em of f2] (i1) {$e^+$};
            \vertex[left=2cm of i1] (i2);
            \node[above = .5em of i2, crossed dot] (x1);

            \diagram* {
            (b1) -- [fermion] (b2),
            (d1) -- [fermion] (d2) -- [fermion] (d3),
            (c1) -- [fermion] (c2),
            (d2) -- [scalar, edge label'=$W^+$] (c2),
            (a1) -- [fermion] (a2),
            (g1) -- [fermion] (g2),
            (h1) -- [fermion] (h3),
            (i1) -- [fermion] (i2),
            (f3) -- [scalar, edge label=$W^+$] (i2),
            (f1) -- [fermion] (f3) -- [fermion] (f2),
            (c2) -- (x1),
            (i2) -- (x1),
            };
            \draw [decoration={brace}, decorate] (d1.south west) -- (a1.north west)
            node [pos=0.5, left] {$p$};
            \draw [decoration={brace}, decorate] (a2.north east) -- (d3.south east)
            node [pos=0.5, right] {$n$};
            \draw [decoration={brace}, decorate] (h1.south west) -- (f1.north west)
            node [pos=0.5, left] {$p$};
            \draw [decoration={brace}, decorate] (f2.north east) -- (h3.south east)
            node [pos=0.5, right] {$n$};
        \end{feynman}
    \end{tikzpicture}
\caption{Feynman-Diagramm für neutrinolosen Doppelbeta-Zerfall \label{fig:5.13}}
\end{figure}
\begin{compactitem}
\item[$\lt$] Nur möglich, wenn $\nu = \bar{\nu}$ \glqq Majorana-Neutrinos\grqq{}
\item[$\ra$] Im Standardmodell verboten
\item[$\ra$] Intensive experimentelle Suche, bisher \tb{nicht} gefunden
\begin{compactitem}
\item[$\lt$] $\tau\left(0\nu 2\beta\right) > 10^{25}\,$a
\end{compactitem}
\end{compactitem}
\end{itemize}

\section{Gamma-Strahlung und innere Konversion}
Angeregte Kerne entstehen
\begin{compactitem}
\item als Zerfalls- oder Spaltprodukte
\item durch Beschuss mit $e,\ \gamma, \ p, \ n, \ N,\ \dots$
\end{compactitem}
Zerfall durch $\gamma$-Emission:
\begin{align}
\boxed{\underbrace{^A_Z X^\star_N}_{J_i^P} \ra \underbrace{^A_Z X_N}_{J_f^P} + \underbrace{\gamma}_{J_\gamma^P}} \qquad \mc{O}_\gamma(\mr{MeV})
\end{align}
$J_\gamma^P$ entspricht Drehimpuls $\vec{L}_\gamma$\\
Drehimpuls und Parität erhalten (elm. Prozess) $\Ra$
\begin{align}
\begin{split}
\vec{J}_i = \vec{J}_f + \vec{L}_\gamma \\
P_i = P_f \cdot P_\gamma
\end{split}
\end{align}
Multipolentwicklung: Es gibt elektrische (E) und magnetische (M) Multipolstrahlung
\begin{table}[!ht]
\centering
\begin{tabular}{|c|c|c|c|}
\hline
$L_\gamma$ & E & M & Name\\
\hline
1 & E1 & M1 & Dipol \\
2 & E2 & M2 & Quadrupol\\
3 & E3 & M3 & Oktupol\\
\dots & \dots & \dots & \dots\\
\hline
 & $P_\gamma = (-1)^{L_\gamma}$ & $P_\gamma = (-1)^{L_\gamma +1}$ & \\
 \hline
\end{tabular}
\end{table}

\tb{Auswahlregeln:} 
\begin{align}
\begin{split}
\labs J_i - J_f \rabs \leq L_\gamma \leq J_i + J_f\\
P_\gamma = (-1)^{L_\gamma} \cdot \llb \begin{matrix}
+1 \ (E) \\ -1\ (M) \end{matrix}\rno\qquad \overset{!}{=} \frac{P_i}{P_f}
\end{split}
\end{align}
\tb{Übergangswahrscheinlichkeit:}\\
\glqq klassisches Argument\grqq{}:
\begin{align*}
\labs \vec{L}_\gamma \rabs = L_\gamma \hbar = x \labs \vec{P}_\gamma \rabs = x \frac{E_\gamma}{c}\\
\Ra x = L_\gamma \frac{\hbar c}{E_\gamma} = L_\gamma \frac{197\,\mr{keV\,fm}}{1\,\mr{MeV}}= 200 L_\gamma \cdot \mr{fm} \gg R_k
\end{align*}
$\Ra$ größere $L_\gamma$ stark unterdrückt! QM-Rechnung:
\begin{align}
\boxed{ \lambda \sim E_\gamma (E_\gamma R_k)^{2L_\gamma}\cdot \llb \begin{matrix}
1 \ (E) \\ 0.1 \ (M)\end{matrix} \rno}
\end{align}
$E_\gamma = 1\,\mr{MeV}, \ A = 100$ $\Ra$ $(E_\gamma R_k) \approx 0.03$ $\Ra$ $(E_\gamma R_k)^2 \approx 10^{-3}$
\begin{itemize}
\item[$\Ra$] Niedrigstes erlaubtes $L_\gamma$ dominiert (oder $L_\gamma +1$, wenn $L_\gamma$ M-Übergang, vgl. Tab.\ref{tab:5.1})
\begin{table}[!ht]
\centering
\begin{tabular}{c|c|c|c|c|c}
$\Delta J = \labs J_i - J_f \rabs$ & 0 & 1 & 2 & 3 & \dots \\
\hline
$\frac{P_i}{P_f}$ = +1 & M1, E2 & M1, E2 & E2 & M3, E4 \\
$\frac{P_i}{P_f}$ = -1 & E1 & E1 & M2, E3 & E3 \\
\hline
\end{tabular}
\caption{Verschiedene Multipolstrahlungen verschiedener Werte für $\Delta J$ und $P_\gamma$ \label{tab:5.1}}
\end{table}

\item[$\lt$] $L_\gamma$ nachweisbar über Winkelcharakteristik der Strahlung $Y^l_m\ \dots$
\begin{figure}[!ht]
\centering
\includegraphics[width=.5\textwidth]{imgs/ep5-fig-5-14.pdf}
\caption{Häufig: Kaskaden über Zwischenzustände\label{fig:5.14}}
\end{figure}\\
\tb{Typische Lebensdauern}\\
$10^{-15} \dots 10^{-9}$\,s\\
Aber: Übergänge mit $\labs \Delta J \rabs \geq 4$ können längerlebig sein, z.B. 
\begin{align*}
^{110}Ag^\star (6^+) \overset{M4}{\ra}\ ^{110}Ag(2^-) + \gamma
\end{align*}
mit $T_{\nicefrac{1}{2}} = 235$\,d (\glqq Isomere\grqq)\\
\tb{Bemerkung:} $6^+$ entspricht $J=6$ und positiver Parität
\item \tb{Innere Konversion}\\
Alternativ zu $\gamma$-Emission
\begin{align}
\boxed{\left( ^A_Z X^\star_N \right)^0 \ra \left( ^A_Z X_N \right)^+ + e^-}
\end{align}
nachfolgend Röntgenstrahlung oder Auger-Elektron
\end{itemize}


\section{Kernspaltung}
Aufbrechen eines Kerns in zwei Tochterkerne und Neutronen:
\begin{align*}
\boxed{^A_ZX_N \ra ^{A_1}_{Z_1}{Y_1}_{N_1} + ^{A_2}_{Z_2}{Y_2}_{N_2} + kn}\\
A = A_1 + A_2 + k; \ Z = Z_1 + Z_2; \ N = N_1 + N_2 +k
\end{align*}
$Q$-Wert $>$ 0 für schwere Kerne, aber: Coulomb-Barriere\\
Kernspaltung kann (in Bezugnahme zum Tröpfchenmodell) beschrieben werden als Verformung:

\begin{figure}[!ht]
	\centering
	\includegraphics[width=.6\textwidth]{imgs/ep5-fig-5-15.pdf}
	\caption{Verzerrung des \glqq Nukleonentropfens\grqq{} bis hin zur Spaltung in zwei neue \glqq Tropfen\grqq \label{fig:5.15}}
	\end{figure}

\begin{figure}[!ht]
	\centering
	\includegraphics[width=.5\textwidth]{imgs/ep5-fig-5-16.pdf}
	\caption{Schematische Skizze zur Aktivierungsenergie $E_\mr{act}$, die für eine Kernspaltug nötig ist \label{fig:5.16}}
\end{figure}

\tb{3 Fälle:}
\begin{itemize}
\item[(i)] $E_\mr{act} < 0$ $\Ra$ Kern nicht gebunden
\item[(ii)] $E_\mr{act} > 0$, aber klein\\\
$\Ra$ Spontane Spaltung durch Tunnelprozess (ähnlich $\alpha$-Zerfall)
\item[(iii)] $E_\mr{act} > 0$ und groß\\
$\Ra$ Spaltung erfordert Energiezufuhr
\end{itemize}
\tb{Im Tröpfchenmodell:}
$E_s$, $E_c$ hängen von $\epsi$ ab.
\begin{align*}
\lno \begin{matrix}
E_s (\epsi) = a_s A^{\nicefrac{2}{3}} \left( 1+ \frac{2}{5}\epsi^2 + \dots \right) \\ E_c(\epsi) = \underbrace{a_c Z^2 A^{-\nicefrac{1}{3}}}_\text{Weizsäcker-Formel} \left( 1 - \frac{1}{5} \epsi^2 + \dots \right)
\end{matrix} \rrb \text{ohne Rechnung (Taylorentwicklung)}\\
\Ra \Delta E (\epsi) =\lno M(A,Z) \rabs_\epsi - \lno M(A,Z) \rabs_{\epsi=0} = \frac{\epsi^2}{5} \left( 2 a_s A^{\nicefrac{2}{3}} - a_c Z^2 A^{-\nicefrac{1}{3}}\right)
\end{align*}
\begin{itemize}
\item[$\Ra$] \tb{Fall (i)}, wenn $\Delta E (\epsi) < 0$
\begin{align}
\lt \ \boxed{\frac{Z^2}{A} > \frac{2 a_s}{a_c} \approx 48}
\end{align}
\begin{itemize}
\item[$\lt$] erfüllt für $Z \geq 114$, $A \gtrsim 290$
\item[$\lt$] Solche Kerne sind nicht gebunden $\ra$ nicht existent
\end{itemize}
\tb{Fall (ii): Spontane Spaltung}
\begin{align}
\begin{matrix}
 & \nearrow & ^{234}_{90}\mr{Th} & \left(\sim 100 \%\right) \qquad \text{Zerfall}\\
^{238}_{92} \mr U & & & \\
 & \searrow & \dots & \left(5\cdot 10^{-5}\%\right) \qquad \text{Spaltung}
\end{matrix}
\end{align}
\tb{Fall (iii): Induzierte Spaltung}\\
Insbesondere durch Beschuss mit Neutronen.\\
Beispiel:
\begin{align*}
^{238}_{92} \mr U + n \ra ^{239}_{92} \mr U + E_\mr{exc}
\end{align*}
Energiebilanz:
\begin{align}
\boxed{E_\mr{exc} = \underbrace{M(A,Z) + M_n - M(A+1, Z)}_{\Delta M (A,Z)} + T_n}\\
 T_n \ = \ \text{kin. Energie des Neutrons}\nonumber
\end{align}
Spaltung, falls $E_\mr{exc} > E_\mr{act}$\\
$\Ra$ 2 Fälle
\begin{itemize}
\item[(i)] $\Delta M (A,Z) > E_\mr{act}$: $T_n$ wird \glqq nicht benötigt\grqq{}, Spaltung durch langsame Neutronen\\
z.B. $^{235}_{92}\mr U$, $^{233}_{90}\mr{Th}$, $^{239}_{94} \mr{Pu}$
\begin{compactitem}
\item[$\lt$] Kettenreaktion
\item[$\lt$] gu-Kern + n $\ra$ gg-Kern
\end{compactitem}
\item[(ii)] $\Delta M(A,Z) < E_\mr{act}$ $\Ra$ $T_n \geq E_\mr{act} - \Delta M(A,Z)$\\
$\Ra$ Spaltung durch schnelle Neutronen. Z.B.:
\begin{align*}
^{238}_{92}\mr U \ \left( T_n \gtrsim 0.6\,\mr{MeV}\right)
\end{align*}
\end{itemize}
\tb{WQ für n-Einfang} (vgl. Kap.\ref{chap:3})
\begin{align}
\boxed{\sigma_n \sim \frac{1}{j_n} \sim \frac{1}{v_n} \sim \frac{1}{\sqrt{T_n}}}
\end{align}
$\Ra$ n-Einfang \tb{viel} effizienter für thermische als für schnelle Neutronen
\end{itemize}
\section{Das Prinzip von Kernreaktoren}
Kettenreaktion durch n-Einfang
\begin{align*}
^{235}_{92}U + n \ra X + Y + \lla 2.5 n\rra
\end{align*}
\tb{Multiplikationsfaktor}
\begin{align*}
\boxed{k = \frac{\# \text{sekundäre Spaltungen}}{\text{Primärspaltung}}}
\end{align*}
$k<1$: keine Kettenreaktionen (subkritisch)\\
$k=1$: stabile Kettenreaktion (kritisch) $\ra$ KKW\\
$k>1$: exponentielles Wachstum (superkritisch) $\ra$ Kernwaffen\\
Ein Reator muss also auf $k=1$ geregelt werden. Dies erfordert:
\begin{compactitem}
\item[$\ra$] Angereichertes $^{235}_{92}U$
\item[$\ra$] Abbremsen der n
\begin{compactitem}
\item[$\ra$] Stöße mit leichten Kernen (z.B.: $H_2O$, $D_2O$, $C$), genannt \glqq Moderator\grqq{}
\item[$\ra$] Ziel: $\frac{\Pa k}{\Pa T} < 0$
\begin{compactitem}
\item[$\lt$] Stabilisierend, z.B. für $H_2O$, $D_2O$
\end{compactitem}
\end{compactitem}
\item[$\ra$] Regelung der n-Dichte
\begin{compactitem}
\item[$\ra$] n-Absorber (z.b. $Cd$), genannt \glqq Kontrollstäbe\grqq{}
\end{compactitem}
\end{compactitem}
\begin{figure}[!ht]
\centering
\includegraphics[width=.6\textwidth]{imgs/ep5-fig-5-17.pdf}
\caption{Schematischer Aufbau eines Reaktorbeckens \label{fig:5.17}}
\end{figure}
Nach Abschaltung:
\begin{compactitem}
\item keine Kettenreaktion
\item Aber \glqq Nachzerfallswärme\grqq{} durch radioaktive Zerfälle und Restneutronen
\item[$\ra$] unkontrollierter Anstieg der Temperatur, wenn das Kühlsystem versagt (\glqq Kernschmelze\grqq{}, z.B. Tschernobyl und Fukushima)
\end{compactitem}

%\chapter{Die Struktur des Nukleons}
Nukleon ist ausgedehntes Objekt.\\
Radius $\sim \mc{O}(\mr{fm})$. Untersuchung in Lepton-Nukleon-Streuung:
\begin{align*}
ep, \ \mu p , \ \nu(\bar{\nu}) p, eA, \mu A, \nu(\bar{\nu}) A
\end{align*}
Im Folgenden wird die Elektron-Proton-Streuung betrachtet.

Energieskala: $\Delta p \, \Delta x > \hbar \ \Ra \ E \gtrsim 1$\,GeV mit $\Delta p = \labs \vec{q}\rabs$ und $\Delta x = R_N < 1$\,fm\\
Typisch für $E$ sind $1 \dots 400$\,GeV

Bei HERA (DESY, Hamburg) ep-Kollisionen. Entspr. 50\,TeV Strahlenenergie in Fixed-Target-Experiment

\section{eN-Streuung -- Kinematik und WQ}
hier: QM (statt QFT nötig $\Ra$ Näherungen), Schrödingergleichung (obwohl relativistische Kinematik)

\begin{itemize}
\item \tb{Kinematik}\\
\begin{figure}[!ht]\centering
\includegraphics[width=.6\textwidth]{imgs/ep5-fig-6-1.pdf}
\caption{Elektron-Nukleon-Streuung \label{fig:6.1}}
\end{figure}
\begin{align}
\begin{split}
p_\mu = (E,p,0,0)\\
p_\mu^\prime = \lb E^\prime, p^\prime \cos \vartheta,  p^\prime \sin \vartheta,0 \rb \\
P_\mu = (M,0,0,0)
\end{split}
\end{align}
$E$- und $p$-Erhaltung:
\begin{align}
\begin{split}
\lb  p_\mu + P_\mu\rb ^2 = \lb p_\mu^\prime + P_\mu^\prime\rb ^2\\
\Ra \underbrace{p_\mu^2 + P_\mu^2 }_{m^2+M^2} + 2p_\mu P^\mu = \underbrace{{p_\mu^\prime}^2 + {P_\mu^\prime}^2}_{m^2+M^2} + 2p_\mu^\prime P^{\mu\prime}\
\Ra p_\mu P^\mu = p_\mu^\prime \lb  p^\mu + P^\mu - p^{\mu\prime}\rb \\
\Ra EM = \underbrace{p_\mu^\prime p^\mu}_{EE^\prime - pp^\prime \cos \vartheta} + E^\prime M - m^2
\end{split}
\end{align}
\begin{align}
\begin{split}
\boxed{E^\prime = E - \underbrace{\frac{EE^\prime - pp^\prime \cos \vartheta - m^2}{M}}_{>0, \text{ klein wenn } M\gg E}}
\end{split}
\end{align}
\begin{enumerate}
\item Wenn $M\gg E$ $\Ra$ $E^\prime = E$ (Rutherford, eN bei $E\sim 10$\,MeV)
\item Wenn $m\ll E$ $\Ra$ $E=p,\ E^\prime = p^\prime$
\begin{align}
\begin{split}
\Ra \boxed{E^\prime = \frac{E}{1 + \frac{E}{M}\lb  1- \cos \vartheta\rb }}
\end{split}
\end{align}
(eN-Streuung bei $E \gtrsim 10$\,MeV)
\end{enumerate}
\item \tb{Potentialstreuung}\\
Fall (1): $M \gg E$, \glqq kein\grqq{} E-Übertrag auf N
\begin{itemize}
\item[$\Ra$] Wie Streuung in ortsfestem Potential $V\lb \vec{r}\rb $ \\$\ra$ Hamiltonian $\Ham\lb \vec{r}\rb $
\begin{figure}[!ht]
\centering
\includegraphics[width=.6\textwidth]{imgs/ep5-fig-6-2.pdf}
\caption{Veranschaulichung des Hamiltonian als WW-Operator\label{fig:6.2}}
\end{figure}
\item[$\Ra$] Nutze: \begin{compactitem}
\item QM-Störungsrechnung
\item Fermis Goldene Regel
\item Zustandsdichte im Endzustand
\end{compactitem}
\begin{align}
\Ra \boxed{ \Pa \sigma = \frac{{E^\prime}^2}{\lb  2\pi\rb ^2} \labs \int \Pa^3 r e^{i \vec{q}\vec{r}} \Ham\lb \vec{r}\rb  \rabs^2 \Pa \Omega }
\end{align}
\begin{compactitem}
\item[mit] $\vec{q} = \vec{p}-\vec{p}^\prime$
\item[] $\Ham\lb \vec{r}\rb  = (-e) V_A\lb \vec{r}\rb $
 \item[] $V_A$ als Coulombpotential des Targets
\end{compactitem}
$V_A\lb \vec{r}\rb $ wird von Ladungsdichte $\rho_A\lb \vec{r}\rb $ erzeugt\\
$\rho_A\lb \vec{r}\rb  = e f_A\lb \vec{r}\rb $ mit $\int \Pa^3 r f_A\lb \vec{r}\rb  = 1$\\
Somit die Poissongleichung (Elektrostatik)
\begin{align}
\boxed{\Lap V_A\lb \vec{r}\rb  = - \frac{\rho_A\lb \vec{r}\rb }{\epso} = -\frac{e}{\epso}f_A\lb \vec{r}\rb }
\end{align}
\begin{align*}
& \Ra \int \Pa^3 r e^{i \vec{q}\vec{r}} \Ham \lb \vec{r}\rb \\
&\qquad \labs\ \Lap e^{i \vec{q}\vec{r}} = - \labs \vec{q}\rabs^2 e^{i \vec{q}\vec{r}} \rno\\
& =  - \int \Pa^3 r \frac{1}{\labs\vec{q}\rabs^2} \lb  \Lap e^{i \vec{q}\vec{r}}\rb  \Ham\lb \vec{r}\rb \\
&\qquad \labs\ \text{2-fache partielle Integration} \rno \\
& = -\frac{1}{\labs \vec{q}\rabs ^2} \int \Pa^3 r e^{i\vec{q}\vec{r}} \underbrace{\lb  \Lap \Ham \lb \vec{r}\rb \rb }_{-e \frac{-\rho_A \lb \vec{r}\rb }{\epso}}\\
& = - \underbrace{\frac{e^2}{\labs \vec{q}\rabs^2 \epso}}_{=\frac{4\pi\alpha}{\labs \vec{q}\rabs^2}} \underbrace{\int \Pa^3 r e^{i\vec{q}\vec{r}} f_A\lb \vec{r}\rb }_\text{= Formfaktor $F\lb \vec{q}\rb $}
\end{align*}
Für punktförmiges Target:
\begin{align}
f_A \lb  \vec{r}\rb  = \delta \lb  \vec{r}\rb  \ \Ra \ F\lb \vec{q}\rb  \equiv 1,
\end{align}
also
\begin{align}
\boxed{ \begin{matrix}
\frac{\Pa \sigma}{\Pa \Omega} & = \frac{{E^\prime}^2}{\lb 2\pi\rb ^2} \labs \int \Pa^3 r e^{i\vec{q}\vec{r}}\Ham\lb \vec{r}\rb  \rabs^2\\
& = \frac{4 \alpha^2}{\labs \vec{q}\rabs^4}{E^\prime}^2 = \lno \frac{\Pa \sigma}{\Pa \Omega}\rabs_\text{Rutherford}
\end{matrix} }
\end{align}
Hierbei eigentlich: Faktor $Z_e^2 Z_p^2$

Erinnerung:
\begin{align}
\labs \vec{q}\rabs ^2 = \labs \vec{p} - \vec{p}^\prime \rabs^2 \overset{p= p^\prime}{=} 2 p^2 - 2p^2 \cos \vartheta = 4 p^2 \sin^2 \frac{\vartheta}{2}\nonumber \\
p^2 \llb \begin{matrix}
E_{kin} \cdot 2m & \text{nicht-relativistisch (Rutherford)}\\ EE^\prime & \text{relativistisch } (E \approx E^\prime = p)
\end{matrix} \rno \nonumber\\
\Ra \boxed{ \lno \frac{\Pa \sigma^\mathrm{eN}}{\Pa \Omega}\rabs _\mathrm{Rf} = \frac{Z_e^2 Z_p^2 \alpha^2 }{4 E^2 \sin^4 \frac{\vartheta}{2}} }
\end{align}
\end{itemize}
\item \tb{Korrekturen}
\begin{itemize}
\item[$\ra$] Ausgedehntes Target:
\begin{align}
\frac{\Pa \sigma}{\Pa \Omega} = \lno \frac{\Pa \sigma}{\Pa \Omega}\rabs_\mathrm{Rf} \cdot \labs F\lb  \vec{q} \rb  \rabs ^2
\end{align}
\item[$\ra$] Rückstoß-Korrektur (Energieübertrag)
\begin{align}
\frac{\Pa \sigma}{\Pa \Omega} = \lno \frac{\Pa \sigma}{\Pa \Omega}\rabs_\text{Rf} \frac{E^\prime}{E}
\end{align}
\item[$\ra$] Magnetisches Moment des $e^-$ (ohne magn. Moment des $N$)
\begin{align}
\frac{\Pa \sigma}{\Pa \Omega} = \lno \frac{\Pa \sigma}{\Pa \Omega}\rabs_\text{Rf} \cdot \lb  1 - \beta_e^2 \sin^2 \frac{\vartheta}{2}\rb 
\end{align}
für $\beta_e \ \ra \ 1$ (relativistisches $e^-$):
\begin{align*}
1-\beta_e^2 \sin^2\frac{\vartheta}{2} \ \ra \ \cos^2 \frac{\vartheta}{2}
\end{align*}
\item[$\Ra$] Rückstreuung ($\vartheta = 180^\circ$) unterdrückt\\
Grund: Helizitätserhaltung (folgt aus relativistischer QM für $\beta_e \ra 1$)
\begin{align*}
\text{Helizität} =  \frac{\vec \sigma \vec{p}}{\labs \vec{\sigma}\rabs \labs \vec{p} \rabs}
\end{align*}
\begin{align}
\boxed{ \frac{\Pa \sigma}{\Pa \Omega} = \frac{Z_e^2 Z_p^2 \alpha^2 }{4 E^2 \sin^4 \frac{\vartheta}{2}} \frac{E^\prime}{E} \cdot \cos^2 \frac{\vartheta}{2} = \lno \frac{\Pa \sigma}{\Pa \Omega}\rabs_\mathrm{Mott} }\\
\text{zusätzlich } \labs F\lb \vec{q}\rb  \rabs^2
\end{align}
\item[$\ra$] Magnetisches Moment des Targets (punktförmiges Dirac-Teilchen)
\begin{align}
\boxed{ \frac{\Pa \sigma}{\Pa \Omega} \ra \lno \frac{\Pa \sigma}{\Pa \Omega}\rabs_\mathrm{Mott}  \lb  1 + 2 \tau \tan^2 \frac{\vartheta}{2}\rb  }
\end{align}
\begin{compactitem}
\item[mit] $\tau = \frac{Q^2}{4 M^2} = \frac{- q_\mu q^\mu}{4 M^2}$
\end{compactitem}
Jetzt: Spin-Flip von Target möglich, $\lno \nicefrac{\Pa \sigma}{\Pa \Omega} \rabs_{\vartheta = 180^\circ} > 0$
\end{itemize}
\end{itemize}

\section{Elastische eN-Streuung}
N ist \tb{nicht} punktförmig $\Ra$ Formfaktor\\
\tb{2} Formfaktoren
\begin{compactitem}
\item $G_E\lb q^2\rb  \ra $ Verteilung der Ladung
\item $G_M\lb q^2\rb  \ra $ Verteilung der magnetischen Momente
\end{compactitem}
\begin{itemize}
\item[$\Ra$] WQ ist (nach Rosenbluth-Formel)
\begin{align}
\boxed{ \frac{\Pa \sigma}{\Pa \Omega} = \lno \frac{\Pa \sigma}{\Pa \Omega} \rabs_\mr{Mott} \lsb  \frac{G_E^2 +\tau \cdot G_M^2}{1 + \tau} + 2 \tau G_M^2 \tan^2 \frac{\vartheta}{2} \rsb  }
\end{align}
\item[$\ra$] Messung von $\frac{\Pa \sigma}{\Pa \Omega}\lb q^2\rb $ ergibt $G_{E,M}$
\begin{align}
\boxed{
G_{E,M}^p \lb Q^2\rb  = \frac{G_{E,M}^p \lb 0\rb }{\lb  1+ \frac{Q^2}{0.71\,\mr{GeV}^2}\rb ^2}
}
\end{align}

Auch:
\begin{align}
G_M^n = \frac{G_M^n \lb  0\rb }{\lb 1+ \frac{Q^2}{0.71\,\mr{GeV}^2}\rb ^2}
\end{align}
\item[$\Ra$] Ladungsverteilung
\begin{figure}[!ht]
\centering
\includegraphics[width=.5\textwidth]{imgs/ep5-fig-6-3.pdf}
\caption{Ladungsverteilung eines Protons\label{fig:6.3}}
\end{figure}

Es gilt hierbei: $R_p \approx 0.86$\,fm
\item[$\ra$] $G_{M,E} \lb  Q^2 = 0\rb $ ist Ladung bzw. magnetisches Moment des Nukleons
\begin{align*}
\begin{matrix}
G_E^p(0) = 1 & \text{Ladung 1}e\\
G_E^n(0) = 0 & \text{Ladung 0}e\\
G_M^p(0) = 2.79 & \text{magnetisches Moment }\mu_p= 2.79\frac{e\hbar}{2M_p}\\
G_M^n(0) = -1.91 & \mu_n = - 1.91\frac{e\hbar}{2M_n}
\end{matrix}
\end{align*}
\end{itemize}

\section{Anregungszustände der Nukleonen}
Inelastische Streuung:
\begin{figure}[!ht]
\centering
\includegraphics[width=.6\textwidth]{imgs/ep5-fig-6-4.pdf}
\caption{Skizze zur inelastischen Streuung eines $e^-$ an einem Teilchen A\label{fig:6.4}}
\end{figure}
Aus \autoref{fig:6.4}
\begin{align*}
q_\mu = p_\mu - p_\mu^\prime\\
p_{X,\mu} = p_{A,\mu} + q_\mu
\end{align*}
Es gilt also somit:
\begin{align}
\begin{split}
\underbrace{p_X^2}_{W^2} = \underbrace{p_A^2}_{M_A^2} + \underbrace{q_\mu q^\mu}_{-Q^2} + \underbrace{2 p_{A,\mu}q^\mu}_{2M_A\lb E-E^\prime\rb =2 M\nu}\\
\Ra W^2 = -Q^2 + M^2 + 2 \nu M = Q^2 \lsb  \frac{2 \nu M}{Q^2} -1 \rsb  +M^2\\
\frac{2\nu M}{Q^2} = \frac{1}{x}
\end{split}
\end{align}
$x$ = \glqq Bjorken-x\grqq
\begin{itemize}
\item[$\lt$] elastisch: $W^2 = M^2 \Ra x=1$
\item[$\lt$] inelastisch: $W^2 > M^2 \Ra x<1$
\item[$\lt$] $\boxed{2}$ Variablen erforderlich, um inelastische Kinematik festzulegen\\
$\lt$ $\lb x, Q^2\rb $, $\lb x, W\rb $, $\lb E^\prime, \vartheta\rb $
\item[$\Ra$] Messung von $\dfrac{\sigma}{W}$

\begin{figure}[!ht]
\centering
\includegraphics[width=.6\textwidth]{imgs/ep5-fig-6-5.pdf}
\caption{Resonanzpeaks der $eN$-Streuung\label{fig:6.5}}
\end{figure}

\item[$\ra$] Peaks = \glqq Resonanzen\grqq{}
\begin{compactitem}
\item[$\Ra$] kurzlebige Zustände
\item[$\Ra$] Nukleonanregungen
\item[$\Ra$] Prominent: $\Delta^+\lb 1232\rb $
\begin{align}
\begin{split}
M\lb \Delta^+\lb 1232\rb \rb  = 1.232\,\mr{GeV}\\
\Gamma_{\Delta^+} \approx 120\,\mr{MeV}
\end{split}
\end{align}
$\ra$ Lebensdauer des $\Delta^+\lb 1232\rb $:
\begin{align}
\tau = \frac{1}{\Gamma}= \frac{197\,\mr{\nicefrac{MeV}{fm}}}{120\,\mr{MeV}}\frac{1}{c} \approx 0.5\cdot 10^{-23} \,\mr s
\end{align}
$\tau$ ist typischer Wert für starke Wechselwirkung
\end{compactitem}
\item[$\lt$] Genauere Untersuchung:
\begin{align*}
\ket{\Delta^+\lb 1232\rb } = \ket{uud} \ \ \ \text{wie $p$}\\
J^p\lb \Delta^+\lb 1232\rb \rb  = \lb  \frac{3}{2}\rb ^+; \ \ \ J^p \lb p\rb  = \lb  \frac{1}{2}\rb  ^+
\end{align*}
\item[$\lt$] $\Delta$-Zerfall
\begin{figure}[!ht]
\centering
    \begin{tikzpicture}
        \begin{feynman}
            \vertex (a1) {$u$};
            \vertex[right=2cm of a1] (a2);
            \vertex[right=3cm of a1] (a3);
            \vertex[right=5cm of a1] (a4) {$u$};
            \vertex[below=2em of a1] (b1) {$u$};
            \vertex[right=2.5cm of b1] (b2);
            \vertex[right=5cm of b1] (b3) {$u$};
            \vertex[below=2em of b1] (c1) {$d$};
            \vertex[right=5cm of c1] (c2) {$d$};
            \vertex[above=2.667em of a4] (d1) {$\bar u$};
            \vertex[above=4em of a4] (e1) {$u$};
            
            \vertex[right=8cm of a1] (f1) {$u$};
            \vertex[right=2cm of f1] (f2);
            \vertex[right=3cm of f1] (f3);
            \vertex[right=5cm of f1] (f4) {$d$};
            \vertex[below=2em of f1] (g1) {$u$};
            \vertex[right=2.5cm of g1] (g2);
            \vertex[right=5cm of g1] (g3) {$u$};
            \vertex[below=2em of g1] (h1) {$d$};
            \vertex[right=5cm of h1] (h2) {$d$};
            \vertex[above=2.667em of f4] (i1) {$\bar d$};
            \vertex[above=4em of f4] (j1) {$u$};

            \diagram* {
            (a1) -- (a2) -- (e1),
            (a2) -- [gluon] (a3) -- (d1),
            (b1) -- (b3),
            (a3) -- (a4),
            (c1) -- (c2),
            
            (f1) -- (f2) -- (j1),
            (f2) -- [gluon] (f3) -- (i1),
            (g1) -- (g3),
            (f3) -- (f4),
            (h1) -- (h2),
            };
            \draw [decoration={brace}, decorate] (c1.south west) -- (a1.north west)
            node [pos=0.5, left] {$\Delta^+$};
            \draw [decoration={brace}, decorate] (a4.north east) -- (c2.south east)
            node [pos=0.5, right] {$p$};
            \draw [decoration={brace}, decorate] (e1.north east) -- (d1.south east)
            node [pos=0.5, right] {$\pi^0$};

            \draw [decoration={brace}, decorate] (h1.south west) -- (f1.north west)
            node [pos=0.5, left] {$\Delta^+$};
            \draw [decoration={brace}, decorate] (f4.north east) -- (h2.south east)
            node [pos=0.5, right] {$n$};
            \draw [decoration={brace}, decorate] (j1.north east) -- (i1.south east)
            node [pos=0.5, right] {$\pi^+$};
        \end{feynman}
    \end{tikzpicture}
\caption{Zerfallsgrafiken für $\Delta^+$ zu p bzw. n}
\end{figure}
\end{itemize}

\section{Tiefinelastische Streuung}
Frage: Was passiert bei höheren Energien?\\
Erwartung: Bessere räumliche Auflösung $\Ra$ Bestandteile des Nukleons
\begin{itemize}
\item Beobachtung 1\\
Bei $W \gtrsim 2.5$\,GeV: Kontinuum von hadronischen Endzuständen
\begin{itemize}
\item[$\lt$] keine Resonanzen
\item[$\lt$] unterschiedliche Hadronenkombinationen
\item[$\lt$] Differentieller WQ hängt von zwei Variablen ab, z.B.
\begin{align*}
\underbrace{Q^2 = - q_\mu q^\mu; \ \ \ x = \frac{Q^2}{2M\nu} = \frac{- q_\mu q^\mu}{2 P_{A,\mu}q^\mu}}_\text{Lorentz-Invariant}
\end{align*}
\item[$\lt$] Darstellung des WQ:
\begin{align}
\dfrac{^2 \sigma}{\Omega \Pa E^\prime} = \lno \dfrac{\sigma}{\Omega}\rabs_\mr{Mott} \lb  W_2\lb x, Q^2\rb  + 2 W_1 \lb x, Q^2\rb  \tan^2 \frac{\vartheta}{2}\rb  
\end{align}
\begin{compactitem}
\item[mit] $W_2\lb x, Q^2\rb $: Ladungs-WW
\item[] $2 W_1 \lb x, Q^2\rb  \tan^2 \frac{\vartheta}{2}$: magnetische WW
\item[] $W_{1,2}$: \tb{\glqq Strukturfunktion\grqq}
\end{compactitem}
Achtung:
\begin{align*}
\underbrace{G_{E,M}\lb Q^2\rb }_\text{elastisch} \ \longrightarrow \ \underbrace{W_{1,2} \lb  x, Q^2\rb }_\text{tiefinelastisch}
\end{align*}
\item[$\lt$] Dimensionslose Strukturfunktionen:
\begin{align}
F_2\lb x, Q^2\rb  = \nu W_2 \lb x, Q^2\rb \\
F_1 \lb x, Q^2\rb  = MW_1 \lb  x, Q^2\rb 
\end{align}
\end{itemize}
\item Beobachtung 2\\
Erste Messungen 1968 ff.
\begin{itemize}
\item[$\Ra$] $F_2\lb  x, Q^2\rb $ hängt (nur sehr schwach) von $Q^2$ ab (\glqq Skaleninvarianz\grqq)
\item[$\Ra$] Interpretation: $Q^2$-Abhängigkeit aus Fourier-Transformation um $\rho (\vec r)$
\begin{itemize}
\item[$\Ra$] $\rho \lb \vec{r}\rb  = \delta \lb  \vec{r}\rb $
\item[$\Ra$] Streuung an \tb{punktförmigen} Teilchen (Konstituenten des Nukleons)
\item[$\Ra$] \tb{Quarks!} Bewegen sich \glqq quasi-frei\grqq{} im Nukleon
\end{itemize}
\item[$\ra$] Kinematik
\begin{figure}[!ht]
\centering
\includegraphics[width=.6\textwidth]{imgs/ep5-fig-6-7.pdf}
\caption{Strukturskizze eines Protons\label{fig:6.7}}
\end{figure}

\ni Aus \autoref{fig:6.7}: alle Massen und Transversalimpulse sind vernachlässigbar\\
Quark-Impuls:
\begin{align}
p_i = \xi_i p_p= \xi_i \lb \gamma M , \gamma M, 0, 0\rb 
\end{align}

\item[$\lt$] ep-Streuung $\hat{=}$ elastische eq-Streuung

\begin{figure}[!ht]
\centering
    \begin{tikzpicture}
        \begin{feynman}
            \vertex (a1);
            \vertex[right = 3cm of a1] (a2);
            \vertex[above right = 3cm of a2] (b1);
            \vertex[below right = 3cm of a2] (c1);
            
            \diagram*{
            (a1) -- [fermion, edge label = $q_i(p_i)$] (a2),
            (a2) -- [boson, edge label = $\gamma(q)$] (b1),
            (a2) -- [fermion, edge label = $q_f(p_f)$] (c1),
            };
        \end{feynman}
    \end{tikzpicture}
\caption{Feynmandiagramm zur Quarkstreeung\label{fig:6.8}}
\end{figure}

\begin{align}
p_f^2 = \lb  p_i + q\rb ^2 =p_i^2 - Q^2 + 2p_{i_\mu}q^\mu \nonumber \\
2p_{i,\mu} q^\mu = 2 \xi_i p_{p,\mu} q^\mu \overset{!}{=}Q^2 \nonumber\\
\boxed{ \xi_i = \frac{Q^2}{2 p_{i,\mu}q^\mu} = \frac{Q^2}{2 M \nu} = x }
\end{align}
\item[$\Ra$] $x=$ Anteil an p-Impuls, den das getroffene Quark trägt!
\end{itemize}
\item Beobachtung 3\\
$F_{1,2} \lb x, Q^2\rb  \approx F_{1,2}\lb x\rb $ ergeben \glqq Quark-Dynamik im Nukleon\grqq{}\\
Vereinfacht:
\begin{align}
\boxed{ \sigma \lb  e, p \rb   = \sum_{q \in p } \sigma \lb eq_i \rb   q_i\lb x\rb }
\end{align}
\begin{compactitem}
\item[mit] $q_i(x)$: \glqq Parton-Verteilungsfunktion\grqq{} (PDF)
\item[] $\sigma \lb ep\rb , \ \sigma\lb eq_i\rb $: eigentlicher differentieller WQ
\end{compactitem}
\tb{\glqq Parton\grqq}: Quark oder Gluon\\
für $q_i$ mit $x$ im Parton:
\begin{align}
\boxed{q_i\lb x\rb  = \dfrac{\lb WS\rb }{x}}
\end{align}
\end{itemize}

\vorlesung{12. Januar 2018}

\begin{figure}[!ht]
\centering
\includegraphics[width=.9\textwidth]{imgs/ep5-fig-6-9.pdf}
\caption{Erwartete Verteilungen für $q_i(x)$\label{fig:6.9}}
\end{figure}
\begin{itemize}
\newpage
\item \tb{Beobachtung 4}\\
$F_1$-Term im WQ kommt vom magnetischen Moment des Targets
\begin{itemize}
\item[$\Ra$] Erwartung:
\begin{align}
F_1\lb  x, Q^2\rb  = \begin{cases}0, \text{ wenn Spin}(q) = 0\\ 
\frac{F_2\lb x, Q^2\rb }{2x}, \text{ wenn Spin}(q) = \frac{1}{2} \end{cases}
\end{align}
\item[$\Ra$] Messung (Callen-Gross-Relation):
\begin{align}
\boxed{ 2xF_1\lb x, Q^2\rb  = F_2 \lb  x, Q^2\rb  }
\end{align}
$\ra$ Quarks sind Fermionen
\end{itemize}
\item \tb{Beobachtung 5}
\begin{align}
\int_0^1 \sum_{q\text{ in } p} x q_i(x) \Pa x \approx 0.5 \neq 1
\end{align}
1 erwartet, wenn nur Quarks q im Proton p wären.\\
$\Ra$ weitere Partonen = Gluonen in p
\newpage

\item \tb{Beobachtung 6}\\
Skaleninvarianz ist verletzt $\Ra$ Effekt der starken WW

\begin{figure}[!ht]
\centering
\includegraphics[width=.6\textwidth]{imgs/ep5-fig-6-10.pdf}
\caption{Verschiebung der Partonen-Verteilung mit zunehmenden $Q^2$ zu kleinen $x$\label{fig:6.10}}
\end{figure}
\end{itemize}

%\chapter{Die starke Wechselwirkung}
\begin{itemize}
\item[$\ra$]  bindet Nukleonen im Atomkern
\item[$\ra$] bindet Quarks in Hadronen
$\Ra$ Starke WW ist auf Quark/Gluon-Niveau theoretisch exakt beschreibbar
\end{itemize}
\section{Grundlegende Struktur}
Theorie der starken WW: Quantenchromodynamik\\
\glqq chromo\grqq{} wegen der Farbe (von Farbladungen)
\begin{figure}[!ht]
\centering
\includegraphics[width=.6\textwidth]{imgs/ep5-fig-7-1.pdf}
\caption{Feynmandiagramme zum Vergleich der starken und elm. WW \label{fig:7.1}}
\end{figure}
\begin{itemize}
\item[$\lt$] Gluonen sind masselos und koppeln an Farbladungen
\item[$\lt$] Es gibt 3 Farbladungen: r = rot, g = grün und b= blau
\item Theoretischer Hintergrund: Lokale Eichinvarianz\\
$\Ra$ fordere Invarianz unter:
\begin{align}
\text{elm. } & \Psi \ra \underbrace{e^{i\Phi\lb x_\mu\rb  } \Psi}_{\in Li(1)} & \Ra \ 1 \text{ \glqq Eichfeld\grqq{} = Photon}\\
\text{stark } & \Psi \begin{pmatrix}
r\\g\\b\end{pmatrix} \ra \underbrace{C\lb x_\mu\rb }_{\in SU(3)} \Psi \begin{pmatrix}r\\g\\b\end{pmatrix} & \Ra \ 8 \text{ Eichfelder = Gluonen, farbgeladen}
\end{align}
\item \tb{Farbladungen}
\begin{itemize}
\item[$\lt$] alle Leptonen, $\gamma$, $Z$, $W^{\pm}$ = 0
\item[$\lt$] alle Quarks: r,g,b
\item[$\lt$] alle Antiquarks: $\mr{\bar{r}, \bar{g}, \bar{b}}$
\item[$\lt$] alle Gluonen: $\mr{r\bar{g}, b \bar{r}, \dots}$
\newpage
\begin{figure}[!ht]
\centering
\includegraphics[width=.75\textwidth]{imgs/ep5-fig-7-2.pdf}
\caption{Kombination von Farbe und Antifarbe\label{fig:7.2}}
\end{figure}

\end{itemize}
\item Hadronen sind farbneutrale gebundene Systeme
\begin{itemize}
\item[$\lt$] Mesonen $\ket{q\bar{q}}$
\item[$\lt$] Baryonen $\ket{qqq}$
\item[$\lt$] \glqq Pentaquarks\grqq{} $\ket{qqqq\bar{q}}$
\item[$\lt$] \glqq Glueballs\grqq{} $\ket{gg}, \ \ket{ggg}$
\item Achtung Pentaquarks und Glueballs konnten noch nicht nachgewiesen werden!
\begin{figure}[!ht]
\centering
\includegraphics[width=.75\textwidth]{imgs/ep5-fig-7-3.pdf}
\caption{Skizze zur Farbladungserhaltung an jedem Vertex\label{fig:7.3}}
\end{figure}
\end{itemize}
\item Gluonen tragen Farbe

\begin{figure}[!ht]
\centering
\includegraphics[width=.4\textwidth]{imgs/ep5-fig-7-4Gluonen.pdf}
\caption{Selbstwechselwirkung von Gluonen\label{fig:7.glue}}
\end{figure}

\item Farbgeladenen frei Teilchen (Quarks, Gluonen, Diquarks, \dots ) gibt es nicht\\
$\lt$ Confinement (\glqq Eingeschlossenheit\grqq)
\end{itemize}
\newpage
\section{Evidenzen für Gluonen und Farbe}
\begin{itemize}
\item \tb{Gluonen}
\begin{enumerate}
\item \tb{Tiefinelastische eN-Streuung}
\begin{align}
\int_0^1 \sum{q \text{ in } p } x q_i (x) \Pa x \approx 0.5
\end{align}
\begin{itemize}
\item[$\Ra$] Weitere \glqq Teilchensorte\grqq{} im Nukleon, mit der $e^-$ nicht wechselwirkt
\item[$\Ra$] elektrisch neutral
\end{itemize}
\item \tb{Jets}\\
\glqq Teilchenbündel\grqq\ in Richtung von Quarks oder Gluonen im Endzustand

\begin{minipage}[c]{.39\textwidth}
\captionsetup{type=figure}
\includegraphics[width=\textwidth]{imgs/ep5-fig-7-4.pdf}
\captionof{figure}{Bsp.1: Feynmandiagramm für $e^+e^- \rightarrow q\bar{q}$ \label{fig:7.4}}
\end{minipage}
\begin{minipage}[c]{.39\textwidth}
\captionsetup{type=figure}
\includegraphics[width=\textwidth]{imgs/ep5-fig-7-5.pdf}
\captionof{figure}{Bsp.2: Feynmandiagramm für $e^+e^- \rightarrow q\bar{q}g$ \label{fig:7.5}}
\end{minipage}

\begin{itemize}
\item[$\rightarrow$] Entstehung von Jets oder auch von Baryonen
\item[$\rightarrow$] nur möglich, wenn einer der Jets von einem Gluon stammt
\item[$\rightarrow$] entdeckt 1979 am DESY
\end{itemize}

\begin{minipage}[c]{.39\textwidth}
\captionsetup{type=figure}
\includegraphics[width=\textwidth]{imgs/ep5-fig-7-6.pdf}
\captionof{figure}{Jet-Entstehung: für $e^+e^-$ entstehen 3 Jets \label{fig:7.6}}
\end{minipage}
\begin{minipage}[c]{.39\textwidth}
\captionsetup{type=figure}
\includegraphics[width=\textwidth]{imgs/ep5-fig-7-7.pdf}
\captionof{figure}{Baryonenenstehung\\ aus Gluon-Jet \label{fig:7.7}}
\end{minipage}

\end{enumerate}
\end{itemize}
\newpage
\begin{itemize}
\item \tb{Farbe}
\begin{itemize}
\item[1.] \tb{Das $\Delta^{++}$-Baryon ($\Delta^{++}(1232))$}\\
Produktion z.B. $\nu_\mu p\rightarrow\mu^-\Delta^{++}$
\begin{align}
\ket{\Delta^{++}}=\ket{u\uparrow u\uparrow u\uparrow};\ J^p=(\frac{3}{2})^+
\end{align}
$\rightarrow$ Bahndrehimpuls $L=0$ (leichterer Q=2e-Baryon)\\
$\rightarrow$ Symmetrische Wellenfunktion (Flavour, Spin, Ort)
$\rightarrow$ symm. unter Tausch von 2 identischen Fermionen $\thor$ Pauli-Prinzip\\
$\rightarrow$ Lösung: $\ket{u_r\uparrow u_b\uparrow u_s\uparrow}$ mit antisymmetrischen Farb-Wellenfkt. %bin nicht sicher ob u_r stimmt konnte es nicht genau lesen!
\item[2.] \tb{WQ für $e^+ e^- \rightarrow q\bar{q} \rightarrow$ Hadronen} (z.B.: Abb.\ref{fig:7.8})
\begin{figure}[!ht]
\centering
\includegraphics[width=.45\textwidth]{imgs/ep5-fig-7-8.pdf}
\caption{Feynmandiagramm zu $e^+e^-\rightarrow f\bar{f}$ \label{fig:7.8}}
\end{figure}

Diese Reaktion $e^+e^-\rightarrow \underbrace{f\bar{f}}_{\text{Fermion/Antifermion-Paar}}$ ist möglich wenn
\begin{align*}
s=(E_{cms})^2 > (2m_f)^2
\end{align*}
$Z^0$-Austausch dominant bei $\sqrt{s}=M_z$, aber klein bei $\sqrt{s}\approx \mathcal{O}(10\,keV)$\\
Wirkungsquerschnitt:
\begin{align}
\frac{\Pa \sigma}{\Pa \Omega}\sim \frac{Z^2_f\alpha^2}{s} (1+cos^2\theta)
\end{align}
Betrachte
\begin{align}
\Pa\frac{\sigma^{Had}}{\Pa\omega}=\sum_{\sqrt{s}>2m_{a_i}\footnotemark}\frac{\Pa \sigma^{a_i}}{\Pa \Omega}
\end{align}
Summenindex hat Schwelle bei $\sim$ 1\,GeV für strange-, $\sim$ 3\,GeV für charme-, $\sim$ 10\,GeV für bottom- und $\sim$ 350\,GeV für top-Quarks.\\
Verwende $e^+e^-\rightarrow\mu^+\mu^-$ als \grqq Eichreaktion\grqq:
\begin{align}
R_\mu=\frac{\nicefrac{d\sigma^{Had}}{d\Omega}}{\nicefrac{d\sigma^\mu}{d\Omega}}=\sum_{\sqrt{s}>2M_{a_i}}Z^2_{a_i}
\end{align}
Summieren über alle Farbzustände (r,g,b)$\rightarrow$ Faktor 3\\
Erhalten also:
\begin{align}\label{eq:7.8}
R_\mu = 3 \bigg[\underbrace{\underbrace{\underbrace{\underbrace{\stackrel{u}{\left(\frac{2}{3}\right)^2}+\stackrel{d}{\left(\frac{1}{3}\right)^2}}_{\sqrt{s}\geq 150\,MeV}+\stackrel{s}{\left(\frac{1}{3}\right)^2}}_{\sqrt{s}\geq 1\,GeV}+\stackrel{c}{\left(\frac{2}{3}\right)^2}}_{\sqrt{s}\geq 3\,GeV}+\stackrel{b}{\left(\frac{1}{3}\right)^2}}_{\sqrt{s}\geq 10\,GeV}+\stackrel{t}{\left(\frac{2}{3}\right)^2}\bigg]
\end{align}
\begin{figure}[!ht]
\centering
\includegraphics[width=.5\textwidth]{imgs/ep5-fig-7-9.pdf}
\caption{Graphische Veranschaulichung der Geichung \ref{eq:7.8} \label{fig:7.9}}
\end{figure}
\begin{itemize}
\item[$\rightarrow$] Faktor 3 durch Messung bestätigt
\item[$\rightarrow$] Es gibt Farbladungen und davor genannt
\end{itemize}
\end{itemize}
\end{itemize}
\section{Die starke Kopplung: Confinement und Asymptotic-Freedom}
Die Kopplungsstärke von Wechselwirkungen hängt von $Q^2$ ab. Grund sind höhere Ordnungen der Störungsreihe
\begin{itemize}
\item Für elektromagnetische WW (QED):
\begin{figure}[!ht]
\centering
\includegraphics[width=.65\textwidth]{imgs/ep5-fig-7-10.pdf}
\caption{Höhere Ordnungen elektromagnetischer Wechselwirkung \label{fig:7.10}}
\end{figure}

$\Rightarrow$ Abschirmung:

\begin{figure}[!ht]
\centering
\includegraphics[width=.5\textwidth]{imgs/ep5-fig-7-11.pdf}
\caption{Dipolwolke um das rechte Elektron schirmt das linke, einfliegende Elektron ab \label{fig:7.11}}
\end{figure}

$\rightarrow$ Ladung umso größer, je \grqq näher man kommt \grqq\, d.h. desto größer $Q^2$
\begin{align}
\Rightarrow\alpha=\alpha(Q^2)=
\begin{cases}
\nicefrac{1}{137} & (Q^2=0)\\
\nicefrac{1}{128} & (Q^2=M^2_z)
\end{cases}
\end{align}

\item Für starke WW (QCD):

\begin{figure}[!ht]
\centering
\includegraphics[width=.5\textwidth]{imgs/ep5-fig-7-12.pdf}
\caption{Höhere Ordnungen der starken Wechselwirkung \label{fig:7.12}}
\end{figure}
\begin{itemize}
\item[$\rightarrow$] dominant!
\item[$\Rightarrow$] $\alpha_s(Q^2)$ wird mit $Q^2$ kleiner!
\begin{align}
\Rightarrow \text{\fbox{$\alpha_s(Q^2)=\frac{12\pi}{(33-2N_f\footnotemark)ln\frac{Q^2}{\Lambda^2\footnotemark}}+h.O.$}} 
\end{align}
\begin{compactitem}
\item[mit] $N_f$: Zahl der Quarkflavours
\item[] $\Lambda$: in QED-Skala $\sim$ 250\,MeV
\end{compactitem}
\begin{align}
\alpha_s(Q^2)=
\begin{cases}
0,1185 & (Q^2=M^2_z)\\
0,3 & (Q^2=m^2_\tau)\\
>1 & (Q^2\leq 0,1\, GeV^2)
\end{cases}
\end{align}
\item[$\Rightarrow$] \textbf{Grenzfläche:}
\begin{itemize}
\item $Q^2\rightarrow 0:\ \alpha_s\rightarrow\infty$
\begin{itemize}
\item[$\rightarrow$] keine Störungsrechnung
\item[$\rightarrow$] keine freien Farbladungen
\item[$\rightarrow$] \grqq Confinement \grqq\
\end{itemize}
\item $Q^2\rightarrow \infty:\ \alpha_s \rightarrow 0$
\begin{itemize}
\item[$\rightarrow$] Störungsrechnung (QCD)
\item[$\rightarrow$] \grqq quasi-freie\grqq{} Quarks in tiefinelastischer Streuung
\item[$\rightarrow$] \grqq Asymptotic-Freedom\grqq\
\end{itemize}
\end{itemize}
\end{itemize}
\end{itemize}

\vorlesung{19. Januar 2018}

\section{Aufbau der Hadronen}
Quark-Gluon-Kopplung unabhängig von Quark-Flavour (u,d,s,c,t,b)
\begin{itemize}
\item[$\ra$] Hadron-Zustände nur von Bindungszustand ($L,J,n$) abhängig?
\item[$\ra$] Nein:            Effekt
\item[$\ra$] Quark-Massen     groß
\item[$\ra$] Quark-Ladungen   klein
\item \tb{Isospin-Symmetrie}\\
u- und d-Quarks haben etwa gleiche Massen (einige MeV)
\begin{itemize}
\item[$\Ra$]u$\leftrightarrow$d lässt die Hadronzustände \textbf{näherungsweise} invariant
\item[$\Ra$] Formel: starke WW \textbf{näherungsweise} invariant unter:
\begin{align*}
\left(\begin{array}{c}u\\d\end{array}\right)\rightarrow SU(2)\left(\begin{array}{c}u\\d\end{array}\right)\\
\Rightarrow \text{\grqq Isospin\grqq}\\
\ket{u}=I, I=\frac{1}{2}, \ket{_3=\frac{1}{2}}\\
\ket{d}=I, I=\frac{1}{2}, \ket{_3=-\frac{1}{2}}
\end{align*}
\item[$\Rightarrow$] Es gibt \grqq Sätze\grqq\ von Hadronen die aus $\stackrel{\leftrightarrow}{u},\stackrel{\leftrightarrow}{d}$ bestehen und mit $SU(2)$-Isospin in  sich selbst übergehen $\rightarrow$ \grqq Isospin-Multipletts\grqq
\begin{itemize}
\item gleiches $I$
\item $\sim$ gleiche Massen
\item Ladung = $f(I_3)$
\end{itemize}
\end{itemize}
$\Rightarrow$ Beispiel: Mesonen

\begin{figure}[!ht]
\centering
\includegraphics[width=.75\textwidth]{imgs/ep5-fig-7-13.pdf}
\caption{Isospinzusammensetzung bei Mesonen \label{fig:7.13}}
\end{figure}

Entstehendes Meson aus
\begin{align*}
L=0;\ S=0\rightarrow J^{PC}=0^{-+}:\ \pi^+,\pi^-,\pi^0\\
L=0;\ S=1\rightarrow J^{PC}=1^{--}:\ \rho^+,\rho^-,\rho^0\\
(M_\rho\approx 770\,MeV)
\end{align*}
\item \tb{Erweiterung auf 3 Quark-Flavours:}\\
$(u,d,s)$, $SU(2)\rightarrow SU(3),d.h.:$
\begin{figure}[!ht]
\centering
\includegraphics[width=.75\textwidth]{imgs/ep5-fig-7-14.pdf}
\caption{Erweiterung der Isopspinsymmetrie um den dritten Quarkflavour s \label{fig:7.14}}
\end{figure}
\begin{align}\begin{split}
A&=\frac{1}{\sqrt{2}}(u\bar{u}-d\bar{d})=\ket{\pi^0}\\
B&=\frac{1}{\sqrt{6}}(u\bar{u}+d\bar{d}-2s\bar{s})\approx\ket{\eta}\\
C&=\frac{1}{\sqrt{3}}(u\bar{u}+d\bar{d}+s\bar{s})\approx\ket{\eta^\prime}
\end{split}\end{align}
Wobei $B$ und $C$ mischen!

Wegen $m_s\gg m_{u,d}$: Massen \textbf{nicht} ähnlich $(M_\pi=139\,\mr{MeV},\ M_K=485\,\mr{MeV})$
\begin{itemize}
\item[$\Rightarrow$] SU(3) keine \grqq gute\grqq Symmetrie, aber bietet gute Ordnungsschema, auch für Baryonen
\end{itemize}
\item \tb{Erweiterung auf c-Quarks}
\begin{itemize}
\item[$\rightarrow$]$SU(3)\rightarrow SU(4)$
\item[$\rightarrow$] 3D-Multipletts
\item[$\rightarrow$] wichtig als Ordnungsschema
\end{itemize}
\end{itemize}

\section{Erzeugung und Zerfall von Hadronen}
\begin{itemize}
\item \tb{Erzeugung}\\
In hadronischen Reaktionen
\begin{itemize}
\item[$\ra$] \grqq Umwandlung von Quark-Linien\grqq\, z.B. $\pi^- p\rightarrow\pi^0 n$

\begin{figure}[!ht]
\centering
\includegraphics[width=.5\textwidth]{imgs/ep5-fig-7-15.pdf}
\caption{Feynmandiagramm zur Umwnadlungsreaktion \label{fig:7.15}}
\end{figure}

\item[$\ra$] Erzeugung/Vernichtung von Quark-Antiquark-Teilchen\\
z.B. $\pi^- p \rightarrow \Delta^0(1232)\rightarrow\pi^0 n$

\begin{figure}[!ht]
\centering
\includegraphics[width=.5\textwidth]{imgs/ep5-fig-7-16.pdf}
\caption{Feynmandiagramm zur $\pi^0n$ Erzeugung\label{fig:7.16}}
\end{figure}

z.B. $\pi^- p \rightarrow \Lambda k^0$
\begin{figure}[!ht]
\centering
\includegraphics[width=.5\textwidth]{imgs/ep5-fig-7-17.pdf}
\caption{Assoziierte Produktion \label{fig:7.17}}
\end{figure}

\item[$\ra$] Inelastisch, z.B. in Jet-Bildung (\grqq Fragmentation\grqq, \grqq Hadronisation\grqq)
\end{itemize}
In $e^+ e^-$-Streuung:
\begin{itemize}
\item[$\ra$] Hadronische $q\bar{q}$-Resonanzen, z.B. $e^+ e^- \rightarrow \nicefrac{J}{\Psi}$

\begin{figure}[!ht]
\centering
\includegraphics[width=.5\textwidth]{imgs/ep5-fig-7-18.pdf}
\caption{Inelastische Erzeugung von $\nicefrac J\Psi$ \label{fig:7.18}}
\end{figure}

Auch für
\begin{align*}
u\bar{u},d\bar{d}: \rho^0\\
s\bar{s}: \phi\\
b\bar{b}: \gamma 
\end{align*}
\item[$\ra$] $e^+ e^- \rightarrow qq(g)$, Jet-Bildung
\end{itemize}
%hier ist iwas zerschossenes es sieht auf jeden fall kackeeee aus
\item \tb{Zerfälle}\\
Alle 3 WW kommen vor
\begin{itemize}
\item[1.] starke WW, z.B. $\rho^0\rightarrow\pi^+\pi^-$
\begin{figure}[!ht]
\centering
\includegraphics[width=.5\textwidth]{imgs/ep5-fig-7-19.pdf}
\caption{Zerfall von $\rho$ durch die starke WW \label{fig:7.19}}
\end{figure}

wenn erlaubt: dominant\\
Typisch : $\Gamma\approx 100...150\,\mr{MeV}$, $\tau\approx 10^{-23}\,\mr{s}$
\item[2.] Elektromagnetische WW (wenn Zerfall über starke WW nicht möglich ist)\\
z.B. $\pi^0\rightarrow\gamma\gamma$ (vgl. Abb. \ref{fig:7.20})

\begin{figure}[!ht]
\centering
\includegraphics[width=.5\textwidth]{imgs/ep5-fig-7-20.pdf}
\caption{Zerfall von $\pi^0$ durch elektromagnetische WW \label{fig:7.20}}
\end{figure}

$\pi^0$ leichteres Hadron, starker Zerfall nicht möglich
$\tau\approx 10^{-16}\,\mr{s}$\\
auch: $\sum^0\rightarrow \Lambda\gamma$
\item[3.] schwache WW (wenn starke und elm. verboten sind)\\
Insbesondere für leichte/leichteste Hadronen mit $S\neq 0,C\neq 0, B\neq 0$\\
z.B. $K^+\rightarrow \mu^+\nu_\mu$ (vgl. Abb.\ref{fig:7.21})

\begin{figure}[!ht]
\centering
\includegraphics[width=.5\textwidth]{imgs/ep5-fig-7-21.pdf}
\caption{Zerfall von $K^+$ durch Austausch eines $W^+$-Bosons \label{fig:7.21}}
\end{figure}

genauso: $\Pi^+(s\leftrightarrow d)$
\item[$\rightarrow$] Spezieller Fall: Quarkonium\\
Zustände aus einem Quark und seinem Antiquark, Mesonen ohne elektrische Ladung oder Flavour.

Für die Lebensdauer gilt: $\tau \sim 0.7\cdot 10^{-20}\,s$ 

\begin{figure}[!ht]
\centering
\includegraphics[width=.5\textwidth]{imgs/ep5-fig-7-22.pdf}
\caption{Feynmandiagramm zum Quarkonium \label{fig:7.22}}
\end{figure}
\end{itemize}
\end{itemize}

%\vorlesung{25. Januar 2018}

\chapter{Die Schwache Wechselwirkung}
Bisher haben wir gesehen:
\begin{itemize}
\item $\beta$-Zerfälle von Kernen
\item Hadron-Zerfälle
\item Neutrinos
\end{itemize}
\tb{Jetzt: }Systematisch
\section{Grundlegende Struktur}
\begin{itemize}
\item WW durch Austausch von Bosonen (wie QED, QCD)
\item Austauschteilchen sind massiv und geladen:\\
$W^\pm$, $M_W = 80.4\,$GeV\\
$Z^0$, $M_Z = 91.2$\,GeV
\item koppelt an \tb{alle} Fermionen (einzige WW von Neutrinos)
\item Immer: $W$- oder $Z$-Propagator
\begin{align}
\boxed{\mc{M} \sim \frac{1}{Q^2 + M_{W,Z}^2} \, \Ra \, \sigma,\ \Gamma \overset{Q^2 \ll M_{W,Z}^2}{\sim} \frac{1}{M_{W,Z}^4}}
\end{align}
Nenner $M_{W,Z}^4$ ist der Grund dafür, warum die schwache WW schwach ist.
\item Eichstruktur (angedeutet)
\begin{align}
\underbrace{SU(2)\begin{pmatrix}\cdot\\ \cdot\end{pmatrix}_L}_{\substack{\text{Eichbosonen:}\\W^-, W^0, W^+}} \hspace*{1cm} \underbrace{U(1) (\cdot)_R}_{\substack{\text{Eichboson:}\\B}}
\end{align}
$SU(2)$ ist linkshändig, $U(1)$ dagegen rechtshändig.\\
Die Eichbosonen $W^0$ und $B$ können mischen. Es gilt dafür:
\begin{align}
\begin{pmatrix}
Z^0 \\ \gamma
\end{pmatrix} = \begin{pmatrix}
\cos \theta_W & - \sin \theta_W\\  \sin \theta_W & \cos \theta_W
\end{pmatrix} \begin{pmatrix}
W^0 \\ B
\end{pmatrix}
\end{align}
(\glqq elektroschwache Vereinigung\grqq{})
\begin{compactitem}
\item[mit] $\theta_W$: Weinberg-Winkel
\item[] $\sin^2 \theta_W \approx 0.222$
\item[] $\Ra$ $Z^0$ hat \glqq elektromagnetischen Anteil\grqq{}
\end{compactitem}
\item Higgs-Boson:\\
Notwendig, um massive Eichbosonen zu erklären, ohne Eichsymmetrie zu verletzen
\end{itemize}

\section{Der geladene Strom}

$W^\pm$-Austausch, koppelt an alle Fermionen:

\begin{figure}[!ht]
\centering
\includegraphics[width=.5\textwidth]{imgs/ep5-fig-8-1.pdf}
\caption{$W$-Kopplung an ein Lepton und das zugehörige Neutrino bzw. an ein Quark \label{fig:8.1}}
\end{figure}

\begin{itemize}
\item Kopplungsstärke
\begin{align}
g_W = \frac{e}{\sin \theta_W}
\end{align}
$\Ra$ \tb{stärker} als bei elm. WW!
\item \tb{Quark-Mischung}\\
Beobachtet z.B. bei Zerfall der leichtesten Mesonen/Baryonen mit $S\neq 0$

\begin{figure}[!ht]
\centering
\includegraphics[width=.5\textwidth]{imgs/ep5-fig-8-2.pdf}
\caption{Quark-Mischung am Beispiel $\Lambda^0 \ra n \pi^0$\label{fig:8.2}}
\end{figure}
\newpage
Erklärung:
\begin{framed}
\begin{center}
Quark-EZ der schwachen WW $\neq$ Quark-EZ der starken WW
\end{center}
\end{framed}
QM: Beide EZ-Vektoren durch unitäre Transformation verknüpft. Für d,s ergibt sich:
\begin{align}
\boxed{ \underbrace{\begin{pmatrix}
d^\prime \\ s^\prime 
\end{pmatrix}}_\text{schwach} = \underbrace{\begin{pmatrix}
\cos \theta_C & \sin \theta_C\\ -\sin\theta_C & \cos \theta_C \end{pmatrix}}_\text{Cabibbo-Matrix} \underbrace{\begin{pmatrix}
d \\ s
\end{pmatrix}}_\text{stark}
 }
\end{align}
Hierbei ist $\theta_C$ der sogenannte Cabibbo-Winkel.
\begin{align*}
\sin \theta_C \approx 0.22, \sin^2 \theta_C \approx 0.05
\end{align*}
\begin{itemize}
\item[$\Ra$] Geladener Strom hat Terme:
\begin{align}
\begin{pmatrix}
u \\ c
\end{pmatrix}^\top U_\mr{Cab} \begin{pmatrix}
d \\ s
\end{pmatrix}=\quad \begin{cases}\quad 
\lno \begin{matrix}
ud \cdot \cos \theta_C & +\\ cs \cdot \cos\theta_C & +
\end{matrix} \rrb \labs \mc{M}\rabs^2 \sim \cos^2 \theta_C \approx 0.95\\ \ \\
\quad \lno \begin{matrix}
us \cdot \sin \theta_C & -\\ cd \cdot \sin\theta_C & -
\end{matrix} \rrb \labs \mc{M}\rabs^2 \sim \sin^2 \theta_C \approx 0.05 \end{cases}
\end{align}
\item[$\lt$] $U_\mr{Cab}$ kann auch als Mischung von $(u,c)$ interpretiert werden
\item[$\Ra$] \glqq Nicht diagonale\grqq{} Terme sind mit Faktor 20 unterdrückt.
\begin{figure}[!ht]
\centering
\includegraphics[width=.5\textwidth]{imgs/ep5-fig-8-3.pdf}
\caption{Offidiagonale ($\sin \theta_C$ Zerfälle unterdrückt \label{fig:8.3}}
\end{figure}

$\Ra$ Verallgemeinerung auf 3 Quark-Familien
\begin{align}
\boxed{ \begin{pmatrix}
d^\prime \\ s^\prime \\ b^\prime 
\end{pmatrix} = U(3) \cdot \begin{pmatrix}
d \\ s\\ b
\end{pmatrix} }
\end{align}
$U(3)$ = unitäre Matrix, genannt Cabibbo-Kobayashi-Maskawa-Matrix (CKMM)
\begin{itemize}
\item[$\lt$] 3 reelle Parameter (\glqq Drehwinkel\grqq{})
\item[$\lt$] komplexe Phase
\item[$\lt$] Beträge der Matrixelemente
\begin{align}
\begin{pmatrix}
0.974 & 0.225 & 0.004 \\
0.225 & 0.973 & 0.041\\
0.009 & 0.040 & 0.999
\end{pmatrix} \approx \begin{pmatrix}
U_\mr{Cab} & \begin{matrix}
0.004 \\ 0.041
\end{matrix} \\
\begin{matrix}
0.009 & 0.040
\end{matrix} & 0.999
\end{pmatrix}
\end{align}
sehr stark diagonaldominant
\item[$\ra$]  Aus unterer Zeile
\begin{align*}
\Gamma \lb  t \ra d\rb : \Gamma \lb t \ra s\rb  : \Gamma\lb t \ra b\rb  = (0.009)^2 : (0.040)^2 : (0.999)^2
\end{align*}

\begin{figure}[!ht]
\centering
\includegraphics[width=.5\textwidth]{imgs/ep5-fig-8-4.pdf}
\caption{Feynmandiagramme zu WW bezüglich der unteren Zeile der CKMM \label{fig:8.4}}
\end{figure}
\end{itemize}
\end{itemize}
\end{itemize}
\section{Paritätsverletzung in der schwachen WW}
Erinnerung: Parität ist verletzt, wenn
\begin{align}\begin{split}
&\hat{P}^\dagger \Ham_{WW} \hat{P} \neq \Ham_{WW}\\
\Ra \mc{M}_P &= \braket{\Psi_f | \hat{P}^\dagger \Ham_{WW} \hat{P} | \Psi_i}\\
& = \braket{\hat{P}\Psi_f | \Ham_{WW} | \hat{P} \Psi_i}\\
& \neq \braket{\Psi_f | \Ham_{WW} | \Psi_i}
\end{split}\end{align}
raumgespiegelten Prozess $\neq$ Wahrscheinlichkeit für ungespiegelten Prozess.

Bis $\sim$1955: Fester \glqq Glaube\grqq{} an Paritätserhaltung.\\Dann: Experiment von C.S. Wu (1956)
\begin{align*}
\underset{\text{Spin 5}}{^{60}_{27} Co} \ra \underset{\text{Spin 4}}{^{60}_{28} Ni^\star} + e^- + \bar{\nu}_e
\end{align*}

\begin{figure}[!ht]
\centering
\includegraphics[width=.5\textwidth]{imgs/ep5-fig-8-5.pdf}
\caption{Experiment zur Paritätsverletzung bei schwacher WW\label{fig:8.5}}
\end{figure}
\newpage
Achtung: $\hat{P} \vec{B} = \vec{B}$, $\hat{P} \vec{J} = \vec{J}$, $\hat{P}\vec{p} = - \vec{p}$
\begin{itemize}
\item[$\Ra$] Paritätsverletzung!
\item[$\Ra$] Interpretation $\bar{\nu}$ ist rechtshändig, denn:

\begin{figure}[!ht]
\centering
\includegraphics[width=.5\textwidth]{imgs/ep5-fig-8-6.pdf}
\caption{$\bar{\nu}$ ist rechtshändig, da die rechten Prozess unterdrückt ist \label{fig:8.6}}
\end{figure}
\item[$\Ra$] Schwache WW koppelt an linkshändige $\nu$'s und an rechtshändige $\bar{\nu}$'s\\
(1957 bestätigt in Goldhader-Experiment)

\begin{figure}[!ht]
\centering
\includegraphics[width=.65\textwidth]{imgs/ep5-fig-8-7.pdf}
\caption{Skizze zum Goldhader-Experiment \label{fig:8.7}}
\end{figure}

\item[$\lt$] $\underbrace{\text{maximale}}_{\begin{pmatrix}
\nu_l\\l\end{pmatrix}_L, \ \begin{pmatrix}
q^\prime \\ q \end{pmatrix}_L}$ $\underbrace{\text{Paritätsverletzung!}}_{\substack{(l)_R, \ (q)_R\\\text{Koppeln nicht an $W$}}}$
\item[$\Ra$] Bisher: Helizität $h$
\begin{align}
h = \frac{\vec{\sigma}\vec{p}}{\labs \vec{\sigma}\rabs \labs \vec{p} \rabs}
\end{align}
In der Theorie der schwachen WW entscheiden \glqq Chiralität\grqq{} $\hat{C}_{L,R}$
\begin{align*}
\ket{\Psi} = \ket{\Psi_L} + \ket{\Psi_R} = \hat{C}_L \ket{\Psi} + \underbrace{\hat{C}_R}_{= (1-\hat{C}_L)}\ket{\Psi}\\
\ket{\bar{\Psi}} = \ket{\bar{\Psi}_R} + \ket{\bar{\Psi}_L} = \hat{C}_L \ket{\bar{\Psi}} + \hat{C}_R\ket{\bar{\Psi}}
\end{align*}
Hierbei koppel $\ket{\Psi_L}$ und $\ket{\bar{\Psi}_R}$ an $W^\pm$\\
Helizität:
\begin{align*}
\ket{\Psi_L} = \sqrt{\frac{1+\beta}{2}} \ket{h=-1} + \sqrt{\frac{1-\beta}{2}} \ket{k=+1}\\
\ket{\Psi_R} = \sqrt{\frac{1+\beta}{2}} \ket{h=+1} + \underbrace{\sqrt{\frac{1-\beta}{2}} \ket{k=-1}}_{\substack{\text{\glqq falsche Helizität\grqq{}}\\\text{unterdrückt bei } \beta \ra 1\\\text{(insbesondere für $\nu$'s)}}}
\end{align*}
\item[$\Ra$] \tb{Beispiel: $\pi^+$-Zerfall}

\begin{figure}[!ht]
\centering
\includegraphics[width=.5\textwidth]{imgs/ep5-fig-8-8.pdf}
\caption{Helizitätsverletzung im $\pi^+$-Zerfall \label{fig:8.8}}
\end{figure}

\begin{align*}
\frac{\Gamma \lb  \pi^+ \ra e^+ \nu_e\rb }{\Gamma \lb  \pi^+ \ra \mu^+ \nu_\mu\rb  }= 1.2\cdot 10^{-4}
\end{align*}
Warum? Phasenraumfaktor ist viel größer für $\pi^+ \ra e^+ \nu_e$, weil $m_e\ll m_\mu$

\begin{figure}[!ht]
\centering
\includegraphics[width=.5\textwidth]{imgs/ep5-fig-8-9.pdf}
\caption{falsche Helizität für $e$, $\mu$ \label{fig:8.9}}
\end{figure}
\newpage

\begin{itemize}
\item[$\lt$] $\nu$ muss $h=-1$ haben
\item[$\lt$] $e$, $\mu$ muss $h=-1$ haben (Drehimpulserhaltung)
\begin{itemize}
\item[$\Ra$] $e$, $\mu$ hat \glqq falsche\grqq{} Helizität
\item[$\Ra$] unterdrückt mit $\frac{1 - \beta}{2}$\\
Faktor:
\begin{align}
\frac{m_e^2}{\lb M_\pi + m_e\rb ^2} \ll \frac{\mu_\mu^2}{\lb M_\pi + m_\mu \rb ^2}
\end{align}

\begin{align*}
\Gamma \lb  \pi^+ \ra e^+ \nu_e\rb  \ll \Gamma \lb  \pi^+ \ra \mu^+ \nu_\mu\rb 
\end{align*}
\end{itemize}
\end{itemize}
\end{itemize}

\section{CP-Symmetrie und CP-Verletzung}
\begin{itemize}
\item \tb{Erinnerung:}\\
$\hat{C}$ = Ladungskonjugation
\begin{align*}
\hat{C} \ket{\Psi} = \pm \ket{\bar{\Psi}}\\
\hat{C}\ket{\pi^0} = + \ket{\pi^0}\\
\hat{C}\ket{\pi^+ \pi^-} = \lb -1\rb ^2 \ket{\pi^+ \pi^-}
\end{align*}
\item \tb{CP-Transformation}
\begin{align}\begin{split}
\hat{C}\hat{P} \ket{\pi^0} = - \ket{\pi^0} \hspace*{1cm} (J^\mr{PC} = 0^{-+})\\
\hat{C}\hat{C} \ket{\gamma} \hspace*{1cm} (J^\mr{PC} = 0^{--})\\
\hat{C}\hat{P} \ket{\pi^+\pi^-} = (-1)^L \ket{\pi^+ \pi^-} \overset{\text{alle }L=0}{=} - \ket{\pi^+ \pi^-\pi^0}
\end{split}\end{align}
\newpage

\item \tb{$\hat{C}$- und $\hat{P}$-Transformatioen von Neutrinos}

\begin{figure}[!ht]
\centering
\includegraphics[width=.65\textwidth]{imgs/ep5-fig-8-10.pdf}
\caption{Skizze zur $C$- und $P$-Transformation \label{fig:8.10}}
\end{figure}

\begin{itemize}
\item[$\Ra$] $C$ ist maximal verletzt (wie $P$)
\item[$\Ra$] $CP$ könnte erhalten sein (fester \glqq Glaube\grqq{} bis 1964)
\end{itemize}
\item \tb{Entdeckung der $CP$-Verletzung} 1964 im $K^0$-System
\begin{align*}
\lno \begin{matrix}
\ket{K^0} = \ket{d\bar{s}}\\ \ket{K^0} = \ket{ s\bar{d}}
\end{matrix}\rrb\text{ können sich ineinander umwandeln}
\end{align*}

\begin{figure}[!ht]
\centering
\includegraphics[width=.5\textwidth]{imgs/ep5-fig-8-11.pdf}
\caption{Möglichkeiten zur $K^0$-Umwandlung \label{fig:8.11}}
\end{figure}

\begin{itemize}
\item[$\Ra$] Beobachtet werden $K^0$-$\bar{K}^0$-Mischungen
\item[$\Ra$] Konstruiere $CP$-Eigenzustände
\begin{align}
\lno \begin{matrix}
\hat{C} \ket{K^0} = \ket{\bar{K}^0}\\
\hat{P} \ket{K^0} = - \ket{K^0}
\end{matrix}\rrb \hat{C}\hat{P} \ket{K^0} = - \ket{\bar{K}^0}\nonumber \\
\lno \begin{matrix}
\hat{C} \ket{\bar{K}^0} = \ket{K^0}\\
\hat{P} \ket{\bar{K}^0} = - \ket{\bar{K}^0}
\end{matrix}\rrb \hat{C}\hat{P} \ket{\bar{K}^0} = - \ket{K^0}\nonumber \\
\boxed{\begin{matrix}
\ket{K_1} = \frac{1}{\sqrt{2}} \lb  \ket{K^0} - \ket{\bar{K}^0} \rb , & \hat{C}\hat{P} \ket{K_1} = \ket{K_1}\\
\ket{K_2} = \frac{1}{\sqrt{2}} \lb  \ket{K^0} + \ket{\bar{K}^0}\rb , & \hat{C}\hat{P}\ket{K_2} = - \ket{K_2}
\end{matrix}}
\end{align}
Wenn $CP$ erhalten.\\
$K_1$ zerfällt in $CP$-Eigenzustand mit $CP=+1$\\
$K_2$ zerfällt in $CP$-Eigenzustand mit $CP=-1$\\
Insbesondere:
\begin{align*}
K_1 \ra 2\pi \hspace*{1cm} &\lb \tau_1 = 0.9 \cdot 10^{-10}\,\mr{s}\rb \\
K_2 \ra 3 \pi \hspace*{1cm} &\underbrace{\lb  \tau_2 = 5.1\cdot 10^{-8}\mr{s}\rb }_{\substack{\text{Phasenraumeffekt}\\lb M_n - 2M_\pi \gg M_n - 3M_\pi)}}
\end{align*}
\tb{Experiment:}
\begin{figure}[!ht]
\centering
\includegraphics[width=.5\textwidth]{imgs/ep5-fig-8-12.pdf}
\caption{Skizze zum experimentellen Aufbau \label{fig:8.12}}
\end{figure}

Ergebnis:
\begin{align}
& \boxed{ K_2 \ra 2\pi \text{ mit } 0.23\,\% \text{ WS}}\\
& \Ra \ CP\text{-Verletzung}\nonumber
\end{align}
\item[$\lt$] $CP$-Verletzung auch in $B$- und $C$-Systemen
\item[$\lt$] $K^0$-Eigenzustände der schwachen WW:
\begin{align*}
\boxed{ \begin{matrix}
\ket{K_S} = \frac{1}{\sqrt{1+\epsi^2}} \lb  \ket{K_2} + \epsi \ket{K_1}\rb \\
\ket{K_L} = \frac{1}{\sqrt{1+\epsi^2}} \lb  \ket{K_1} - \epsi \ket{K_2}\rb 
\end{matrix}}
\end{align*}
Hierbei $S$ = Short, $L$ = Long und $\epsi = 0.0023$
\item[$\Ra$] $CP$-Verletzung bricht Symmetrie zwischen Materie und Antimaterie.\\
Aber: gemessener Effekt zu klein, um Materie Überschuss im Universum zu erklären.
\item[$\Ra$] Gemessene $CP$-Verletzung kann erklärt werden durch komplexe Phase in $CKMM$
\end{itemize}
\end{itemize}


\section{Neutraler Strom und Z-Boson}

\begin{itemize}
\item[$\lt$] $Z$-Austausch:

\begin{figure}[!ht]
\centering
\includegraphics[width=.5\textwidth]{imgs/ep5-fig-8-13.pdf}
\caption{Möglichkeiten des $Z$-Austausches \label{fig:8.13}}
\end{figure}

\item[$\lt$] Kopplungsstärke:
\begin{align}
\boxed{g_Z = \frac{g_W}{\cos \theta_W} = \frac{e}{\sin \theta \cos \theta}}
\end{align}
\item[$\lt$] Wenn keine $\nu$'s beteiligt sind, sind $Z^0$- und $\gamma$-Austausch beide möglich (vgl. Abb.\ref{fig:8.14}).

\begin{figure}[!ht]
\centering
\includegraphics[width=.5\textwidth]{imgs/ep5-fig-8-14.pdf}
\caption{Beispiel zum $Z$- und $\gamma$-Austausch \label{fig:8.14}}
\end{figure}

\begin{align*}
\Ra \ \sigma, \Gamma \sim \labs \mc{M}_Z + \mc{M}_\gamma \rabs^2
\end{align*}
\begin{itemize}
\item[$\Ra$] Interferenzterme, Unterscheidung $Z$- und $\gamma$-Austausch ist für einzelne Prozesse nicht möglich
\item[$\Ra$] Bei $Q^2 \ll M_Z^2$ dominiert $\gamma$-Austausch
\end{itemize}
\item[$\lt$] $Z$ koppelt an $\ket{\Psi_L}$ und $\ket{\Psi_R}$ (ladungsabhängig)\\
$Z$ koppelt nur an $\nu_L$, $\bar{\nu}_R$
\item[$\lt$] $Z$-Quark-Kopplungen:\\
\glqq Algebraisch\grqq{}
\begin{align*}
\begin{pmatrix}
d^\prime\\ s^\prime \\ b^\prime
\end{pmatrix}^\top \ \ \begin{pmatrix}
d^\prime \\ s^\prime \\ b^\prime
\end{pmatrix} = \begin{pmatrix}
d\\ s\\ b\end{pmatrix}^\top \underbrace{U_\mr{CMK}^\dagger U_\mr{CMK}}_{\mb{1}_3} \begin{pmatrix}
d \\ s\\ b
\end{pmatrix} = \begin{pmatrix}
d \\ s \\ b 
\end{pmatrix}^\top \begin{pmatrix}
d \\ s\\ b
\end{pmatrix}
\end{align*}
$\Ra$ \tb{Keine} \glqq Flavour-changing neutral currents\grqq{} (FCNC)

\begin{figure}[!ht]
\centering
\includegraphics[width=.5\textwidth]{imgs/ep5-fig-8-15.pdf}
\caption{Beispiel des Zerfalls $K^0 \ra \mu^+ \mu^-$ mit Branchingratio $BR=1.5\times 10^{-6}$ \label{fig:8.15}}
\end{figure}

\dots weil $\underbrace{\sim V_{du} V_{su}^\star + V_{dc}V_{sc}^\star + V_{dt}V_{st}^\star}_\text{Elemente der $CKMM$} =0$\\
(der sog. GIM-Mechanismus: Glashow, Iliopoulos, Maiami)\\
\dots trotzdem möglich, weil $m_u \ll m_c \ll m_t$
\end{itemize}

\section{Direkte Produktion von \texorpdfstring{$W$}- und \texorpdfstring{$Z$}-Bosonen}
\begin{itemize}
\item \tb{Geschichte:}
\begin{itemize}
\item[1968:] Theorie der elktroschwachen Wechselwirkung (Glashow, Weinberg, Salam Nobel-Preis 1979)
\item[1973:] Erster Nachweis von Reaktionen mit $Z$-Austausch ($\nu A \ra \nu A$, CERN)
\item[1983:] Direkter Nachweis von $Z,\ W$ (CERN) (NP 1984: Rubbia, van der Meer)
\item[199er:] $Z$-Präzisionsmessungen (CERN) 
\end{itemize}
\item \tb{Entdeckung von $W$ und $Z$}\\
$p - \bar{p}$-Collider Sp$\bar{\mr{p}}$S, CERN
\item \tb{$W$-Erzeugung}

\begin{figure}[!ht]
\centering
\includegraphics[width=.5\textwidth]{imgs/ep5-fig-8-16.pdf}
\caption{$W$ Erzeugung bei Wechselwirkung von Proton und Antiproton \label{fig:8.16}}
\end{figure}

\glqq Reste\grqq{} von $p$ und $\bar{p}$ (also $uu$ und $\bar{u}\bar{d}$) bilden Jets im Strahlrohr.\\
Das $W$ hat einen Impuls $E_p \lb xd - x\bar{u}\rb $ in Strahlrichtung
\item \tb{$W$-Zerfall:}
\begin{align*}
\begin{matrix}
\text{leptonisch } & \llb \begin{matrix} W^\pm \ra & e^+ \nu_e, e^- \bar{\nu}_e\\ & \mu^+ \nu_\mu, \mu^- \bar{\nu}_\mu\\ &\tau^+ \nu_\tau, \tau^- \bar{\nu}_\tau \end{matrix} \rrb & \text{ je 10.9\,\% (Lepton-Universalität)}\\ & & \\
\text{hadronisch } & W^\pm$ $\ra\ \underbrace{q \bar{q}^\prime}_{ud^\prime, cs^\prime} \ra  2\text{ jets} & 67.2\,\%
\end{matrix}
\end{align*}

$ud^\prime,\ cs^\prime$ schwache EZ, 3 mögliche Farben. Somit $2 \cdot 3 \cdot 11.2\,\%$\\
-- Achtung: $W^\pm \ra t+x,\bar{t}+x$ verboten wegen $m_t > M_W$
\item[$\ra$] \tb{$W$-Identifikation}\\
Zuerst in leptonischen Zerfällen
\begin{itemize}
\item \glqq missing $p_T$\grqq{} von $\nu$
\item einzelne Teilchenspur $(e)$
\item charkteristisches Signal im Kalorimeter $(e)$
\end{itemize}
\item[$\ra$] \tb{$W$-Eigenschaften}\\
$M_W= 80.385$\,GeV, $\Gamma_W = 2.085$\,GeV, $\tau = 3\times 10^{-25}$\,s
\item \tb{$Z$-Erzeugung: Wie $W$}

\begin{figure}[!ht]
\centering
\includegraphics[width=.5\textwidth]{imgs/ep5-fig-8-17.pdf}
\caption{$Z$-Erzeugung bei Proton-Antiproton-Streuung\label{fig:8.17}}
\end{figure}

\item \tb{$Z$-Zerfall}
\begin{align*}
\begin{matrix}
\text{leptonisch } & \llb Z^0 \ra  \begin{matrix}
\lno\begin{matrix}
e^+e^-\\ \mu^+\mu^- \\ \tau^+ \tau^-
\end{matrix}\rrb &\text{ je 3.4\,\% (Lepton-Universalität)}\\
\lno\begin{matrix}
\nu \bar{\nu}\\ \text{(invisible)\grqq{}}
\end{matrix}\rrb & 20\,\%
\end{matrix}\rno \\
\text{hadronisch: } & Z^0 \ra q\bar{q}\ \ra 2\text{ jets}\qquad \qquad \qquad 69.9\,\% \qquad \qquad
\end{matrix}
\end{align*}

$q\bar{q} = d\bar{d},u\bar{u},s\bar{s},c\bar{c},b\bar{b},\text{ nicht } t\bar{t}$
\item \tb{$Z$-Eigenschaften}\\
$M_Z = 91.1876$\,GeV, $\Gamma = 2.4952$\,GeV, $\tau \approx 2\times 10^{-25}$\,s
\item \tb{Präzisionsmessung des $Z$}\\
LEP: $e^+e^-$-Collider, CERN

\begin{figure}[!ht]
\centering
\includegraphics[width=.5\textwidth]{imgs/ep5-fig-8-18.pdf}
\caption{$e^+e^-$-WW mit Fermion-Antifermionenpaar \label{fig:8.18}}
\end{figure}

wenn $\Gamma_s = M_Z$ (d.h.$E_{e^-} = E_{e^+} = \frac{M_Z}{2}$) $\Ra$ $Z$ ruht im Laborsystem\\
WQ bei $\Gamma_s = M_Z$ ist groß, aber endlich. (Warum endlich?)
\begin{itemize}
\item $Z$ instabil
\begin{align}
\Psi_Z \sim e^{iM_Z} e^{-\frac{\Gamma_s t}{2}}
\end{align}
\item Im Propagator:
\begin{align}
 M_z \ra M_Z - i \frac{\Gamma_Z}{2}\\
 \Ra \labs \frac{1}{S-M_Z^2}\rabs^2 \ra \labs \frac{1}{S- M_Z^2 + i \frac{\Gamma_Z}{2} M_Z}\rabs\\
 = \frac{1}{\lb S- M_Z^2\rb ^2 + M_Z^2 \Gamma_Z^2}
\end{align}
$\Ra$ WQ $\lb e^+e^-\ra Z\rb $ hat Resonanz bei $\Gamma_S = M_Z$

\begin{figure}[!ht]
\centering
\includegraphics[width=.5\textwidth]{imgs/ep5-fig-8-19.pdf}
\caption{Diagramm zur $M_Z$-Resonanz \label{fig:8.19}}
\end{figure}


Bei LEP: $\sim 2 \times 10^7$ $Z$-Ereignisse (4 Experimente)\\
Bei SLAC: $\sim0.5\times 10^6$ $Z$-Ereignisse mit polarisierten $e^+, \ e^-$ (1 Experiment)
\item Präzisionsmessung der $Z$-Resonanz
\item Genaue Messung der elektroschwachen Parameter:\\
z.B. $M_Z$, $\Gamma_Z$, $\sin \theta_W$, Kopplungen, \dots
\begin{itemize}
\item[$\Ra$] $\boxed{\text{Alle Ergebnisse im Einklang mit dem Standardmodell}}$
\item[$\Ra$] Test von höheren Ordnungen!
\item[$\ra$] Besipiel: Bestimmung der Zahl leichter $\nu$-Flavours aus $\Gamma_Z$:
\begin{align}
\Aboxed{\Gamma_Z = \Gamma_\mr{had} + \Gamma_e + \Gamma_\mu + \Gamma_\tau + \underbrace{\Gamma_\nu}_{\sim N \nu}}
\end{align}
Messung von $\Gamma_Z$ aus:
\begin{align}
\Aboxed{\sigma_\mr{had} = \sigma\lb e^+e^- \ra \text{Hadronen}\rb   = A \cdot \frac{S}{\lb S-M_Z^2\rb ^2 + M_Z^2 \Gamma_Z^2}}
\end{align}
$A$ aus Theorie bekannt\\
$S$ von Beschleuniger\\
$M_Z$ von Resonanz-Maximum\\
$\lt$ $\sigma_\mr{had} \lb  \Gamma_S = M_Z \rb  \sim \frac{1}{\Gamma_Z^2}$\\
$\lt$ $\Gamma_S \approx M_Z$ mit MeV-Genauigkeit gemessen ($\Ra$ erfordert Korrekturen auf Gezeiten, Wasserstand in Genfer See, Züge, \dots )\\
$\lt$ Ergebnis:
\begin{align}
\Aboxed{N_\nu = 2.984 \pm 0.08}
\end{align}
\item Nicht besprochen: Strahlungskorrekturen

\begin{figure}[!ht]
\centering
\includegraphics[width=.65\textwidth]{imgs/ep5-fig-8-20.pdf}
\caption{Feynmandiagramm zu Nebenstrahlungen, die korrigiert werden müssen \label{fig:8.20}}
\end{figure}

$\lt$ berechnet mit QED
\end{itemize}
\end{itemize}
\item \tb{$e^+e^- \ra W^+W^-$}\\
LEP-Energie erhöht auf $\sim 110$\,GeV $\Ra \ e^+e^-\ra \underbrace{W^+W^-}_\text{reell}$ möglich

\begin{figure}[!ht]
\centering
\includegraphics[width=.5\textwidth]{imgs/ep5-fig-8-21.pdf}
\caption{Feynmandiagramm zur Wechselwirkung von $e^+e^- \ra W^+W^-$ über ein $Z$-Boson\label{fig:8.21}}
\end{figure}

Ergebnis: beide Graphen erforderlic!ht
\end{itemize}

%\vorlesung{08. Februar 2018}

\chapter{Massive Neutrinos}
\begin{itemize}
\item[$\ra$] Im Standardmodell: $m_\nu = 0$
\item[$\ra$] Aus $\nu$-Oszillationen: $m_\nu >0$ (mindestens ein $m_\nu > 0.05$\,eV)
\item[$\ra$] Aus direkten Messungen: $m_\nu < 2$\,eV\\
(Kosmologie: $\sum m_\nu \lesssim 0.12$\,eV)
\end{itemize}
\section{Massen- und Flavoureigenzustände von Neutrinos}
Flavour: $\nu$-EZ der schwachen WW
\begin{align*}
\boxed{\nu_e, \ \nu_\mu, \ \nu_\tau}
\end{align*}
Masse: Eigenzustände der Propagation
\begin{align*}
\boxed{ \begin{matrix}
\nu_1, & \nu_2, & \nu_3\\
\ua & \ua & \ua\\
m_1, & m_2, & m_3
\end{matrix} }
\end{align*}
Oszillationen $\Ra$ $\lb  \nu_e, \ \nu_\mu, \ \nu_\tau\rb  \neq \lb \nu_1, \ \nu_2, \ \nu_3\rb $\\
$\Ra$ Mischungmatrix (ähnlich wie bei Quarks)
\begin{align}
\boxed{ \begin{pmatrix}
\nu_e \\ \nu_\mu \\ \nu_\tau 
\end{pmatrix} = U \lb 3\rb  \cdot \begin{pmatrix}
\nu_1 \\ \nu_2 \\ \nu_3
\end{pmatrix}}
\end{align}
Parameter: 3 Winkel, 1 komplexe Phase (Pontecorro-Maki-Nakagawa-Sakata-Matrix, PMNS)

Aus Oszillationen: Messung von 
\begin{align}
\Delta^2 m_{ij} = m_i^2 - m_j^2
\end{align}
$\lt$ 2 unabhängige Werte
\begin{align}
\labs \Delta m_{23}^2 \rabs \approx 2.4 \cdot 10^{-3}\,\mr{eV^2}\\
\Delta m_{13}^2 \approx 7.6\cdot 10^{-5}\,\mr{eV^2}\\
\Ra \text{ ein } m_i \geq \sqrt{\labs m_{23}^2 \rabs} \approx 0.05\,\mr{eV}
\end{align}
Achtung: $\nu_e, \ \nu_\mu,\ \nu_\tau$ haben \tb{keine} wohldefinierte Masse!\\
Experimentelle Grenzen auf:
\begin{align}
\lla m^2 \lb \nu_e\rb  \rra = \sum \labs U_{e,i} \rabs^2 m_i^2 \qquad (\beta\text{-Zerfall})\\
\lla m_{\beta\beta} \rra = \labs \sum U_{e,i}^2 m_i \rabs \qquad (0\nu 2\beta)\\
\lla m_\mr{kosm.} \rra = \sum m_i \qquad (\text{Kosmologie})
\end{align}
\section{Direkte Messung der Neutrinomasse}
Erinnerung an \ref{sec:5.2}:\\
Messung von $\lla m^2 \lb \nu_e\rb \rra$ aus $e^-$-Spektrum bei $\beta$-Zerfall
\begin{figure}[!ht]
\centering
\includegraphics[width=.6\textwidth]{imgs/ep5-fig-9-1.pdf}
\captionof{figure}{Veränderung der Kurve aus Abb.\ref{fig:5.10} durch massebehaftete Neutrinos \label{fig:9.1}}
\end{figure}
\tb{Erforderlich:}
\begin{itemize}
\item Niedriger $Q$-Wert
\item $e^-$ werden ohne (stochstischen) Energieverlust freigesetzt
\item Hohe Zahl von Zerfällen
\item Extrem präzise Messung des Spektrums
\end{itemize}
Derzeit beste Lösung: Verwende Tritium-Zerfall
\begin{align}
\boxed{ ^3 H \ra ^3He + e^- + \bar{\nu}_e} \qquad Q = 18.6\,\mr{keV}
\end{align}
$\ra$ Spektrometer-MAC-E-Filter (magnetic adiabatic collimation, electrostatic)
\begin{figure}[!ht]
\centering
\includegraphics[width=.5\textwidth]{imgs/ep5-fig-9-2.pdf}
\captionof{figure}{Prinzip des MAC-E-Filters,  mit $\labs \vec{B}\rabs$ bei $A_\mr{min}$ im Bereich von Tesla\label{fig:9.2}}
\end{figure}
Die $e^-$ folgen den Feldlinien und alle solche $e^-$ mit $p_z >0$ werden \glqq eingefangen\grqq{}
\begin{align}
0 < E_\parallel = \frac{p_z^2}{2m_e} \leq Q
\end{align}
In der Mitte der Versuchsanordnung gilt $\labs \vec{B}\rabs$ klein:
\begin{align}
B_\mr{min} = \frac{A_\mr{min}}{A_\mr{max}} B_\mr{max}
\end{align}
Der $e^-$-Transport entlang $\vec{B}$ ist adiabatisch\\
$\Ra$ magnetisches Moent der Kreisbewegung um $\vec{B}$ erhalten\\
$e^-$-Beugung in Ebene $\perp\vec{B}$ erzeugt magnetisches Moment
\begin{align}
\mu = I \cdot a = \pi e \nu R^2
\end{align}
\begin{compactitem}
\item[mit] $I$: Strom $e \cdot \nu$
\item[] $\nu$: Umlauffrequenz
\item[] $a$: Fläche der Kreisbahn
\end{compactitem}
\begin{align}
F_L = F_Z \nonumber\\
e v_T B = \frac{m_e v_T^2}{R} = \frac{2E_T}{R}\\
\Ra \ \mu = \pi e\nu R^2 = \frac{E_T}{B} = const.\\
\Ra E_T \lb B_\mr{min}\rb  = \frac{B_\mr{min}}{B_\mr{max}} E_T \lb B_\mr{max}\rb  = \frac{A_\mr{min}}{A_\mr{max}} E_T \lb B_\mr{max}\rb \\
\text{mit } \Delta E = \frac{B_\mr{min}}{B_\mr{max}} \cdot Q \nonumber
\end{align}
$\Ra$ In Spektrometermitte sind alle $e^-$-Impulse $\parallel\vec{B}$\\
$\Ra$ Zähle $e^-$, die Gegenspannung $V = \frac{Q}{e} - \frac{\delta E}{e}$ überwinden.\\
$\delta E$ einstellbar von 0 bis einige eV.

Bisherige Ergebnisse verträglich mit $m_\nu = 0$. Obergrenze bisher:
\begin{align}
\boxed{ \lla m\lb \nu_e\rb \rra < 2\,\mr{eV} \ @ \ 95\,\% C.L. }
\end{align}
Zukunft: KATRIN-Experiment, Sensitivität 0.2\,eV

Achtung: Auch Einschränkung aus Kosmologie, $\lla m_\mr{kosm.} \rra \lesssim 0.12$\,eV\\
Aber: Modellabhängig
\newpage
\section{Neutrinooszillationen}
\begin{itemize}
\item \tb{Was ist das?}\\
Neutrino-Flavour wandelt sich \glqq im Flug\grqq{} um, \tb{keine} WW involviert
\begin{figure}[!ht]
\centering
\includegraphics[width=.5\textwidth]{imgs/ep5-fig-9-3.pdf}
\captionof{figure}{Schematische Neutrinooszillation\label{fig:9.3}}
\end{figure}
\item Wie funktionieren sie?\\
\tb{Beispiel:} 2-Flavour-Mischung im Vakuum
\begin{align}
\binom{\nu_1}{\nu_2} = \begin{pmatrix}
\cos \theta & - \sin \theta \\ \sin \theta & \cos \theta
\end{pmatrix}\binom{\nu_e}{\nu_\mu}
\end{align}
(nicht relativistisch, aber gut zur Demonstration des Mechanismus)
\begin{itemize}
\item[$\ra$] $t=0$: Produktion eines $\nu_e$
\begin{align}
\Ra \ \ket{\nu \lb t=0\rb } = \ket{\nu_e} = \ket{\nu_1} \cdot \cos \theta + \ket{\nu_2} \cdot \sin \theta
\end{align}
\item[$\ra$] Propagation im Vakuum bis $t=T$
\begin{align}
\ket{\nu\lb t=T\rb } = \ket{\nu_1} \cdot \cos \theta \cdot e^{-iE_1t} + \ket{\nu_2} \sin \theta e^{-iE_2t}
\end{align}
Hierbei entwickelt sich jeder Zustand entsprechend seiner Masse
\item[$\ra$] Amplitude für $\lno \nu_e \rabs_{t=0} \ra \lno \nu_e\rabs_{t=T}$
\begin{align}
A_e = \braket{\nu_e | \nu\lb t=T\rb } = \cos^2 \theta e^{-iE_1t} + \sin^2 \theta e^{-iE_2t}\\
\braket{\nu_i|\nu_j} = \delta_{ij} \nonumber
\end{align}
\item[$\ra$] Wahrscheinlichkeit für $\nu_e \ra \nu_e$:
\begin{align}\begin{split}
P_e \labs A\rabs^2 = A_e \cdot A_e^*\\
P_e = \cos^4 \theta + \sin^4 \theta + \sin^2\theta \cos^2 \theta \underbrace{\lsb  e^{-i\lb E_1-E_2\rb t} + e^{i\lb E_1-E_2\rb t} \rsb }_{2 \cos \lb \Delta E \cdot t\rb }\\
= 1- 2 \sin^2 \theta \cos^2 \theta \lsb 1- \cos \lb \Delta E \cdot t\rb \rsb 
\end{split}\end{align}
Im Folgenden wird in Näherung davon ausgegangen, dass die Impulse identisch sind
\begin{align}
\begin{split}
\Delta E &= \sqrt{p^2+m_1} - \sqrt{p^2 +m_2^2}\\
& = p\lsb  1+ \frac 12 \frac{m_1^2}{p^2} -1- \frac 12 \frac{m_2^2}{p^2}\rsb \\
& = \frac{m_1^2 -m_2^2}{2p} = \frac{\Delta m^2}{2p}
\end{split}
\end{align}
weiterhin gilt für die Näherungen $v_\nu\approx c$, $t=\frac{L}{\mr{c}}$ und $p=E$, dass
\begin{align}
\begin{split}
1- \cos \lb \Delta E \cdot t\rb  = 2 \sin^2 \lb \frac{\Delta E \cdot t}{2}\rb  = 2 \sin^2 \lb  \frac{\Delta m^2 \cdot t}{4p}\rb = 2 \sin^2 \lb \frac{\Delta m^2 \cdot L}{4E}\rb 
\end{split}
\end{align}
und es ist
\begin{align}
2 \sin^2 \theta \cos^2 \theta = \frac 12 \sin^2 \lb 2 \theta\rb \nonumber \\
\boxed{P_e = 1- \sin^2 \lb 2 \theta \rb  \sin^2 \lb  \frac{\Delta m^2 L}{4E}\rb }
\end{align}
\item[$\Ra$] Oszillation nur, wenn $\theta \neq 0$ \tb{und} $\Delta m^2 \neq 0$ !
\item[$\ra$] Oszillationslänge:
\begin{align}
L_0 = \frac{\pi}{\nicefrac{\Delta m^2}{4E}} = 2.48 \, \mr{\frac{\nicefrac{E}{MeV}}{\nicefrac{\Delta m^2}{eV^2}}\cdot m}
\end{align}
\end{itemize}
\item Woher wissen wir, dass es $\nu$-Oszillationen gibt?
\begin{enumerate}
\item Sonnenneutrinos

In Sonne: Kernfusion, dominater Zyklus:
\begin{align}
\begin{split}
2e^- + 4p \ra ^4\mr{He} + 2 \nu_e + 26.73\,\mr{MeV}\\
2\lla E_\nu\rra = 0.59\,\mr{MeV}
\end{split}
\end{align}
Diese Energie entspricht ungefähr 2\,\% der elmag Energie.

Neutrinofluss auf der Erde:
\begin{align*}
\Phi_{\nu_e} = 2\cdot \frac{S}{26.73\,\mr{MeV}} = 6.5\cdot 10^{10} \, \mr{cm^{-2}\cdots^{-1}}\\
S = 8.5\times 10^{11}\,\mr{\nicefrac{MeV}{cm^2s}}
\end{align*}
\begin{compactitem}
\item[mit] $S$: Solarkonstante = Strahlungsleistung der Sonne auf der Erde
\end{compactitem}
Zusätzlich: $\nu$'s von Fusionsreaktionen schwerer Kerne $\ra$ $E$-Spektrum bis $\sim\ 20\,\mr{MeV}$.

\tb{Nachweis:}
\begin{itemize}
\item[$\ra$] Radiochemische Experimente\\
Homestake (1970-94): $\nu_e +^{37}U \ra e^- + ^{37}Ag$\\
GALLEX, SAGE ('90er): $\nu_e + ^{71}Ga \ra e^- + ^{71}Ge$\\
chemische Extraktion, Nachweis über Zerfall
\item[$\ra$] Direkt\\
Kamiokanne (1985-95) und Super-Kamiokanne (1996- : $\nu_ee^- \ra \nu_ee^-$)\\
SNO (1999-2006) $\nu_ed \ra e^-pp$, $\nu_xd \ra np$

Ergebnisse:
\begin{itemize}
\item[$\lt$] weniger $\nu_e$ als von Sonnenmodell erwartet in allen Experimenten
\end{itemize}
Aber: SNO-Experiment findet erwartete $\nu$-Fluss in $\nu_e + \nu_\mu+\nu_\tau$
\begin{itemize}
\item[$\Ra$] $\boxed{\text{Oszillation } \nu_e \ra \nu_{\cancel{e}}}$
\end{itemize}
\end{itemize}
\item Atmosphärische Neutrinos

Werden in Reaktionen kosmischer Strahlung mit Atomkernen der Atmosphäre gebildet
\begin{align}
\begin{split}
(p \text{ oder } A) +A \ra \pi^\pm \pi \pi \dots\\
\pi^\pm \ra \mu^\pm + \nu_e \text{ oder } \bar{\nu}_e\\
\mu^\pm \ra e^\pm + (\nu_e + \nu_\mu \text{ oder } \bar{\nu}_e + \bar{\nu}_\mu)
\end{split}
\end{align}
Damit wird ein Verhältnis erwartet:
\begin{align}
\Ra \frac{\nu_\mu + \bar{\nu}_\mu}{\nu_e + \bar{\nu}_e} \approx 2
\end{align}
\begin{itemize}
\item[$\ra$] Gemessen:
\begin{itemize}
\item[$\ra$] $\nu_\mu$ \glqq von unten\grqq{} fehlen
\item[$\ra$] kein Überschuss an $\nu_e$
\item[$\Ra$] $\boxed{\text{Übergang } \nu_\mu \ra \nu_\tau}$
\end{itemize}
\end{itemize}
\item Reaktor-Neutrinos

Stammen aus Kernspaltung, d.h. Umwandlungen $n\ra p$
\begin{align}
\boxed{n \ra p + e^- + \bar{\nu}_e}
\end{align}
Mit $E_{\bar{\nu}_e}\approx \mc{O}\lb 1\,\mr{MeV}\rb $

\tb{Nachweis} über $\bar{\nu}_e + p \ra e^+ + n$\\
Experimente u.a. KAMLAND (Japan), Daya Bay (China), Double Chooz (Belgien)\\
Information zu: Übergänge $\nu_e \ra \nu_e$
\end{enumerate}
\item \tb{Stand der Messungen}
\begin{align}
\Delta m_{12}^2 = 7.4\times 10^{-5}\,\mr{eV^2}\\
\labs \Delta m_{23}^2\rabs \approx 2.5\times 10^{-3}\,\mr{eV^2}\\
\lno \begin{matrix}
\sin^2 \theta_{12} \approx 0.297\\ \sin^2 \theta_{13} \approx 0.021 \\ \sin^2 \theta_{23} \approx 0.5
\end{matrix}\rrb \text{ PMNS-Matrix nicht diagonal-dominant}
\end{align}
Noch unbekannt:
\begin{itemize}
\item[$\lt$] Vorzeichen von $\Delta m_{23}^2$
\item[$\lt$] komplexe Phase
\item[$\lt$] absoluter Wert von $m_1$, $m_2$, $m_3$
\end{itemize}
\newpage

\item \tb{Neutrinooszillation in Materie}

zusätzliche Feynman-Graphen für $\nu_e$, $\bar{\nu}_e$
\begin{figure}[!ht]
\centering
\includegraphics[width=.8\textwidth]{imgs/ep5-fig-9-4.pdf}
\caption{Feynmandiagramm für die Oszillation in Materie, wobei für Elektronen-(Anti-)Neutrinos auch unter Austausch eines $W^-$-Bosons wechselwirken können.}
\end{figure}
\begin{itemize}
\item[$\Ra$] unterschiedliche Amplitude für Vorwärtsstreuung
\item[$\ra$] unterschiedlicher Brechungsindex
\item[$\ra$] modifiziert $\nu$-Oszillation
\item[$\ra$] Entscheidend für Oszillationen von Sonnenneutrinos
\end{itemize}
\end{itemize}


\listoffigures
\end{document}

